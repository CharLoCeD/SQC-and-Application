\documentclass[12pt]{amsart}

\usepackage{amsmath,amssymb,amsthm,amsfonts}
\usepackage[numbers]{natbib}
\usepackage[english]{babel}
\usepackage[X2,T1]{fontenc}
\usepackage{libertine}
\usepackage{libertinust1math}
\usepackage[utf8]{inputenc} %'utf8' instead of 'latin1'
\usepackage{enumitem} %modified eunumerate environment
\usepackage{tikz-cd}
\usepackage{hyperref}
\usepackage{stmaryrd}
\usepackage{quiver}
\usepackage{appendix}
\usepackage{mathtools}



\newtheorem{theorem}{Theorem}[section]
\newtheorem{fact}[theorem]{Fact}
\newtheorem{lemma}[theorem]{Lemma}
\newtheorem{conjecture}[theorem]{Conjecture}
\newtheorem{corollary}[theorem]{Corollary}
\newtheorem{claim}[theorem]{Claim}
\newtheorem{proposition}[theorem]{Proposition}
\newtheorem{observation}[theorem]{Observation}

\theoremstyle{definition}
\newtheorem{example}[theorem]{Example}
\newtheorem{definition}[theorem]{Definition}
\newtheorem{remark}[theorem]{Remark}
\newtheorem{question}[theorem]{Open Question}
\newtheorem{assumption}[theorem]{Assumption}
\newtheorem{qquestion}[theorem]{Question}
\newtheorem{axiom}{Axiom}

\newcommand{\mc}[1]{\mathcal{#1}}
\newcommand{\mb}[1]{\mathbf{#1}}
\newcommand{\mbb}[1]{\mathbb{#1}}
\newcommand{\T}{\mbb T}
\newcommand{\I}{\mbb I}
\newcommand{\gn}[1]{\ulcorner\! #1 \!\urcorner}
\newcommand{\mr}[1]{\mathrm{#1}}
\newcommand{\mf}[1]{\mathfrak{#1}}
\newcommand{\ms}[1]{\mathsf{#1}}
\newcommand{\Ind}{\mathbf{Ind}}
\newcommand{\Pro}{\mathbf{Pro}}
\newcommand{\Set}{\mb{Set}}
\newcommand{\Prop}{\mb{Prop}}
\newcommand{\sSet}{\mb{sSet}}
\newcommand{\sSeto}{\mb{sSet}_{\le 1}}
\newcommand{\End}{\operatorname{End}}
\newcommand{\Hom}{\operatorname{Hom}}
\newcommand{\Fin}{\mb{Fin}}
\newcommand{\Cnt}{\mb{Cnt}}
\newcommand{\Enm}{\mb{Enm}}
\newcommand{\Grp}{\mb{Grp}}
\newcommand{\sub}{\mr{Sub}}
\newcommand{\Lex}{\mb{Lex}}
\newcommand{\Pos}{\mb{Pos}}
\newcommand{\alg}{\text{-}\mb{Alg}}
\newcommand{\Var}{\mb{Var}}
\newcommand{\Aff}{\mb{Aff}}
\newcommand{\Vect}{\mb{Vect}}
\newcommand{\CRing}{\mb{CRing}}
\newcommand{\DL}{\mb{DL}}
\newcommand{\BA}{\mb{BA}}
\newcommand{\HA}{\mb{HA}}
\newcommand{\JL}{\mb{JL}}
\newcommand{\ML}{\mb{ML}}
\newcommand{\ProMon}{\mb{ProMod}}
\newcommand{\CTopoi}{\mb{CTopoi}}
\newcommand{\PTc}{\mb{PT}_c}
\newcommand{\Topoi}{\mb{Topoi}}
\newcommand{\sh}{\mb{Sh}}
\newcommand{\psh}{\mb{Psh}}
\newcommand{\Cont}{\mb{Cont}}
\newcommand{\Cart}[1]{#1^\to_{\text{cart}}}
\newcommand{\Str}{\mb{Str}}
\newcommand{\op}{^{\mathrm{op}}}
\newcommand{\inv}{^{\mathrm{-1}}}
\newcommand{\pf}[1]{\widehat{#1}}
\newcommand{\qsi}[1]{\tilde{#1}}
\newcommand{\cob}{\vartriangleleft}
\newcommand{\other}{\mathrm{otherwise}}
\newcommand{\geo}[1]{\left|#1\right|}
\newcommand{\ov}[1]{\overline{#1}}
\newcommand{\set}[1]{\{\,#1\,\}}
\newcommand{\eff}{\Leftrightarrow}
\newcommand{\conjt}{\;\&\;}
\newcommand{\pair}[1]{\left\langle#1\right\rangle}
\newcommand{\id}{\mathrm{id}}
\newcommand{\ev}{\mathrm{ev}}
\newcommand{\elem}{\int\!\!}
\newcommand{\nt}{\Rightarrow}
\newcommand{\scomp}[2]{\{\,#1\mid#2\,\}}
\newcommand{\yon}{\mathtt{y}}
\newcommand{\surj}{\twoheadrightarrow}
\newcommand{\inj}{\rightarrowtail}
\newcommand{\hook}{\hookrightarrow}
\newcommand{\cg}{\operatorname{\sim}}
\newcommand{\im}{\operatorname{img}}
\newcommand{\cgf}[2]{\leftindex_{#1}{\cg}_{#2}}
\newcommand{\coten}{\pitchfork}
\newcommand{\ppr}{\operatorname{\hat\times}}
\newcommand{\rl}{^{\perp}}
\newcommand{\llo}[1]{\leftindex_{}^{\perp} {#1}}
\newcommand{\dl}{^{\circ}}
\newcommand{\prth}[1]{\left(#1\right)}
\newcommand{\fn}{_{\mr{f.}}}
\newcommand{\fp}{_{\mr{f.p.}}}
\newcommand{\fpp}{_{\mr{f.p.,+}}}
\newcommand{\cp}{_{\mr{c.p.}}}
\newcommand{\cpp}{_{\mr{c.p.,+}}}
\newcommand{\pls}{^+}
\newcommand{\mns}{^-}
\newcommand{\dv}{\operatorname{\uparrow}}
\newcommand{\cv}{\operatorname{\downarrow}}
\newcommand{\et}{_{\text{\'et}}}
\newcommand{\N}{\mb N}
\newcommand{\Q}{\mbb Q}
\newcommand{\Z}{\mbb Z}
\newcommand{\Deltao}{\Delta_{\le 1}}
\newcommand{\Deltaw}{\Delta_{\omega}}
\newcommand{\sk}{\ms{sk}}
\newcommand{\csk}{\ms{csk}}
\newcommand{\sInt}{\mb{sInt}}
\newcommand{\jcan}{J_{\mr{can}}}
\newcommand{\wCPO}{\omega\mb{CPO}}
\newcommand{\shape}{\operatorname{\smallint}}
\newcommand{\dneg}{\neg\neg}
\newcommand{\prt}{_{\bot}}
\newcommand{\cprt}{_{\top}}
\newcommand{\fa}[2]{\forall #1\!\colon\!\!#2.\ }
\newcommand{\ex}[2]{\exists #1\!\colon\!\!#2.\ }
\newcommand{\exu}[2]{\exists_! #1\!\colon\!\!#2.\ }
\newcommand{\ld}[2]{\lambda #1\!\colon\!\!#2.\ }
\newcommand{\subopen}{\subseteq_{\mbb \I}}
\newcommand{\emp}{\emptyset}
\newcommand{\eq}{\leftrightarrow}
\newcommand{\ass}[1]{\llbracket#1\rrbracket} %\usepackage{stmaryrd}
\newcommand{\pss}[1]{||#1||} %\usepackage{stmaryrd}
\newcommand{\tp}{\ms{Type}}
\newcommand{\pp}{\ms{Prop}}
\newcommand{\st}{\ms{Set}}
\newcommand{\cnt}{\ms{Cnt}}
\newcommand{\gp}{\ms{Grpd}}
\newcommand{\pcat}{\ms{PCat}}
\newcommand{\cat}{\ms{Cat}}
\newcommand{\pcatt}{\ms{PCAT}}
\newcommand{\catt}{\ms{CAT}}
\newcommand{\PCat}{\mb{PCat}}
\newcommand{\Cat}{\mb{Cat}}
\newcommand{\PCAT}{\mb{PCAT}}
\newcommand{\CAT}{\mb{CAT}}
\newcommand{\Catt}{\mf{Cat}}
\newcommand{\CATT}{\mf{CAT}}
\newcommand{\Top}{\mf{Top}}
\newcommand{\Tp}{\ms{TYPE}}
\newcommand{\Pp}{\ms{PROP}}
\newcommand{\St}{\ms{SET}}
\newcommand{\Gp}{\ms{GRPD}}
\newcommand{\fst}{\ms{Fin}}
\newcommand{\quot}[1]{/_{\pair{#1}}}
\newcommand{\List}{\ms{List}}
\newcommand{\hp}{\text{-}}
\newcommand{\PG}{\ms{PG}}
\newcommand{\uv}[1]{\underline{#1}}
\newcommand{\mmod}[1]{#1\text{-}\mathbf{Mod}}
\newcommand{\func}{\mb{Func}}
\newcommand{\tm}[1]{#1\text{-}\mathrm{Term}}
\newcommand{\eqn}[1]{#1\text{-}\mathrm{Eqn}}
\newcommand{\horn}[1]{#1\text{-}\mathrm{Horn}}
\newcommand{\gr}[2]{[#1|#2]}
\newcommand{\VT}{\mbb V_\T}
\newcommand{\spec}{\operatorname{Spec}}
\newcommand{\El}{\mr{El}}
\newcommand{\lan}{\ms{lan}}
\newcommand{\ran}{\ms{ran}}
\newcommand{\upp}{_{\ms U}}
\newcommand{\dsg}[1]{\!\pair{#1}}
\DeclareFontFamily{U}{dmjhira}{}
\DeclareFontShape{U}{dmjhira}{m}{n}{ <-> dmjhira }{}
\DeclareRobustCommand{\yon}{\text{\usefont{U}{dmjhira}{m}{n}\symbol{"48}}}
\DeclareRobustCommand{\noy}{\text{\reflectbox{\yon}}\!}

\makeatletter
\newcommand{\ct@}[2]{%
  \vtop{\m@th\ialign{##\cr
    \hfil$#1\operator@font lim$\hfil\cr
    \noalign{\nointerlineskip\kern1.5\ex@}#2\cr
    \noalign{\nointerlineskip\kern-\ex@}\cr}}%
}
\newcommand{\ct}{%
  \mathop{\mathpalette\ct@{\rightarrowfill@\textstyle}}\nmlimits@
}
\makeatother
\makeatletter
\newcommand{\lt@}[2]{%
  \vtop{\m@th\ialign{##\cr
    \hfil$#1\operator@font lim$\hfil\cr
    \noalign{\nointerlineskip\kern1.5\ex@}#2\cr
    \noalign{\nointerlineskip\kern-\ex@}\cr}}%
}
\newcommand{\lt}{%
  \mathop{\mathpalette\lt@{\leftarrowfill@\textstyle}}\nmlimits@
}
\makeatother


\title{Quasi-coherence is (almost) all you need for synthetic domain theory}
\author{Lingyuan Ye}
\date{\today}



\begin{document}
%

%
%\titlerunning{Abbreviated paper title}
% If the paper title is too long for the running head, you can set
% an abbreviated paper title here
%

%
% \Endhorrunning{L. Ye}
% First names are abbreviated in the running head.
% If there are more than two authors, 'et al.' is used.
%
% \institute{New College\\

%
\maketitle              % typeset the header of the contribution
%

\section{Introduction}

Let $\T$ be a Horn theory. The type system we work with is based on the quasi-coherence principle available in the classifying topoi of subcanonical quotients of $\T$~\cite{blechschmidt2021using,blechschmidt2020general}. (The envelopping $\infty$-topoi of) such topoi provide models for the type theory we work with in this paper.

In this paper we will be working in a dependent type theory with a univalent universe $\tp$. Recall the notion of $h$-levels~\cite{hottbook}. For us the most important h-levels are -1 and 0, which are \emph{propositions} and \emph{sets}. In particular, we can define subuniverses $\pp$, $\st$.

We will also assume the existence of propositional truncation. This allows us to define existential quantifier of a family $P : X \to \pp$, 
\[ \ex x{X} P(x) := \pss{\sum_{x:X}P(x)} \] 
and disjunction of propositions,
\[ P \vee Q := \pss{P + Q}. \]
Note that by univalence, $\pp$ is already closed under dependent product. But to emphasis that the result is again a proposition, we may also write dependent products of propositions as $\fa xX P(x)$. 

The universe $\pp$ is used to define subtypes, i.e. we identify a subtype of $X$ as a family of propositions $A : X \to \pp$ indexed over $X$, where the projection $\sum_{x:X}A(x) \inj X$ is the inclusion of this subtype. We often write the sum as a collection,
\[ \scomp{x:X}{A(x)} := \sum_{x:X}A(x). \]
When $X$ in fact is a set, then any of its subtype is also a set, and we will emphasis by calling them \emph{subsets}.

Sets are used to define \emph{algebras} for a Horn theory. Given any Horn theory $\T$, by a \emph{$\T$-model} we will always mean a \emph{set} $A$ equipped with operations according to the signature of $\T$, such that it satisfies the axioms of $\T$. Since $A$ is a set, being a $\T$-model is again a proposition for a given family of operations.

At the end, we remark that $\N$ is the usual inductive type of natural numbers. For any $n:\N$, we will also use $n$ to denote the finite type with $n$ elements, which is viewed as an initial segment of $\N$. We stress that this is only syntactic sugar, and can be made more precise if need be.


\section{Quasi-coherence and affine spaces}

In this section we discuss the notion of quasi-coherent algebra and affine spaces. The technical results presented in this section is essentially contained in~\cite{Cherubini_Coquand_Hutzler_2024}, based on the work in~\cite{blechschmidt2021using}. 

As a starting point, we assume there is a generic $\T$-model in our type system:

\begin{axiom}
  There is a set $\I$ equipped with a $\T$-model structure.
\end{axiom}

\emph{Quasi-coherence} is a condition on \emph{$\I$-algebras}: An $\I$-algebra is a $\T$-model $A$ equipped with a homomorphism $\I \to A$. The type of $\I$-algebras will be denoted as $\I\alg$. More generally, for any $\T$-algebras $A,B$, we will use $\I\alg(A,B)$ to denote the set of homomorphisms of $\I$-algebras from $A$ to $B$. For instance, this can be used to define the \emph{spectrum} $A$,
\[ \spec A := \I\alg(A,\I). \]

A lot of sets can be naturally viewed as the spectrum of some $\I$-algebra. For the simplest example, since $\I$ is the initial $\I$-algebra,
\[ \spec \I = \I\alg(\I,\I) = 1. \]
Now if we use $\I[x]$ to denote the free $\I$ algebra on one generator, we can identify $\I$ itself as a spectrum,
\[ \spec \I[x] = \I\alg(\I[x],\I) = \I. \]
However, notationally we will distinguish $\I$ as an algebraic gagdet and as a spectrum, which we view as a geometrical object. The latter will be denoted as $\I$, and the equivalence $\I \to \I$ takes $i : \I$ to the evaluation map $\ev_i : \I[x] \to \I$. More generally, we have:

\begin{example}\label{exm:cubeaffine}
  For any $n:\N$, the $n$-th cube $\I^n$ is the following spectrum,
  \[ \spec \I[n] = \I\alg(\I[n],\I) = \I^n. \]
\end{example}

\begin{definition}
  $A$ is \emph{quasi-coherent}, if the canonical evaluation
  \[ A \to \I^{\spec A} \]
  is an equivalence of $\I$-algebras, where $\I^{\spec A}$ has the point-wise $\I$-algebra structure. We write $\ms{isQC}(A)$ for the proposition of being quasi-coherent.
\end{definition}

\begin{example}\label{exm:intervalqc}
  $\I$ itself by definition is quasi-coherent. We have seen that $\spec \I = 1$. Under this equivalence, the canonical map 
  \[ \I \to \I^{\spec \I} = \I \]
  is exactly the identity on $\I$, hence is an equivalence.
\end{example}

Note by construction, $\spec$ is a contravariant functor on $\I$-algebras. For quasi-coherent algebras, we have a more general duality result:

\begin{proposition}\label{prop:duality}
  If $B$ is quasi-coherent, then for any $\I$-algebras $A$, we have
  \[ \I\alg(B,A) = \spec B^{\spec A}. \]
\end{proposition}
\begin{proof}
  By quasi-coherence of $B$, we have the following equivalences,
  \[ \I\alg(A,B) = \I\alg(A,\I^{\spec B}) = \I\alg(A,\I)^{\spec B} = \spec A^{\spec B}. \]
  The second identity holds since by quasi-coherence the $\I$-algebra structure on $\I^{\spec B}$ is point-wise.
\end{proof}

The above duality result tells us that we can infer equivalence of algebras via equivalence of their spectrum:

\begin{corollary}\label{cor:dualityeqv}
  For any $f : \I\alg(A,B)$, if $A,B$ are quasi-coherent, then $f$ is an equivalence iff it induces an equivalence $\spec B \to \spec A$. 
\end{corollary}

To illustrate the usefulness of the above duality result, let us also introduce the following notion: 
\begin{definition}
  We say an $\I$-algebra $A$ is \emph{stably quasi-coherent} if all finitely generated principle congruences of $A$ are also quasi-coherent, i.e. for any finite $n : \N$ and for any $a,b : n \to A$, $A/a = b$ is again quasi-coherent.
\end{definition}
 
By definition, any finitely generated principle congruences of a stably quasi-coherent algebra will again be stably quasi-coherent. For a stably quasi-coherent $\I$-algebra $A$,  its spectrum always ``has enough points'' to distinguish elements:

\begin{lemma}\label{lem:completeness}
  For any stably quasi-coherent $\I$-algebra $A$ and for any $a,b : A$,
  \[ a = b \eq \fa x{\spec A} xa = xb. \]
\end{lemma}
\begin{proof}
  The $\nt$ direction is trivial. For $\Leftarrow$, by the universal property of the quotient $A/a = b$, we have
  \[ \spec(A/a = b) = \scomp{x : \spec A}{xa = xb} \subseteq \spec A. \]
  If $\fa x{\spec A}xa = xb$ holds, then the inclusion $\spec(A/a = b) \inj \spec A$ is an equivalence. Since by assumption $A/a=b$ is also quasi-coherent, by Corollary~\ref{cor:dualityeqv} the quotient $A \surj A/a=b$ is an equivalence. This exactly says $a = b$.
\end{proof}

Dually, we can use quasi-coherence to define the notion of \emph{affine spaces}:

\begin{definition}
  We say a set $X$ is \emph{affine}, if $\I^X$ is quasi-coherent, and the canonical evaluation map
  \[ X \to \spec \I^X \]
  is an equivalence. Similarly, we say $X$ is \emph{stably affine} if $\I^X$ is furthermore stably quasi-coherent.
\end{definition}

\begin{proposition}
  For any set $X$, $X$ is affine iff there exists a quasi-coherent $A$ such that $X = \spec A$.
\end{proposition}
\begin{proof}
  The only if direction holds by definition, since $A$ can be taken as $\I^X$. For the if direction, it suffices to show $\spec A$ is affine whenever $A$ is quasi-coherent. Now by quasi-coherence, $A = \I^{\spec A}$ thus $\I^{\spec A}$ is quasi-coherent. Furthermore, the canonical map $\spec A \to \spec \I^{\spec A}$ is an equivalence again by quasi-coherence of $A$.
\end{proof}

At the end of this section, we describe finite limits of affine spaces. By Example~\ref{exm:intervalqc}, the terminal type $1$ is affine. More generally, pullbacks of affine spaces can be computed as a spectrum:

\begin{proposition}\label{prop:pullbackofaffine}
  Let $A,B,C$ be $\I$-algebras. Given $f : \I\alg(B,A)$ and $g : \I\alg(B,C)$, the pullback of $\spec f$ and $\spec g$ is given by the spectrum
  \[ \spec A \times_{\spec B} \spec C = \spec (A \otimes_B C), \]
  where $A \otimes_B C$ is the following pushout of $\I$-algebras,
  \[
    \begin{tikzcd}
      B \ar[r, "f"] \ar[d, "g"'] & A \ar[d] \\
      C \ar[r] & A \otimes_B C
      \arrow["\lrcorner"{anchor=center, pos=0.125, rotate=180}, draw=none, from=2-2, to=1-1]
    \end{tikzcd}
  \]
\end{proposition}
\begin{proof}
  This essentially holds by the universal property of the pushout.
\end{proof}

\begin{remark}
  Notice that in the above result, if $\spec B$ is affine, then by Proposition~\ref{prop:duality} any map from $\spec A$ to $\spec B$ is induced by an $\I$-algebra morphism, and similarly for $\spec C$. Thus, any such pullback will be computed as a spectrum.
\end{remark}

Following~\cite{Cherubini_Coquand_Hutzler_2024}, the above result provides an extremely easy formula to compute the coproducts of quasi-coherent algebras:

\begin{corollary}\label{cor:tensorasspace}
  For any $\I$-algebra $A,B$, if $B$ is quasi-coherent, then there is an equivalence
  \[ \I^{\spec A \otimes B} = B^{\spec A}. \]
\end{corollary}
\begin{proof}
  Since $1$ is affine, thus $\spec A \times \spec B$ can be computed as $\spec A \otimes B$. Then by quasi-coherence of $B$, we have
  \[ \I^{\spec(A \otimes B)} = \I^{\spec A \times \spec B} = B^{\spec A}. \]
  Following this equivalence, the evaluation map $A \otimes B \to \I^{\spec A \otimes B}$ is exactly the one given above.
\end{proof}

Under the above identification, if $B$ is quasi-coherent, then the canonical map $A \otimes B \to \I^{\spec A \otimes B}$ can be equivalently viewed as the following map 
\[ c \mapsto \ld{x}{\spec A} (x\otimes B)(c) : A \otimes B \to B^{\spec A}. \]
Of course when $A \otimes B$ is quasi-coherent, then the above map will again be an equivalence. We can similarly calculate the canonical inclusions. For the left inclusion $A \to A \otimes B$, given any $a : A$ and $x : \spec A$,
\[ (x \otimes B)(a \otimes 1) = xa, \]
where we view $xa : \I$ as an element in $B$ using its an $\I$-algebra. For the right inclusion $B \to A \otimes B = B^{\spec A}$, for any $b : B$ and $x : \spec A$,
\[ (x \otimes B)(1 \otimes b) = b, \]
which implies the image of $b$ is simply the constant function on $b$.


\section{Open propositions and dominance}\label{sec:dominance}

Notice that up till this piont we have not used any spectial property of $\T$ rather than the fact that it is a Horn theory. However, to move closer to the intended application in synthetic domain theory, we start by assuming our theory $\T$ is \emph{propositional}: 

\begin{definition}
  We say a Horn theory $\T$ is \emph{propositional}, if it extends the theory of meet-semi-lattices, and truth of an element is computed by slicing: For any $\T$-model $A$ and $a:A$, the quotient $A/a=1$ is given by
  \[ a \wedge - : A \surj A/a, \]
  where $A/a$ by definition is $\cv a := \scomp{b:A}{b\le a}$.
\end{definition}

\begin{remark}
  The theory $\mbb M$ of meet-semi-lattices, $\mbb D$ of distributive lattices, $\mbb H$ of Heyting algebras, and $\mbb B$ of Boolean algebras are all examples of propositional theories in the above sense. In fact, all finitary quotients of $\mbb H$ or $\mbb B$ will again be propositional. More generally, for any propositional theory $\T$ and any $\T$-model $D$, the theory of $D$-algebras will again be propositional. We call such a theory a theory of $\T$-algebra.
\end{remark}

For a propositional theory $\T$, we think of the generic model $\I$ as certain interval object, since it is equipped with a partial order. For propositional theories, more sets can be realised as spectrums. The important examples are \emph{simplices}:

\begin{example}\label{exm:simplicesaffine}
  For any $n : \N$, let $\Delta^n \subseteq \I^n$ be the following subset,
  \[ \Delta^n := \scomp{i : n \to \I}{i_1 \ge \cdots \ge i_n}. \]
  This type is indeed a spectrum, since by definition we have
  \[ \Delta^n = \spec\I[i_1,\cdots,i_n]/i_1\ge\cdots\ge i_n. \]
  To simplify later discussion, we might also introduce the following types isomorphic to simplices above,
  \[ \Delta_n := \scomp{i : n \to \I}{i_1 \le \cdots \le i_n}. \]
\end{example}

The constant $1 : \I$, which is the top element in $\I$, induces a predicate
\[ \ms t : \I \to \pp, \]
which takes $i : \I$ to the proposition $i = 1$. The first observation is that $\ms t$ takes $i : \I$ to a spectrum:

\begin{lemma}\label{lem:openpropaffine}
  For any $i : \I$, $\ms ti = \spec \I/i$.
\end{lemma}
\begin{proof}
  By definition, $\spec \I/i = \I\alg(\I/i,\I)$. Since $\I \surj \I/i$ is a quotient, there is a homomorphism from $\I/i$ to $\I$ iff $i = 1$, and in this case the map is unique.
\end{proof}

To say something about the propositions that lies in $\ms t$, we assume a minimal amount of axioms:

\begin{axiom}[SQCI]
  $\I$ is stably quasi-coherent.
\end{axiom}

We already know that $\I$ is quasi-coherent. If it is stably so, then in particular for any $i : \I$, the principle congruence $\I/i$ will also be (stably) quasi-coherent. This implies the following result:

\begin{lemma}[SQCI]\label{lem:intconserve}
  The interval $\I$ is conservative,
  \[ \fa{i,j}{\I} (\ms ti \eq \ms tj) \eq i = j. \]
\end{lemma}
\begin{proof}
  It suffices to show
  \[ \fa{i,j}{\I}(\ms ti \to \ms tj) \to i \le j. \]
  Take $i,j$ with $\ms ti \to \ms tj$. By Lemma~\ref{lem:openpropaffine} and (SQCI), each $\ms ti$ and $\ms tj$ will be affine. Then we get a restriction map between $\I$-algebras,
  \[ \I/j = \I^{\ms tj} \to \I^{\ms ti} = \I/i. \]
  By the universal property of the quotient and the fact that $\T$ is a propositional theory, such a map exists iff $i \le j$, which is the desired result.
\end{proof}

Thus, under (SQCI), we may view the generic algebra $\I$ as a subuniverse of \emph{open} propositions via the embedding $\ms t$:

\begin{definition}[SQCI]
  Given a proposition $p$, we say it is \emph{open} if $p$ is of the form $\ms ti$ for some $i$ in $\I$,
  \[ \ms{isopen}(p) := \sum_{i:\I}p \eq \ms ti. \]
\end{definition}

By Lemma~\ref{lem:intconserve}, the $i$ that $p \eq \ms ti$ will be unique, thus being open is a proposition. Also, open propositions are evidently closed under finite conjunctions, since $\ms t$ preserves them. More generally, we may define the notion of open subtypes:

\begin{definition}[SQCI]
  A subtype $U$ of $X$ is \emph{open}, if for any $x:X$ the proposition $U(x)$ is open,
  \[ \ms{isOpen}(U) := \fa xX \ms{isopen}(U(x)). \]
\end{definition}

In other words, an open proposition is equivalently a map $U : X \to \I$, where the subtype is classified as the following pullback,
\[
\begin{tikzcd}
  U \ar[d, hook] \ar[r] & 1 \ar[d] \\ 
  X \ar[r] & \I
  \arrow["\lrcorner"{anchor=center, pos=0.125}, draw=none, from=1-1, to=2-2]
\end{tikzcd}
\]
This also implies that open subtypes are closed under pullbacks. Note that this crucial depends on conservativity of $\I$. 

If $X$ is affine, open subtypes are easy to classify:

\begin{lemma}[SQCI]\label{lem:openofaffinegivesalgebra}
  Let $X = \spec A$ be affine. Then any open subset of $X$ is of the form $D_a$ for some $a:A$, where
  \[ D_a := \scomp{x : X}{\ms t(xa)} = \spec A/a. \]
  If $X$ is stably affine, then an open subset of it is again stably affine.
\end{lemma}
\begin{proof}
  This follows from the fact that the type of opens of $X$ is $\I^X = A$ since $X$ is affine. If $X$ is stably affine, then so is $D_a$.
\end{proof}

One observation from~\cite{Cherubini_Coquand_Hutzler_2024} is that the type of open propositions forms a \emph{dominance} in the sense of~\cite{rosolini1986continuity}. In other words, open propositions are closed under truth and dependent product. The \emph{loc. cit.} uses certain choice principle on affine spaces to show this. However, in our situation the conservativity of the interval $\I$ allows direct computation of the type of open subsets of affine spaces as shown above. Hence, our proof does not require any additional choice principle:

\begin{proposition}[SQCI]\label{prop:Idominance}
  The type of opens $\I$ forms a dominance.
\end{proposition}
\begin{proof}
  Suppose $p$ is an open proposition. By definition $p = \ms ti$ for some $i:\I$. Now if $q$ is an open subset of $p$, since $p = \ms ti$ is affine, Lemma~\ref{lem:openofaffinegivesalgebra} implies that $q = D_j$ for some $j : \I/i$. Since $\T$ is propositional, equivalently $j$ can be viewed as an element $j : \I$ with $j \le i$. This way, $q = \ms tj$.
\end{proof}

\begin{corollary}\label{cor:opentransitive}
  Open subtypes are transitive.
\end{corollary}


\section{Lifting}\label{sec:lifting}

Given then dominance structure $\I$, we can construct internally the lifting functor. For any type $X$, its lift is given by
\[ X\prt := \sum_{i:\I}X^{\ms ti}. \]
The functoriality is easy to express: For any $f : X \to Y$, we have
\[ f\prt(i,x) := (i,\ld{w}{\ms ti}fxw). \]
There is an evident unit $\eta\prt : X \to X\prt$, where
\[ \eta := \ld x X(1,\ld\hp 1 x). \]
The dominance structure on $\I$ also gives a multiplication $\mu : (X\prt)\prt \to X\prt$, where $\mu$ takes any $(i,u)$ with $i : \I$ and $u : \ms ti \to X\prt$ first to $(j,x)$, where $j$ is the dependent sum
\[ j := \sum_{w:\ms ti} (uw)_0, \]
and $x : \ms tj \to X$ is the partial element such that for $w : \ms ti$ and $v : (uw)_0$
\[ x(w,v) := (uw)_1(v). \]

\begin{example}
  By definition, it is easy to see that
  \[ 1\prt = \sum_{i:\I}\ms ti = \I. \]
\end{example}

For synthetic domain theory, the object of particular importance is the lift of the type of open propositions $\I$ itself. In fact, the lifting of any stably quasi-coherent $\I$-algebra can be computed fairely explicitly:

\begin{lemma}\label{lem:liftingofalgebra}
  If $A$ is stably quasi-coherent, then we have
  \[ A\prt = \scomp{i : \I, a : A}{a \le i}. \]
\end{lemma}
\begin{proof}
  Note for any $i : \I$, the quotient $A/i$ is again quasi-coherent. Now notice that we do have
  \[ A/i = A \otimes \I/i. \]
  By Corollary~\ref{cor:tensorasspace} and quasi-coherence of $A/i$, 
  \[ A/i = A \otimes \I/i = A^{\spec \I/i} = A^{\ms ti}. \]
  This way, it follows that 
  \[ A\prt = \sum_{i:\I}A^{\ms ti} = \sum_{i:\I}A/i = \scomp{i : \I, a : A}{a \le i}. \qedhere \]
\end{proof}

\begin{corollary}[SQCI]
  For the interval $\I$, $\I\prt = \Delta^2$.
\end{corollary}
\begin{proof}
  By Lemma~\ref{lem:liftingofalgebra}, 
  \[ \I\prt = \scomp{i,j : \I}{i \ge j} = \Delta^2. \]
  The second equivalence again uses the fact that $\T$ is propositional.
\end{proof}

\begin{remark}
  One interesting thing to notice here is that, though $\I\prt$ by computation is a dependent sum of algebras, it is naturally equivalent to a \emph{spectrum}, which is a geometric object. In some sense the source is the assumption that $\T$ is propositional, which allows us to identify the algebraic object $\I/i$ as a subset $\scomp{j : \I}{j \le i}$.
\end{remark}


More generally, for domain theoretic applications we would want to compute the liftings of the simplices introduced as spectrums in Example~\ref{exm:simplicesaffine}. It is first of all easy to see that the lifting of any affine space is also straight forward to compute by duality:

\begin{lemma}\label{lem:liftofaffine}
  If $X = \spec A$ is affine, then the lifting of $X$ is given by
  \[ X\prt = \sum_{i:\I}\I\alg(A,\I/i). \]
\end{lemma}
\begin{proof}
  By Proposition~\ref{prop:duality}, since $X = \spec A$ is affine,
  \[ X\prt = \sum_{i:\I}X^{\ms ti} = \sum_{i:\I}\spec A^{\spec \I/i} = \sum_{i:\I}\I\alg(A,\I/i). \qedhere \]
\end{proof}

Thus, motivated by domain theory, we find ourselves naturally move towards the following axiomatisation:

\begin{axiom}[SQCF]
  All finitely generated free $\I$-algebras, i.e. $\I[n]$ for $n : \N$, are stably quasi-coherent.
\end{axiom}

Equivalently, (SQCF) says that any finitely presented $\I$-algebra is (stably) quasi-coherent, where an $\I$-algebra is finitely presented if it is merely of the form $\I[n]/s=t$ with $s,t : m \to \I[n]$ for some finite $n,m$.

Of course, (SQCF) implies (SQCI) when taking $n$ to be 0. Furthermore, (SQCF) implies that the simplices $\Delta^n$ are now (stably) affine as well. This way, we can indeed compute their lifts:

\begin{lemma}[SQCF]
  For any $n : \N$, we have
  \[ \Delta^n\prt = \Delta^{n+1}. \]
\end{lemma}
\begin{proof}
  By Lemma~\ref{lem:liftofaffine}, for the affine space $\Delta^n = \spec\I[n]/i_1 \ge \cdots \ge i_n$,
  \begin{align*}
    \Delta^n\prt 
    &= \sum_{i:\I}\I\alg(\I[n]/i_1\ge\cdots\ge i_n,\I/i) \\
    &= \scomp{i,i_1,\cdots,i_n:\I}{i \ge i_1 \ge \cdots \ge i_n} \\
    &= \Delta^{n+1}
  \end{align*}
  The second equality again uses the fact that $\T$ is propositional.
\end{proof}

In particular, from a geometric perspective, the unit
\[ \eta : \Delta^n \inj \Delta^n\prt = \Delta^{n+1} \]
takes $i_1 \ge \cdots \ge i_n$ in $\Delta^n$ to $1 \ge i_1 \ge \cdots \ge i_n$ in $\Delta^{n+1}$. 

\section{Intrinsic order and a locality axiom}\label{sec:local}

Following~\cite{hyland2006first}, given the dominance $\I$ there is indeed a canonical way to define an order on any type $X$:

\begin{definition}
  The \emph{intrinsic order} on a type $X$ is defined as follows:
  \[ x \preceq y := \fa{U}{X\to\I} U(x) \le U(y). \]
\end{definition}

By definition, the intrinsic order is reflexive and transitive. As already observed in \emph{loc. cit.}, one important property of the intrinsic order is that \emph{every} map is monotone w.r.t. this order:

\begin{lemma}\label{lem:anymapmonotoneintriscorder}
  For $f : X \to Y$, $x \preceq y$ in $X$ implies $fx \preceq fy$ in $Y$.
\end{lemma}
\begin{proof}
  This simply follows from compositionality of functions.
\end{proof}

Just as how we have computed the lifting of affine spaces in Section~\ref{sec:lifting}, the quasi-coherence principle also determines the intrinsic order on affine spaces. For instance, if $X = \spec A$ is affine, then $\I^X = A$, which means for any $x,y : X$,
\[ x \preceq y \eq \fa aA xa \le ya. \]
In other words, the order $\preceq$ on an affine space is the point-wise order in $X = \I\alg(A,\I)$. In practice, we usually have an explicit presentation of the algebra $A$, and in this case we have a more explicit description:

\begin{lemma}\label{lem:orderonaffine}
  Let $A = \I[X]/R$ be a quasi-coherent $\I$-algebra for some generating set $X$ and relation $R$. Then the intrinsic order on $\spec A$ is induced as a subspace of the point-wise order on $\spec\I[X] = \I^X$.
\end{lemma}
\begin{proof}
  By the previous discussion, for any $x,y : \spec A$ we have
  \[ x \preceq y \eq \fa p{\I[X]/R} xp \le yp. \]
  Now if we view $x,y$ as objects in $\spec \I[X] = \I^X$, then for any $p : \I[X]/R$, $xp$ is simply the evaluation of $p$ on $x$. Of course then $xp \le yp$ for all $p$ iff $x \le y$ point-wise, since $p$ can be chosen as the generating variables.
\end{proof}

\begin{corollary}[SQCF]\label{cor:syntheticorderinterval}
  The intrinsic order on cubes $\I^n$ and on the simplices $\Delta^n$ are exactly the one we expect. In particular, the maps between them are all monotone.
\end{corollary}

Given the intrinsic order on $X$, one natural question is to ask when it is classified by the interval object $\I$ itself. However, to even formulate this property, we should at least require that $\I$ has an additional boundary point $0$, which is the minimal element.

Besides this, motivated by applications in domain theory, the existence of a minimal element in the dominance $\I$ is also accounted for divergent computation. For this reason, we will now make an additional assumption on our theory $\T$:

\begin{definition}
  We say a propositional theory $\T$ is admits \emph{falsum}, if it also has a constant $0$ denoting the minimal element in the order induced by the meet-semi-lattice structure.
\end{definition}

\begin{remark}
  A minimal example of such $\T$ is the theory of meet-semi-lattices equipped with an additional constant $0$. Of course, the theory of distributive lattices, Heyting algebras and Boolean algebras are all of such type, and so is any theory of a $\T$-algebra for a propositional theory $\T$ with falsum.
\end{remark}

From now on, we work with a propositional theory $\T$ that admits falsum. To make the dominace $\I$ closed under falsum, we may further assume the following non-triviality axiom:

\begin{axiom}[NT]\label{ax:nt}
  For $\I$, $0 \neq 1$.
\end{axiom}

\begin{remark}
  The above axiom semantically corresponds to working with certain subtopoi of the classifying topos $\Set[\T]$. For instance, the minimal topology on the underlying site $\mmod\T\fp$ we might choose for (NT) to hold is to assert that the trivial $\T$-model is covered by the empty sieve. Since the trivial $\T$-model is a strict terminal object, this topology will be subcanonical, thus (SQCF) still holds.
\end{remark}

Before we return to our previous question, let us first observe some elementary consequence of (NT). For instance, the lifting monad $(-)\prt$ defined in Section~\ref{sec:lifting} is now also pointed, where for any $X$ we can define
\[ \bot := (0,?) : X\prt, \]
where $? : \ms t0 = \emp \to X$ is the unique map from $\emp$.

As another consequence, now $\emp = \ms t0$ is affine, in fact an open proposition. For any $\I$-algebra $A$, we say $A$ is \emph{trivial} if it is equivalent to the trivial $\I$-algebra, which we denote as $0$. Equivalently, this is saying that $0 = 1$ in $A$. In the case of (NT), we have the following weak form of nullstellensatz:

\begin{lemma}[NT]\label{lem:nulls}
  For any affine $X = \spec A$, $X = \emp$ iff $A$ is trivial.
\end{lemma}
\begin{proof}
  By assumption, $A = \I^{\spec A} = \I^\emp = 0$. 
\end{proof}

Let us look at a simple example of the application of the above weak nullstellensatz. Notice that the constant $0$ also induces a predicate on $\I$,
\[ \ms f : \I \to \pp, \]
which takes any $i : \I$ to $i = 0$. 

\begin{lemma}[NT, SQCI]\label{lem:prefield}
  For any $i : \I$, $\ms ti \eq \neg \ms fi$.
\end{lemma}
\begin{proof}
  If $\ms fi$ then $\neg\ms ti$ by (NT). On the other hand, by the conservativity result in Lemma~\ref{lem:intconserve} and (NT), if $i \neq 1$ then $i = 0$ since $0 \neq 1$.
\end{proof}

\begin{remark}\label{rem:prefield}
  At this point, we do not have the converse $\ms fi \eq \neg \ms ti$, since as shown in the above proof it requires $\ms f$ also being an embedding. But the proof of Lemma~\ref{lem:intconserve} heavily relies on the fact that $\T$ is propositional. For the dual case to hold for $\ms f$, it will require $\T$ to also have join $\vee$, and furthermore the quotient $A/a = 0$ is computed by the coslice $a/A$ for any $\T$-model $A$. This will be true for distributive lattices and Boolean algebras.

  For the theory of Heyting algebras this is \emph{not} true. For instance, under (SQCI), the affine space $\I/i=0$ is equivalent to $\I/\neg i$. It then follows by Lemma~\ref{lem:intconserve} that $\ms fi = \ms fj$ iff $\neg i = \neg j$, which not necessarily implies $i = j$.
\end{remark}

Now let us get back to the discussion of the intrinsic order. Under the current set up, the interval $\I$ has a minimal element $0$. Hence, it now makes sense to ask when the intrinsic order to be classified by maps from $\I$, which is usually called the \emph{Phoa's principle} in the literature:

\begin{definition}
  We say a type $X$ satisfies the \emph{Phoa's principle}, or simply is \emph{Phoa}, if its intrinsic order is classified by $\pair{\ev_0,\ev_1} : X^\I \to X \times X$.
\end{definition}

The Phoa's principle has many consequences. For instance, if $X$ is Phoa, since the intrinsic order on a type is in particular a relation, the evaluation pair $X^\I \to X \times X$ will be an \emph{embedding}. Intuitively, it means that any map from the interval $f : \I \to X$ is completely determined by its boundary. Furthermore, since $\preceq$ is also \emph{transitive}, there will then be a well-defined \emph{composition} on functions from $\I$ to $X$. These are indeed certain internal \emph{orthogonality} classes that can be formulated in our type system. We will consider them in more detail in Section~\ref{sec:synposet}.

Another consequence of the Phoa's principle is that, it also implies an extensionality principle on intrinsic orders on function spaces:

\begin{lemma}[SQCF]\label{lem:phoaexponential}
  For any $X,Y$, if $X$ is Phoa, then the intrinsic order on $X^Y$ is induced point-wise by the intrinsic order on $X$, and is also Phoa.
\end{lemma}
\begin{proof}
  Since $\I \to X^Y$ is equivalent to $Y \to X^\I$, it suffices to show that the intrinsic order on $X^Y$ is point-wise. For any $f,g : X^Y$, suppose $f \preceq g$. For any $y : Y$, since the evaluation map $\ev_y : X^Y \to X$ is monotone by Lemma~\ref{lem:anymapmonotoneintriscorder}, we have $fy \preceq gy$. On the other hand, suppose for any $y : Y$ we have $fy \preceq gy$. This way, by the Phoa's principle on $X$, we get a map $Y \to X^\I$, whose transpose 
  \[ [f,g] : \I \to X^Y, \]
  satisfies $[f,g](0) = f$ and $[f,g](1) = g$. Then consider any $U : X^Y \to \I$. Notice that by Corollary~\ref{cor:syntheticorderinterval} the composite $U[f,g]$ is monotone, thus
  \[ Uf = U[f,g](0) \le U[f,g](1) = Ug. \]
  Hence, indeed we have $f \preceq g$.
\end{proof}

As a more or less trivial example, let us first consider discrete sets:

\begin{lemma}[NT, SQCF]\label{lem:discretephoa}
  If $M$ is a set with decidable equality, then the intrinsic order on $M$ is discrete, in the sense that for $n,m : M$,
  \[ n \preceq m \to n = m. \]
  In particular, $M$ is Phoa, in the sense that $M \to M^\I$ is an equivalence.
\end{lemma}
\begin{proof}
  Suppose $n \preceq m$. If $n \neq m$, we can construct a map $f : M \to \I$ with $f(n) = 1$ and $f(m) = 0$, contradictory to $n \preceq m$. Hence, $n = m$. 
  
  Now suppose $f : \I \to M$. For any $i:\I$, by monotonicity in Lemma~\ref{lem:anymapmonotoneintriscorder}, $f(i) \preceq f(1)$, which implies $f(i) = f(1)$. Hence, $f$ is the constant function on $1$, which implies $M \to M^\I$ is an equivalence.
\end{proof}

\begin{remark}
  The above result implies that $2^\I = 2$ and $\N^\I = \N$, which in some sense says that $\I$ is \emph{connected} and \emph{compact}. In fact, the above proof applies to any type $X$ whose intrinsic order has a maximal or minimal element. Thus by Corollary~\ref{cor:syntheticorderinterval}, the same result applies to the affine spaces $\I^n$ and $\Delta^n$ under (NT) and (SQCF).
\end{remark}

For types that are not necessarily discrete, one might expect that quasi-coherence will imply all affine spaces would be Phoa. As a minimal requirement, one crucial axiom of synthetic domain theory is that the interval $\I$ itself should be Phoa; cf.~\cite{hyland2006first}. However, under the current assumption, this is not necessarily the case. Given an affine space $X = \spec A$, it is easy to compute $X^\I$ by duality,
\[ X^\I = \I\alg(A,\I[x]). \]
By the characterisation previously, this coincides with the intrinsic order on $X$ iff the following pair of evaluation on polynomials
\[ \pair{\ev_0,\ev_1} : \I[x] \to \I \times \I \]
classifies the order on $\I$. What's interesting is that this is a purely \emph{algebraic} statement. Hence, we should look for a Horn theory where this property holds.

A canonical family of examples consists of the theory of \emph{distributive lattices}, or more generally the theory of $D$-algebras for any distributive lattice $D$. The observation that this special property of distributive lattices can be used to show Phoa's principle is already made in~\cite{gratzer2024directed}. Hence, from now on, we will specifically assume we work in a theory of $D$-algebras for some distributive lattice $D$.

\section{Distributive lattice and Phoa's principle}\label{sec:distphoa}

Working with distributive lattices not only gives us the Phoa's principle. As already mentioned in Remark~\ref{rem:prefield}, the theory of distributive lattices also satisfies the dual axiom of being propositional. Hence, previous developments would also hold when we exchange $1$ for $0$ and $\wedge$ for $\vee$.

For instance, the same proof as in Lemma~\ref{lem:intconserve} also implies the following:

\begin{corollary}[SQCI]
  $\ms f : \I \to \pp$ is an embedding.
\end{corollary}

We will now call propositions in the image of $\ms f$ as \emph{closed propositions}, and a subtype classified by $\ms f$ a \emph{closed subtype}. Completely similar to Lemma~\ref{lem:openofaffinegivesalgebra}, closed subtypes of affine spaces are again of the following form:

\begin{corollary}[SQCI]\label{cor:closedaffinealgebra}
  If $X = \spec A$ is affine, then closed subtypes of $X$ are all of the form
  \[ C_a := \scomp{x : X}{\ms f(xa)} = \spec a/A, \]
  for some $a:A$. If $X$ is stably affine, then so is $C_a$.
\end{corollary}

\begin{corollary}[SQCI]
  $\ms f$ forms a dominance.
\end{corollary}

There is an accompanying \emph{colifting} monad $(-)\cprt$, defined by
\[ X\cprt := \sum_{i:\I} X^{\ms fi}. \]
And similarly, under (SQCF), we can explicitly compute the colifting of the simplices $\Delta^n$,
\[ \Delta^n\cprt = \Delta^{n+1}, \]
where now the unit $\eta : \Delta^n \inj \Delta^n\cprt = \Delta^{n+1}$ takes a sequence $i_1 \ge \cdots \ge i_n$ to $i_1 \ge \cdots \ge i_n \ge 0$ in $\Delta^{n+1}$. To distinguish the two inclusions $\Delta^n \inj \Delta^n\prt = \Delta^{n+1}$ and $\Delta^n \inj \Delta^{n}\cprt = \Delta^{n+1}$, we will denote the former as $\eta\prt$ and the latter as $\eta\cprt$.

Furthermore, the dual statement of Lemma~\ref{lem:prefield} now also holds:

\begin{corollary}[NT, SQCI]\label{cor:field}
  $\I$ is a field in the sense that
  \[ \fa i{\I} \neg \ms ti \eq \ms fi, \quad \fa i\I \neg\ms fi \eq \ms ti. \]
  In particular, the embedding $2 \inj \I$ induced by $0,1$ is $\neg\neg$-dense,
  \[ \fa i\I \dneg(\ms ti \vee \ms fi). \]
\end{corollary}

This allows us to observe that the open and closed propositions are exactly complementary to each other:

\begin{corollary}[NT, SQCI]\label{cor:opendnegclose}
  For any proposition $p$, $p$ is open iff $\neg p$ is closed and vice versa. Furthermore, open and closed propositions are $\dneg$-stable.
\end{corollary}

\begin{remark}
  Notice that the above result only uses the fact that $\T$ also satisfies the dual axiom of propositional, thus in particular applies to the case of Boolean algebras as well. However, the content will in some sense be trivialised in that case. By the presence of a classical negation, a proposition is open iff it is closed.
\end{remark}

More importantly, as indicated at the end of the previous section, working with distributive lattices allows us to show that all affine spaces will satisfy the Phoa's principle:

\begin{theorem}\label{thm:phoaaffine}
  Any affine space $X$ is Phoa.
\end{theorem}
\begin{proof}
  As mentioned, for any affine space $X$ we have 
  \[ X^\I = \I\alg(A,\I[i]). \]
  Hence it suffices to show that the pair of evaluation maps
  \[ \pair{\ev_0,\ev_1} : \I[i] \to \I \times \I \]
  classifies the order on $\I$. This follows from a normal form result of polynomials for distributive lattices, i.e. any polynomial $p$ is of the form 
  \[ p = p(0) \vee x \wedge p(1). \]
  See e.g.~\cite[Thm. 10.11]{lausch2000algebra}. This way, a map $\I \to X$ is exactly described as two points $x,y : X$ such that $\fa aA xa \le ya$, which coincides with $x \preceq y$.
\end{proof}

\begin{remark}
  Crucially, we emphasis that the above proof does \emph{not} in fact rely on the affineness of the interval $\I$ itself. Rather, it is purely the consequence of the fact that for any distributive lattice $A$, the free algebra $A[i]$ is equivalent to the order on $A$. This is a perfect example of how an algebraic property of a theory has a non-trivial effect on the internal logic of its classifying topos.
\end{remark}

As a consequence, we can also show that the algebraic objects also satisfy the Phoa's principle:

\begin{theorem}[SQCF]\label{them:phoaalgebra}
  For any quasi-coherent $\I$-algebra $A$, the intrinsic order on $A$ coincides with its usual order, and it is Phoa.
\end{theorem}
\begin{proof}
  Since $A$ is quasi-coherent, we have $A = \I^{\spec A}$. By Theorem~\ref{thm:phoaaffine} the affine space $\I$ is Phoa, hence Lemma~\ref{lem:phoaexponential} implies that the intrinsic order on $A$ coincides with the point-wise order on $\I^{\spec A}$, which is indeed the order on $A$. Lemma~\ref{lem:phoaexponential} then also implies $A$ satisfies the Phoa's principle.
\end{proof}



At the end of this section, we describe another interesting perspective arising from the proof of Theorem~\ref{thm:phoaaffine}. We have seen that the dualising object $\I$ has a double role: It is both an algebra and a spectrum. The proof of Phoa's principle gives us many more such examples. For instance, $\I[i]$ classifies the order on $\I$, which by definition is also a spectrum. Now since for us the simplice $\Delta^2$ is the dual order, we will then identify it with $\I[i]$ along the following map,
\[ \pair{\ev_1,\ev_0} : \I[i] \to \Delta^2. \]
In fact, \emph{all} the simplices in this case have an algebraic description:

\begin{proposition}\label{prop:simplicesasalgebra}
  For any $n : \N$ and $\I$-algebra $A$, for any $a_0 \ge \cdots \ge a_n$ in $A$, define a polynomial in $A[i_1,\cdots,i_n]$ as follows,
  \[ a_0 \wedge i_1 \vee a_1 \wedge \cdots \wedge i_n \vee a_n. \]
  This is well-defined in the quotient $A[n]/i_1 \le \cdots \le i_n$, in the sense that no matter how you arrange the parenthesis there is a unique value in the quotient $A[n]/i_1 \le \cdots \le i_n$. In fact this gives us an equivalence
  \[ \Delta[A]^{n+1} = A[n]/i_1 \le \cdots \le i_n, \]
  where we write $\Delta[A]^{n+1}$ as $\scomp{a_0,\cdots,a_n : A}{a_0 \ge \cdots \ge a_n}$.
\end{proposition}
\begin{proof}
  This again follows from a normal form result on algebras of the form $A[n]/i_1 \le \cdots \le i_n$, which is a consequence of the general normal form for multivariable polynomials on distributive lattices; see~\cite[Thm. 10.21]{lausch2000algebra}.
\end{proof}

Now we can identify the simplices $\Delta^n$ as a finitely presented $\I$-algebra as shown above. Under such equivalences, the inclusion $\eta\prt$ simply corresponds to the canonical inclusion as shown below,
\[
  \begin{tikzcd}
    \Delta^{n+1} \ar[r, tail, "\eta\prt"] \ar[d, "\simeq"'] & \Delta^{n+2}  \ar[d, "\simeq"] \\ 
    \I[n]/i_1 \le \cdots \le i_n \ar[r, tail] & \I[n\!+\!1]/i_0 \le \cdots \le i_{n} 
  \end{tikzcd}
\]

\section{Infinitay domain theory}

Untill this point, we have seen that all the finitary axioms for synthetic domain theory is a consequence of (SQCF) for a theory of distributive algebra, with potentially the assumption of (NT).

There is one final crucial infinitary axiom of synthetic domain theory. From the observation in~\cite{hyland2006first,JIBLADZE1997185}, we can define internally in type theory the initial algebra and final coalgebra for the lifting functor $(-)\prt$, which we denote as $\omega$ and $\ov\omega$, respectively. There is a canonical inclusion $\omega \inj \ov\omega$, and the final axiom for synthetic domain theory states that the interval $\I$ is \emph{complete}, in the sense that the induced map $\I^{\ov\omega} \to \I^\omega$ is an \emph{equivalence}. The goal of this section is to explain that, this completeness axiom can also be realised as a consequence of quasi-coherence.

Since $(-)\prt$ by construction preserves all connected limits, the final coalgebra $\ov\omega$ can be easily characterised as a limit. As shown in~\cite{hyland2006first}, it can be identified as the object of infinite descending sequences in $\I$,
\[ \ov\omega := \scomp{i : \N \to \I}{\fa n\N i_n \ge i_{n+1}}, \]
which can be viewed as the following sequential limit,
\[
\begin{tikzcd}
  \cdots \ar[r] & \Delta^2 \ar[r, "!\prt"] & \I \ar[r, "!"] & 1
\end{tikzcd}
\]
where the transition map $\Delta^{n+1} \to \Delta^n$ takes the sequence $i_0 \ge \cdots \ge i_n$ to the final segment $i_1 \ge \cdots \ge i_n$. 

According to the above description, $\ov\omega$ is in fact a \emph{spectrum},
\[ \ov\omega = \spec(\I[\N]/\forall n.\, i_n \ge i_{n+1}). \]
However, the corresponding algebra is no longer finitely presented, but countably presented. By a countably presented, or c.p. in short, $\I$-algebra, we mean an $\I$-algebra of the form $\I[I]/s=t$ for some \emph{decidable} subsets $I,J$ of $\N$ with $s,t : J \to \I[I]$. In particular, all finitely presented $\I$-algebra will also be countably presented. Motivated by the above characterisation of $\ov\omega$, we naturally consider the following stronger quasi-coherence principle:

\begin{axiom}[SQCC]
  All c.p. $\I$-algebras are (stably) quasi-coherent.
\end{axiom}

\begin{remark}
  The quasi-coherence principle for countably presented algebras is investigated in~\cite{cherubini2024foundation}. Just like the finitary version (SQCF), which is true in the classifying topos $\Set[\T] = [\mmod\T\fp,\Set]$, the countable version (SQCC) will be valid in a larger presheaf topos $[\mmod\T\cp,\Set]$, or any subtopos induced by a subcanonical topology. For instance, the semantics of~\cite{cherubini2024foundation} is based on the topos of light condensed sets introduced by Clausen and Scholze, which is a subtopos of $[\mmod{\mbb B}\cp,\Set]$.
\end{remark}

As a first example of the usefulness of (SQCC), let us observe that it implies the following form of Markov principle:

\begin{lemma}[NT, SQCC]\label{lem:markov}
  For any $i : \ov\omega$, we have
  \[ \neg\fa{n}{\N}\ms ti_n \to \ex n\N\ms fi_n. \]
\end{lemma}
\begin{proof}
  Let $i : \ov\omega$. Notice that similar to Lemma~\ref{lem:openpropaffine}, the proposition $\fa n\N \ms ti_n$ by construction is the following affine space, 
  \[ \spec\I/i = \I\alg(\I/i,\I) = \fa n\N \ms ti_n, \]
  where we have abbreviated the c.p. $\I$-algebra $\I/\forall k.\, i_k = 1$ as $\I/i$. Now if we have $\neg\fa n\N \ms ti_n$, then $\spec\I/i = \emp$ which implies $\I/i$ is trivial by Lemma~\ref{lem:nulls}. But this algebra is trivial iff there merely exists $n : \N$ that $\ms fi_n$ holds, thus $\ex n\N \ms fi_n$.
\end{proof}

Now let us consider the initial algebra $\omega$ for the lifting functor. It is shown in~\cite{JIBLADZE1997185} that we can identify $\omega \inj \ov\omega$ as the following subset,
\[ \omega := \scomp{i : \ov\omega}{\fa\phi{\pp} (\fa n{\N} (\ms ti_n \to \phi) \to \phi) \to \phi}. \]
For another proof, see e.g.~\cite{VANOOSTEN2000233}. In the presence of the Markov principle above, this description can be drastically simplified:

\begin{lemma}[NT, SQCC]\label{lem:omegacolimit}
  $\omega$ is equivalent to the following subset of $\ov\omega$,
  \[ \omega = \scomp{i : \ov\omega}{\ex n{\N} \ms fi_n}. \]
\end{lemma}
\begin{proof}
  Let $i : \ov\omega$. It suffices to show that
  \[ \prth{\fa\phi{\pp} (\fa n{\N} (\ms ti_n \to \phi) \to \phi) \to \phi} \to \ex{n}\N \ms fi_n, \]
  We can take $\phi$ to be $\emp$. By assumption we have $\neg\fa n\N \neg\neg\ms ti_n$, which by Corollary~\ref{cor:opendnegclose} is equivalent to $\neg\fa n\N \ms ti_n$. Then Lemma~\ref{lem:markov} implies this is $\ex n\N \ms fi_n$.
\end{proof}

By~\cite[Cor. 1.10]{VANOOSTEN2000233}, the above result exactly means that $\omega$ is indeed the internal colimit of the following sequence,
\[ 
\begin{tikzcd}
  \emp \ar[r, "?"] & 1 \ar[r, "?\prt = \eta\cprt"] & \I \ar[r, "\eta\cprt"] & \Delta^2 \ar[r] & \cdots
\end{tikzcd}
\]
Using this, we can now show the most important infinitary axiom for synthetic domain theory:

\begin{theorem}[NT, SQCC]
  The canonical map $\I^{\ov\omega} \to \I^\omega$ is an equivalence.
\end{theorem}
\begin{proof}
  Since $\ov\omega$ is now affine, we have
  \[ \I^{\ov\omega} = \I[\N]/\forall n.\,i_n\ge i_{n+1}. \]
  On the other hand, since $\omega$ is internally the colimit of $\Delta^n$, we have
  \[ \I^\omega = \lt_{n:\N}\I^{\Delta^n} = \lt_{n:\N} \I[n]/i_1 \ge \cdots \ge i_n. \]
  Note that the transition maps induced by $\eta\cprt : \Delta^n \to \Delta^{n+1}$ under quasi-coherence gives us the following maps on algebras:
  \[
  \begin{tikzcd}
    \I^{\Delta^{n+1}} \ar[r, "\I^{\eta\cprt}"] & \I^{\Delta^n} \\ 
    \I[n\!+\!1]/i_1 \ge \cdots \ge i_{n+1} \ar[u, "\simeq"] \ar[r, "i_{n+1} \mapsto 0"'] & \I[n]/i_1 \ge \cdots \ge i_n \ar[u, "\simeq"']
  \end{tikzcd}
  \]
  Now taking the limit of the above sequence of algebras gives us
  \[ \lt_{n:\N}\I[n]/i_1 \ge \cdots \ge i_n = \I[\N]/\forall n.\,i_n \ge i_{n+1} = \I^{\ov\omega}. \]
  This shows that $\I^{\omega} = \I^{\ov\omega}$.
\end{proof}

Finally, at the end of this section, we observe that the algebraic description of the simplices given at the end of Section~\ref{sec:distphoa} can be extended to the above infinitary case:

\begin{lemma}
  $\I^{\ov\omega} = \I[\N]/\forall n.\, i_n \ge i_{n+1}$ is equivalent to the type of increasing sequences in $\I$,
  \[ \Delta_\infty := \scomp{i : \N \to \I}{\fa n\N i_n \le i_{n+1}}. \]
\end{lemma}
\begin{proof}
  Following the discussion at the end of Section~\ref{sec:distphoa}, we have the following equivalences,
  \[
    \begin{tikzcd}
      \Delta_{n+1} \ar[r, tail, "\eta\cprt"] \ar[d, "\simeq"'] & \Delta_{n+2}  \ar[d, "\simeq"] \\ 
      \I[n]/i_1 \ge \cdots \ge i_n \ar[r, tail] & \I[n\!+\!1]/i_1 \ge \cdots \ge i_{n+1} 
    \end{tikzcd}
  \]
  Then taking the sequential colimit on both sides, we get an equivalence
  \[ \Delta_{\infty} = \I[\N]/\forall n.\, i_n \ge i_{n+1}. \qedhere \]
\end{proof}

% \section{Schemes}\label{sec:scheme}

% To develop the theory of schemes, we first introduce the following notion of size:

% \begin{definition}[Size]
%   A \emph{notion of size} is a type $\ms U$ of sets, such that
%   \begin{itemize}
%     \item $\ms U$ is closed under \emph{finite types} $n$ for $n:\N$;
%     \item $\ms U$ is closed under \emph{identity types}, i.e. for any $X : \ms U$,
%     \[ \fa{x,y}X x = y : \ms U. \]
%     \item $\ms U$ is closed under \emph{dependent sums},
%     \[ A : \ms U \conjt B : A \to \ms U \nt \sum_{a:A}B(a) : \ms U. \]
%     \item $\ms U$ is \emph{projective}: Any $\ms U$-small type $X$ is projective.
%     \item It has a small classifier: There is a set $U$ with $\El : U \to \ms U$, that for $A : \ms U$ and $B : A \to \ms U$, there merely exists $\gn B : A \to U$ with
%     \[ \fa a A \pss{B(a) = \El(\gn B(a))}. \]
%   \end{itemize}
%   We say a set is \emph{$\ms U$-small} if it belongs to $\ms U$. 
% \end{definition}

% For our first example, let us consider the notion of finite:

% \begin{example}
%   We say a type $X$ is \emph{finite} if it is merely equivalent to a finite cardinal,
%   \[ \ms{isFin}(X) := \sum_{n:\N} \pss{X = n}. \]
%   The type of finite sets will be denoted as $\fst$. Since the $n$ that $X = n$ is unique, being finite is a proposition. Any finite type will be a set with decidable equality. $\fst$ being closed under dependent sums can be shown by induction. Also, finite sets are projective. The classifier of $\fst$ can be chosen as $\N$.
% \end{example}

% For the next example, let us consider \emph{countability}. In this case, to show it forms a notion of size, we need to assume \emph{countable choice (CC)}, i.e. $\N$ is projective:

% \begin{example}[CC]\label{exm:countablesize}
%   A type $X$ is \emph{countable} if it is merely a decidable subset of the natural numbers,
%   \[ \ms{isCnt}(X) := \ex{I}{\N\to\pp} \ms{isDec}(I) \wedge \pss{X = I}. \]
%   Here $I$ being decidable means that
%   \[ \ms{isDec}(I) := \fa{n}{\N}I(n) \vee \neg I(n). \]
%   By construction, this means we are chosing the classifier to be,
%   \[ \sub_{\ms{Dec}}(\N) := \scomp{I : \N \to \pp}{\ms{isDec}(I)}. \]
%   However, this is a classifier again relies on projectivity of $\N$.
%   It is not hard to see that if $\N$ is projective then all countable sets will also be projective, since they merely are retracts of $\N$. Suppose we have $X$ countable and $Y : X \to \cnt$ a countable family of countable sets. To show $\sum_{x:X}Y_x$ is countable, since this is a proposition, we may assume $X$ is $I$ for some decidable subset $I$. Then by assumption, 
%   \[ \prod_{i:I}\ex{J}{\N \to \pp} \ms{isDec}(J) \wedge \pss{Y_i = J}. \] 
%   By projectivity of $I$, we merely get a family $J_i$ of decidable subsets of $\N$ such that $\pss{J_i = Y_i}$. Thus, the dependent sum $\sum_{x:X}Y_x$ will be equivalent to $\sum_{i:I}J_i$, which is a decidable subset of $\N \times \N$, which makes it countable. 
% \end{example}

% Now let us fix a notion of size $\ms U$. We will now unify (SQCF) and (SQCC) as the following axiom:

% \begin{axiom}[SQC]
%   Any $\ms U$-presented $\I$-algebra $A$ is quasi-coherent.
% \end{axiom}

% From now on, under (SQC), we will simply identify an affine space $X$ as the spectrum $\spec A$ of a $\ms U$-presented $\I$-algebra $A$.

% \begin{lemma}[SQC]
%   Affine spaces are closed under 
% \end{lemma}

% To develop more glueing, let us first introduce the following stronger locality principle:

% \begin{axiom}[L]
%   $\I$ is local, in the sense that $\ms t$ preserves finite joins, i.e. $0 \neq 1$ and for any $i,j : \I$, $\ms t(i\vee j) \eq \ms ti \vee \ms tj$.
% \end{axiom}

% In other words, (L) is saying that open propositions are also closed under finite disjunctions. We first observe some elementary consequence of (L):

% \begin{lemma}[L, SQCF]
%   $\I$ is not Boolean.
% \end{lemma}
% \begin{proof}
%   If $\I$ is Boolean, then $\I = 2$ by Corollary~\ref{cor:field}, since in that case we have for any $i : \I$,
%   \[ \ms ti \vee \ms fi = \ms ti \vee \ms t\neg i = \ms t(i \vee \neg i) = 1. \]
%   However, this contradicts Lemma~\ref{lem:discretephoa}, because if $\I = 2$ then $2^\I = 2^2$, which cannot be equivalent to $2$.
% \end{proof}

% (L) also allows us to construct more affine spaces:

% \begin{example}[L]
%   Consider the right outer horn $\Lambda^2_2$ as a pushout,
%   \[
%     \begin{tikzcd}
%       1 \ar[d, "1"'] \ar[r, "1"] & \I \ar[d] \\
%       \I \ar[r] & \Lambda^2_2
%       \arrow["\lrcorner"{anchor=center, pos=0.125, rotate=180}, draw=none, from=2-2, to=1-1]
%     \end{tikzcd}
%   \]
%   If we view $\Lambda^2_2$ as a subspace of $\I^2$, we would have
%   \[ \Lambda^2_2 = \scomp{i,j : \I}{\ms ti \vee \ms tj = \ms t(i\vee j)}, \]
%   which now implies that
%   \[ \Lambda^2_2 = \spec \I[i,j]/i \vee j. \]
% \end{example}

% \begin{remark}
%   However, at this point the left outer horn $\Lambda^2_0$ is \emph{not} necessarily affine. This will be if we also assume the dual axiom of (L), which states that for any $i,j : \I$, $\ms f(i \wedge j) \eq \ms fi \vee \ms fj$.
% \end{remark}

% For any type $X$ and a family $U : K \to (X \to \pp)$ of subtypes of $X$, we say $U$ is a \emph{cover} of $X$ if
% \[ \fa xX \ex iI U_i(x). \]

% \begin{definition}
%   We say $X$ is a \emph{scheme}, if there merely is a finite family of open covers $U : n \to \I^X$, where each $U_r \subseteq X$ is affine for $r : n$.
% \end{definition}

% \begin{lemma}\label{lem:schemeglue}
%   Let $X,Y$ be schemes, and $Z \hook X$ and $Z \hook Y$ be open embeddings. Then the pushout $X \sqcup_Z Y$ is again a scheme.
% \end{lemma}
% \begin{proof}
%   The proof is essentially the same in~\cite[Prop. 5.3.1]{cherubini2024foundation}.
% \end{proof}



% \begin{lemma}
%   Any open subtype of a scheme is again a scheme.
% \end{lemma}
% \begin{proof}
%   This follows from Lemma~\ref{lem:openofaffinegivesalgebra}.
% \end{proof}

% \begin{lemma}
%   Any closed subtype of a scheme is again a scheme.
% \end{lemma}
% \begin{proof}
%   Let $X$ be a scheme and $C$ a closed subtype. If $U$ is an affine open of $X$, $U \cap C$ will be open in $C$. Hence, it suffices to show $U \cap C$ is affine. But since $U \cap C$ is closed in $U$, by Corollary~\ref{cor:closedaffinealgebra} this is indeed affine.
% \end{proof}


\section{Locality, generalised lifting, and intrinsic posets}\label{sec:synposet}

\begin{definition}
  We say $X$ is an \emph{intrinsic poset}, if the intrinsic order on $X$ is anti-symmetric, and it satisfies the Phoa's principle.
\end{definition}

\begin{lemma}
  Any intrinsic poset is a set.
\end{lemma}
\begin{proof}
  By definition, for any intrinsic poset $X$ and $x,y : X$, 
  \[ x = y \eq x \preceq y \wedge y \preceq x, \] 
  thus is a proposition.
\end{proof}

\begin{lemma}
  All affine spaces are intrinsic posets.
\end{lemma}
\begin{proof}
  As shown in Theorem~\ref{thm:phoaaffine}, any affine space $X = \spec A$ is Phoa, and the intrinsic order on $X$ is induced by the point-wise order on $\I\alg(A,\I)$, which is anti-symmetric.
\end{proof}

\begin{lemma}[SQCF]
  All quasi-coherent $\I$-algebras are intrinsic posets.
\end{lemma}
\begin{proof}
  Again, by Theorem~\ref{them:phoaalgebra}, a quasi-coherent algebra $A$ is Phoa, and the intrinsic order coincides with the order on $A$, hence is anti-symmetric.
\end{proof}

Intrinsic posets satisfy the closure properties of an orthogonality class:

\begin{lemma}[SQCF]
  The intrinsic order on limits of intrinsic posets is point-wise, and are again intrinsic posets. They in fact form an exponential ideal.
\end{lemma}
\begin{proof}
  Suppose we have a limit $\lt_{r:I}X_r$ of intrinsic posets. Given $(x_r)_{r:I}$ and $(y_r)_{r:I}$ in the limit, by monotonicity in Lemma~\ref{lem:anymapmonotoneintriscorder}, if $(x_r)_{r:I} \preceq (y_r)_{r:I}$, then the projection maps imply that $x_r \preceq y_r$ for all $r:I$. On the other hand, if $x_r \preceq y_r$ for all $r:I$, then since each $X_r$ is Phoa, we get a unique map $f_r : \I \to X_r$. We then need to show that this gives us $(f_r)_{r:I} : \I \to X_r$. This indeed follows from the Phoa's principle on each $X_r$. Hence, the order on the limit is again point-wise, and the limit is Phoa. Being an exponential ideal follows from~\ref{lem:phoaexponential}.
\end{proof}

We would like to furthermore show that intrinsic posets are closed under lifting. To this end, we introduce a stronger locality axiom:

\begin{axiom}[L]
  $\I$ is local, i.e. $0 \neq 1$ and $\ms t(i\vee j) \eq \ms ti \vee \ms tj$ for all $i,j : \I$.
\end{axiom}

We observe some elementary consequences of (L):

\begin{lemma}[L, SQCF]
  $\I$ is not Boolean.
\end{lemma}
\begin{proof}
  If $\I$ is Boolean, then $\I = 2$ by Corollary~\ref{cor:field}, since in that case we have for any $i : \I$,
  \[ \ms ti \vee \ms fi = \ms ti \vee \ms t\neg i = \ms t(i \vee \neg i) = 1. \]
  However, this contradicts Lemma~\ref{lem:discretephoa}, because if $\I = 2$ then $2^\I = 2^2$, which cannot be equivalent to $2$.
\end{proof}

Furthermore, (L) also allows us to define more affine spaces:

\begin{example}[L]
  Consider the right outer horn $\Lambda^2_2$ as a pushout,
  \[
    \begin{tikzcd}
      1 \ar[d, "1"'] \ar[r, "1"] & \I \ar[d] \\
      \I \ar[r] & \Lambda^2_2
      \arrow["\lrcorner"{anchor=center, pos=0.125, rotate=180}, draw=none, from=2-2, to=1-1]
    \end{tikzcd}
  \]
  By viewing $\Lambda^2_2$ as a subspace of $\I^2$, we may identify it as follows,
  \[ \Lambda^2_2 = \scomp{i,j : \I}{\ms ti \vee \ms tj}. \]
  Since $\ms ti \vee \ms tj = \ms t(i\vee j)$, it follows that 
  \[ \Lambda^2_2 = \spec \I[i,j]/i \vee j. \]
\end{example}

The most important consequence of (L) is that it allows us to show intrinsic posets are closed under dependent sums over certain affine spaces:

\begin{definition}[SQCF]
  We say an affine space $X$ is \emph{finitary}, if it is of the form $\spec\I[n]/s=t$ for some finitely presented $\I$-algebra. $X$ is thus a subspace of $\I^n$, and we say $X$ is \emph{topped}, if $1 : \I^n$ belongs to $X$.
\end{definition}

\begin{example}
  All the finitary examples of affine spaces considered so far, including the cubes $\I^n$, the simplices $\Delta^n$ and $\Delta_n$, and the outer horn $\Lambda^2_2$, are all topped.
\end{example}

One interesting fact is that, since $\I$ is a dominance, it will always be orthogonal to the following family of maps:

\begin{lemma}[SQCF]\label{lem:toppedfamilyortho}
  Let $X$ be a finitary topped affine space, $Y : X \to \tp$ a type family. Then $\I$ is weakly orthogonal to the embedding $Y_1 \hook \sum_{i : X}Y_i$,
  \[
  \begin{tikzcd}
    Y_1 \ar[d, hook] \ar[r, "f"] & \I \\ 
    \sum_{i : X} Y_i \ar[ur, dashed, "\qsi f"']
  \end{tikzcd}
  \]
\end{lemma}
\begin{proof}
  Using the dominance structure, we define $\qsi f$ as follows,
  \[ \qsi f(i,y) := \sum_{*:\ms ti} f(y). \]
  This is well-defined, and evidently extends $f$.
\end{proof}

For any finitary topped $X$, by viewing it  we may define the generalised lifting functor over $X$ as follows:
\[ L_XY := \sum_{i:X}Y^{\ms ti}. \]
Of course, $L_\I$ is the usual lifting functor $(-)\prt$. The most crucial observation of this section is that intrinsic posets are closed under generalised liftings:

\begin{theorem}[L, SQCF]
  Let $X$ be a finitary topped affine space and $Y$ an intrinsic poset. Then the intrinsic order on $L_XY$ is given by
  \[ (i,x) \preceq (j,y) \eq \ms ti \to (\ms tj \wedge x \preceq_{Y} y). \]
  In particular, $L_XY$ is again an intrinsic order.
\end{theorem}
\begin{proof}
  Recall from Lemma~\ref{lem:orderonaffine} the intrinsic order on $X$ is induced from the point-wise order on $\I^n$. For the left to right direction, if $(i,x) \preceq (j,y)$, then by monotonicity $i \le j$. If $\ms ti$ holds, then $\ms tj$ holds, and $x \preceq_Y y$ since by Lemma~\ref{lem:toppedfamilyortho} any map $Y \to \I$ extends to one on $L_XY$.

  On the other hand, suppose we have $(i,x) \preceq (j,y)$ in $L_XY$. We construct a map $f : \I \to L_XY$. On the base, it must take $k$ to $i \vee k \wedge j$. Hence, assuming $\ms t(i \vee k \wedge j)$, we need an element in $Y$. By (L), equivalently $\ms ti \vee \ms tk \wedge \ms tj$. Thus, we can define this by case distinction:
  \begin{itemize}
    \item If $\ms ti$ holds, then since $x \preceq_Y y$, by the fact that $Y$ is an intrinsic poset, we get a unique map $g : \I \to X$, and we define $f(k) := g(k)$.
    \item If $\ms tk \wedge \ms tj$ holds, we simply put $fk := y$.
  \end{itemize}
  This is evidently well-defined, thus we get a map $f : \I \to L_XY$.
\end{proof}



\bibliographystyle{apalike} 
\bibliography{mybib}


\end{document}


