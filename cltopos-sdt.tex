\documentclass[12pt]{amsart}

\usepackage{amsmath,amssymb,amsthm,amsfonts}
\usepackage[numbers]{natbib}
\usepackage[english]{babel}
\usepackage[X2,T1]{fontenc}
\usepackage{libertine}
\usepackage{libertinust1math}
\usepackage[utf8]{inputenc} %'utf8' instead of 'latin1'
\usepackage{enumitem} %modified eunumerate environment
\usepackage{tikz-cd}
\usepackage{hyperref}
\usepackage{stmaryrd}
\usepackage{quiver}
\usepackage{appendix}
\usepackage{mathtools}



\newtheorem{theorem}{Theorem}[section]
\newtheorem{fact}[theorem]{Fact}
\newtheorem{lemma}[theorem]{Lemma}
\newtheorem{conjecture}[theorem]{Conjecture}
\newtheorem{corollary}[theorem]{Corollary}
\newtheorem{claim}[theorem]{Claim}
\newtheorem{proposition}[theorem]{Proposition}
\newtheorem{observation}[theorem]{Observation}

\theoremstyle{definition}
\newtheorem{example}[theorem]{Example}
\newtheorem{definition}[theorem]{Definition}
\newtheorem{remark}[theorem]{Remark}
\newtheorem{question}[theorem]{Open Question}
\newtheorem{assumption}[theorem]{Assumption}
\newtheorem{qquestion}[theorem]{Question}
\newtheorem{axiom}{Axiom}

\newcommand{\mc}[1]{\mathcal{#1}}
\newcommand{\mb}[1]{\mathbf{#1}}
\newcommand{\mbb}[1]{\mathbb{#1}}
\newcommand{\T}{\mbb T}
\newcommand{\I}{\mbb I}
\newcommand{\gn}[1]{\ulcorner\! #1 \!\urcorner}
\newcommand{\mr}[1]{\mathrm{#1}}
\newcommand{\mf}[1]{\mathfrak{#1}}
\newcommand{\ms}[1]{\mathsf{#1}}
\newcommand{\Ind}{\mathbf{Ind}}
\newcommand{\Pro}{\mathbf{Pro}}
\newcommand{\Set}{\mb{Set}}
\newcommand{\Prop}{\mb{Prop}}
\newcommand{\sSet}{\mb{sSet}}
\newcommand{\sSeto}{\mb{sSet}_{\le 1}}
\newcommand{\End}{\operatorname{End}}
\newcommand{\Hom}{\operatorname{Hom}}
\newcommand{\Fin}{\mb{Fin}}
\newcommand{\Cnt}{\mb{Cnt}}
\newcommand{\Enm}{\mb{Enm}}
\newcommand{\Grp}{\mb{Grp}}
\newcommand{\sub}{\mr{Sub}}
\newcommand{\Lex}{\mb{Lex}}
\newcommand{\Pos}{\mb{Pos}}
\newcommand{\alg}{\text{-}\mb{Alg}}
\newcommand{\Var}{\mb{Var}}
\newcommand{\Aff}{\mb{Aff}}
\newcommand{\Vect}{\mb{Vect}}
\newcommand{\CRing}{\mb{CRing}}
\newcommand{\DL}{\mb{DL}}
\newcommand{\BA}{\mb{BA}}
\newcommand{\HA}{\mb{HA}}
\newcommand{\JL}{\mb{JL}}
\newcommand{\ML}{\mb{ML}}
\newcommand{\ProMon}{\mb{ProMod}}
\newcommand{\CTopoi}{\mb{CTopoi}}
\newcommand{\PTc}{\mb{PT}_c}
\newcommand{\Topoi}{\mb{Topoi}}
\newcommand{\sh}{\mb{Sh}}
\newcommand{\psh}{\mb{Psh}}
\newcommand{\Cont}{\mb{Cont}}
\newcommand{\Cart}[1]{#1^\to_{\text{cart}}}
\newcommand{\Str}{\mb{Str}}
\newcommand{\op}{^{\mathrm{op}}}
\newcommand{\inv}{^{\mathrm{-1}}}
\newcommand{\pf}[1]{\widehat{#1}}
\newcommand{\qsi}[1]{\tilde{#1}}
\newcommand{\cob}{\vartriangleleft}
\newcommand{\other}{\mathrm{otherwise}}
\newcommand{\geo}[1]{\left|#1\right|}
\newcommand{\ov}[1]{\overline{#1}}
\newcommand{\set}[1]{\{\,#1\,\}}
\newcommand{\eff}{\Leftrightarrow}
\newcommand{\conjt}{\;\&\;}
\newcommand{\pair}[1]{\left\langle#1\right\rangle}
\newcommand{\id}{\mathrm{id}}
\newcommand{\ev}{\mathrm{ev}}
\newcommand{\elem}{\int\!\!}
\newcommand{\nt}{\Rightarrow}
\newcommand{\scomp}[2]{\{\,#1\mid#2\,\}}
\newcommand{\yon}{\mathtt{y}}
\newcommand{\surj}{\twoheadrightarrow}
\newcommand{\inj}{\rightarrowtail}
\newcommand{\hook}{\hookrightarrow}
\newcommand{\cg}{\operatorname{\sim}}
\newcommand{\im}{\operatorname{img}}
\newcommand{\cgf}[2]{\leftindex_{#1}{\cg}_{#2}}
\newcommand{\coten}{\pitchfork}
\newcommand{\ppr}{\operatorname{\hat\times}}
\newcommand{\rl}{^{\perp}}
\newcommand{\llo}[1]{\leftindex_{}^{\perp} {#1}}
\newcommand{\dl}{^{\circ}}
\newcommand{\prth}[1]{\left(#1\right)}
\newcommand{\fn}{_{\mr{f.}}}
\newcommand{\fp}{_{\mr{f.p.}}}
\newcommand{\fpp}{_{\mr{f.p.,+}}}
\newcommand{\cp}{_{\mr{c.p.}}}
\newcommand{\cpp}{_{\mr{c.p.,+}}}
\newcommand{\pls}{^+}
\newcommand{\mns}{^-}
\newcommand{\dv}{\operatorname{\uparrow}}
\newcommand{\cv}{\operatorname{\downarrow}}
\newcommand{\et}{_{\text{\'et}}}
\newcommand{\N}{\mb N}
\newcommand{\Q}{\mbb Q}
\newcommand{\Z}{\mbb Z}
\newcommand{\Deltao}{\Delta_{\le 1}}
\newcommand{\Deltaw}{\Delta_{\omega}}
\newcommand{\sk}{\ms{sk}}
\newcommand{\csk}{\ms{csk}}
\newcommand{\sInt}{\mb{sInt}}
\newcommand{\jcan}{J_{\mr{can}}}
\newcommand{\wCPO}{\omega\mb{CPO}}
\newcommand{\shape}{\operatorname{\smallint}}
\newcommand{\dneg}{\neg\neg}
\newcommand{\prt}{_{\bot}}
\newcommand{\cprt}{_{\top}}
\newcommand{\fa}[2]{\forall #1\!\colon\!\!#2.\ }
\newcommand{\ex}[2]{\exists #1\!\colon\!\!#2.\ }
\newcommand{\exu}[2]{\exists_! #1\!\colon\!\!#2.\ }
\newcommand{\ld}[2]{\lambda #1\!\colon\!\!#2.\ }
\newcommand{\subopen}{\subseteq_{\mbb \I}}
\newcommand{\emp}{\emptyset}
\newcommand{\eq}{\leftrightarrow}
\newcommand{\ass}[1]{\llbracket#1\rrbracket} %\usepackage{stmaryrd}
\newcommand{\pss}[1]{||#1||} %\usepackage{stmaryrd}
\newcommand{\tp}{\ms{Type}}
\newcommand{\pp}{\ms{Prop}}
\newcommand{\st}{\ms{Set}}
\newcommand{\cnt}{\ms{Cnt}}
\newcommand{\gp}{\ms{Grpd}}
\newcommand{\pcat}{\ms{PCat}}
\newcommand{\cat}{\ms{Cat}}
\newcommand{\pcatt}{\ms{PCAT}}
\newcommand{\catt}{\ms{CAT}}
\newcommand{\PCat}{\mb{PCat}}
\newcommand{\Cat}{\mb{Cat}}
\newcommand{\PCAT}{\mb{PCAT}}
\newcommand{\CAT}{\mb{CAT}}
\newcommand{\Catt}{\mf{Cat}}
\newcommand{\CATT}{\mf{CAT}}
\newcommand{\Top}{\mf{Top}}
\newcommand{\Tp}{\ms{TYPE}}
\newcommand{\Pp}{\ms{PROP}}
\newcommand{\St}{\ms{SET}}
\newcommand{\Gp}{\ms{GRPD}}
\newcommand{\fst}{\ms{Fin}}
\newcommand{\quot}[1]{/_{\pair{#1}}}
\newcommand{\List}{\ms{List}}
\newcommand{\hp}{\text{-}}
\newcommand{\PG}{\ms{PG}}
\newcommand{\uv}[1]{\underline{#1}}
\newcommand{\mmod}[1]{#1\text{-}\mathbf{Mod}}
\newcommand{\func}{\mb{Func}}
\newcommand{\tm}[1]{#1\text{-}\mathrm{Term}}
\newcommand{\eqn}[1]{#1\text{-}\mathrm{Eqn}}
\newcommand{\horn}[1]{#1\text{-}\mathrm{Horn}}
\newcommand{\gr}[2]{[#1|#2]}
\newcommand{\VT}{\mbb V_\T}
\newcommand{\spec}{\operatorname{Spec}}
\newcommand{\El}{\mr{El}}
\newcommand{\lan}{\ms{lan}}
\newcommand{\ran}{\ms{ran}}
\newcommand{\upp}{_{\ms U}}
\newcommand{\dsg}[1]{\!\pair{#1}}
\DeclareFontFamily{U}{dmjhira}{}
\DeclareFontShape{U}{dmjhira}{m}{n}{ <-> dmjhira }{}
\DeclareRobustCommand{\yon}{\text{\usefont{U}{dmjhira}{m}{n}\symbol{"48}}}
\DeclareRobustCommand{\noy}{\text{\reflectbox{\yon}}\!}

\makeatletter
\newcommand{\ct@}[2]{%
  \vtop{\m@th\ialign{##\cr
    \hfil$#1\operator@font lim$\hfil\cr
    \noalign{\nointerlineskip\kern1.5\ex@}#2\cr
    \noalign{\nointerlineskip\kern-\ex@}\cr}}%
}
\newcommand{\ct}{%
  \mathop{\mathpalette\ct@{\rightarrowfill@\textstyle}}\nmlimits@
}
\makeatother
\makeatletter
\newcommand{\lt@}[2]{%
  \vtop{\m@th\ialign{##\cr
    \hfil$#1\operator@font lim$\hfil\cr
    \noalign{\nointerlineskip\kern1.5\ex@}#2\cr
    \noalign{\nointerlineskip\kern-\ex@}\cr}}%
}
\newcommand{\lt}{%
  \mathop{\mathpalette\lt@{\leftarrowfill@\textstyle}}\nmlimits@
}
\makeatother


\title{Quasi-coherence is (almost) all you need for synthetic domain theory}
\author{Lingyuan Ye}
\date{\today}



\begin{document}
%

%
%\titlerunning{Abbreviated paper title}
% If the paper title is too long for the running head, you can set
% an abbreviated paper title here
%

%
% \Endhorrunning{L. Ye}
% First names are abbreviated in the running head.
% If there are more than two authors, 'et al.' is used.
%
% \institute{New College\\

%
\maketitle              % typeset the header of the contribution
%

\section{Introduction}

Let $\T$ be a Horn theory. The type system we work with is based on the quasi-coherence principle available in the classifying topoi of subcanonical quotients of $\T$~\cite{blechschmidt2021using,blechschmidt2020general}. (The envelopping $\infty$-topoi of) such topoi provide models for the type theory we work with in this paper.

In this paper we will be working in a dependent type theory with a univalent universe $\tp$. Recall the notion of $h$-levels~\cite{hottbook}. For us the most important h-levels are -1 and 0, which are \emph{propositions} and \emph{sets}. In particular, we can define subuniverses $\pp$, $\st$.

We will also assume the existence of propositional truncation. This allows us to define existential quantifier of a family $P : X \to \pp$, 
\[ \ex x{X} P(x) := \pss{\sum_{x:X}P(x)} \] 
and disjunction of propositions,
\[ P \vee Q := \pss{P + Q}. \]
Note that by univalence, $\pp$ is already closed under dependent product. But to emphasis that the result is again a proposition, we may also write dependent products of propositions as $\fa xX P(x)$. 

The universe $\pp$ is used to define subtypes, i.e. we identify a subtype of $X$ as a family of propositions $A : X \to \pp$ indexed over $X$, where the projection $\sum_{x:X}A(x) \inj X$ is the inclusion of this subtype. We often write the sum as a collection,
\[ \scomp{x:X}{A(x)} := \sum_{x:X}A(x). \]
When $X$ in fact is a set, then any of its subtype is also a set, and we will emphasis by calling them \emph{subsets}.

Sets are used to define \emph{algebras} for a Horn theory. Given any Horn theory $\T$, by a \emph{$\T$-model} we will always mean a \emph{set} $A$ equipped with operations according to the signature of $\T$, such that it satisfies the axioms of $\T$. Since $A$ is a set, being a $\T$-model is again a proposition for a given family of operations.

At the end, we remark that $\N$ is the usual inductive type of natural numbers. For any $n:\N$, we will also use $n$ to denote the finite type with $n$ elements, which is viewed as an initial segment of $\N$. We stress that this is only syntactic sugar, and can be made more precise if need be.


\section{Quasi-coherence and affine spaces}

In this section we discuss the notion of quasi-coherent algebra and affine spaces. The technical results presented in this section is essentially contained in~\cite{Cherubini_Coquand_Hutzler_2024}, based on the work in~\cite{blechschmidt2021using}. 

As a starting point, we assume there is a generic $\T$-model in our type system:

\begin{axiom}
  There is a set $\I$ equipped with a $\T$-model structure.
\end{axiom}

\emph{Quasi-coherence} is a condition on \emph{$\I$-algebras}: An $\I$-algebra is a $\T$-model $A$ equipped with a homomorphism $\I \to A$. The type of $\I$-algebras will be denoted as $\I\alg$. More generally, for any $\T$-algebras $A,B$, we will use $\I\alg(A,B)$ to denote the set of homomorphisms of $\I$-algebras from $A$ to $B$. For instance, this can be used to define the \emph{spectrum} $A$,
\[ \spec A := \I\alg(A,\I). \]

A lot of sets can be naturally viewed as the spectrum of some $\I$-algebra. For the simplest example, since $\I$ is the initial $\I$-algebra,
\[ \spec \I = \I\alg(\I,\I) = 1. \]
Now if we use $\I[x]$ to denote the free $\I$ algebra on one generator, we can identify $\I$ itself as a spectrum,
\[ \spec \I[x] = \I\alg(\I[x],\I) = \I. \]
However, notationally we will distinguish $\I$ as an algebraic gagdet and as a spectrum, which we view as a geometrical object. The latter will be denoted as $\I$, and the equivalence $\I \to \I$ takes $i : \I$ to the evaluation map $\ev_i : \I[x] \to \I$. More generally, we have:

\begin{example}\label{exm:cubeaffine}
  For any $n:\N$, the $n$-th cube $\I^n$ is the following spectrum,
  \[ \spec \I[n] = \I\alg(\I[n],\I) = \I^n. \]
\end{example}

\begin{definition}
  $A$ is \emph{quasi-coherent}, if the canonical evaluation
  \[ A \to \I^{\spec A} \]
  is an equivalence of $\I$-algebras, where $\I^{\spec A}$ has the point-wise $\I$-algebra structure. We write $\ms{isQC}(A)$ for the proposition of being quasi-coherent.
\end{definition}

\begin{example}\label{exm:intervalqc}
  $\I$ itself by definition is quasi-coherent. We have seen that $\spec \I = 1$. Under this equivalence, the canonical map 
  \[ \I \to \I^{\spec \I} = \I \]
  is exactly the identity on $\I$, hence is an equivalence.
\end{example}

Note by construction, $\spec$ is a contravariant functor on $\I$-algebras. For quasi-coherent algebras, we have a more general duality result:

\begin{proposition}\label{prop:duality}
  If $B$ is quasi-coherent, then for any $\I$-algebras $A$, we have
  \[ \I\alg(B,A) = \spec B^{\spec A}. \]
\end{proposition}
\begin{proof}
  By quasi-coherence of $B$, we have the following equivalences,
  \[ \I\alg(A,B) = \I\alg(A,\I^{\spec B}) = \I\alg(A,\I)^{\spec B} = \spec A^{\spec B}. \]
  The second identity holds since by quasi-coherence the $\I$-algebra structure on $\I^{\spec B}$ is point-wise.
\end{proof}

The above duality result tells us that we can infer equivalence of algebras via equivalence of their spectrum:

\begin{corollary}\label{cor:dualityeqv}
  For any $f : \I\alg(A,B)$, if $A,B$ are quasi-coherent, then $f$ is an equivalence iff it induces an equivalence $\spec B \to \spec A$. 
\end{corollary}

To illustrate the usefulness of the above duality result, let us also introduce the following notion: 
\begin{definition}
  We say an $\I$-algebra $A$ is \emph{stably quasi-coherent} if all finitely generated principle congruences of $A$ are also quasi-coherent, i.e. for any finite $n : \N$ and for any $a,b : n \to A$, $A/a = b$ is again quasi-coherent.
\end{definition}
 
By definition, any finitely generated principle congruences of a stably quasi-coherent algebra will again be stably quasi-coherent. For a stably quasi-coherent $\I$-algebra $A$,  its spectrum always ``has enough points'' to distinguish elements:

\begin{lemma}\label{lem:completeness}
  For any stably quasi-coherent $\I$-algebra $A$ and for any $a,b : A$,
  \[ a = b \eq \fa x{\spec A} xa = xb. \]
\end{lemma}
\begin{proof}
  The $\nt$ direction is trivial. For $\Leftarrow$, by the universal property of the quotient $A/a = b$, we have
  \[ \spec(A/a = b) = \scomp{x : \spec A}{xa = xb} \subseteq \spec A. \]
  If $\fa x{\spec A}xa = xb$ holds, then the inclusion $\spec(A/a = b) \inj \spec A$ is an equivalence. Since by assumption $A/a=b$ is also quasi-coherent, by Corollary~\ref{cor:dualityeqv} the quotient $A \surj A/a=b$ is an equivalence. This exactly says $a = b$.
\end{proof}

Dually, we can use quasi-coherence to define the notion of \emph{affine spaces}:

\begin{definition}
  We say a set $X$ is \emph{affine}, if $\I^X$ is quasi-coherent, and the canonical evaluation map
  \[ X \to \spec \I^X \]
  is an equivalence. Similarly, we say $X$ is \emph{stably affine} if $\I^X$ is furthermore stably quasi-coherent.
\end{definition}

\begin{proposition}
  For any set $X$, $X$ is affine iff there exists a quasi-coherent $A$ such that $X = \spec A$.
\end{proposition}
\begin{proof}
  The only if direction holds by definition, since $A$ can be taken as $\I^X$. For the if direction, it suffices to show $\spec A$ is affine whenever $A$ is quasi-coherent. Now by quasi-coherence, $A = \I^{\spec A}$ thus $\I^{\spec A}$ is quasi-coherent. Furthermore, the canonical map $\spec A \to \spec \I^{\spec A}$ is an equivalence again by quasi-coherence of $A$.
\end{proof}

At the end of this section, we describe finite limits of affine spaces. By Example~\ref{exm:intervalqc}, the terminal type $1$ is affine. More generally, pullbacks of affine spaces can be computed as a spectrum:

\begin{proposition}\label{prop:pullbackofaffine}
  Let $A,B,C$ be $\I$-algebras. Given $f : \I\alg(B,A)$ and $g : \I\alg(B,C)$, the pullback of $\spec f$ and $\spec g$ is given by the spectrum
  \[ \spec A \times_{\spec B} \spec C = \spec (A \otimes_B C), \]
  where $A \otimes_B C$ is the following pushout of $\I$-algebras,
  \[
    \begin{tikzcd}
      B \ar[r, "f"] \ar[d, "g"'] & A \ar[d] \\
      C \ar[r] & A \otimes_B C
      \arrow["\lrcorner"{anchor=center, pos=0.125, rotate=180}, draw=none, from=2-2, to=1-1]
    \end{tikzcd}
  \]
\end{proposition}
\begin{proof}
  This essentially holds by the universal property of the pushout.
\end{proof}

\begin{remark}
  Notice that in the above result, if $\spec B$ is affine, then by Proposition~\ref{prop:duality} any map from $\spec A$ to $\spec B$ is induced by an $\I$-algebra morphism, and similarly for $\spec C$. Thus, any such pullback will be computed as a spectrum.
\end{remark}

Following~\cite{Cherubini_Coquand_Hutzler_2024}, the above result provides an extremely easy formula to compute the coproducts of quasi-coherent algebras:

\begin{corollary}\label{cor:tensorasspace}
  For any $\I$-algebra $A,B$, if $B$ is quasi-coherent, then there is an equivalence
  \[ \I^{\spec A \otimes B} = B^{\spec A}. \]
\end{corollary}
\begin{proof}
  Since $1$ is affine, thus $\spec A \times \spec B$ can be computed as $\spec A \otimes B$. Then by quasi-coherence of $B$, we have
  \[ \I^{\spec(A \otimes B)} = \I^{\spec A \times \spec B} = B^{\spec A}. \]
  Following this equivalence, the evaluation map $A \otimes B \to \I^{\spec A \otimes B}$ is exactly the one given above.
\end{proof}

Under the above identification, if $B$ is quasi-coherent, then the canonical map $A \otimes B \to \I^{\spec A \otimes B}$ can be equivalently viewed as the following map 
\[ c \mapsto \ld{x}{\spec A} (x\otimes B)(c) : A \otimes B \to B^{\spec A}. \]
Of course when $A \otimes B$ is quasi-coherent, then the above map will again be an equivalence. We can similarly calculate the canonical inclusions. For the left inclusion $A \to A \otimes B$, given any $a : A$ and $x : \spec A$,
\[ (x \otimes B)(a \otimes 1) = xa, \]
where we view $xa : \I$ as an element in $B$ using its an $\I$-algebra. For the right inclusion $B \to A \otimes B = B^{\spec A}$, for any $b : B$ and $x : \spec A$,
\[ (x \otimes B)(1 \otimes b) = b, \]
which implies the image of $b$ is simply the constant function on $b$.


\section{Open propositions and dominance}\label{sec:dominance}

Notice that up till this piont we have not used any spectial property of $\T$ rather than the fact that it is a Horn theory. However, to move closer to the intended application in synthetic domain theory, we start by assuming our theory $\T$ is \emph{propositional}: 

\begin{definition}
  We say a Horn theory $\T$ is \emph{propositional}, if it extends the theory of meet-semi-lattices, and truth of an element is computed by slicing: For any $\T$-model $A$ and $a:A$, the quotient $A/a=1$ is given by
  \[ a \wedge - : A \surj A/a, \]
  where $A/a$ by definition is $\cv a := \scomp{b:A}{b\le a}$.
\end{definition}

\begin{remark}
  The theory $\mbb M$ of meet-semi-lattices, $\mbb D$ of distributive lattices, $\mbb H$ of Heyting algebras, and $\mbb B$ of Boolean algebras are all examples of propositional theories in the above sense. In fact, all finitary quotients of $\mbb H$ or $\mbb B$ will again be propositional. More generally, for any propositional theory $\T$ and any $\T$-model $D$, the theory of $D$-algebras will again be propositional. We call such a theory a theory of $\T$-algebra.
\end{remark}

For a propositional theory $\T$, we think of the generic model $\I$ as certain interval object, since it is equipped with a partial order. For propositional theories, more sets can be realised as spectrums. The important examples are \emph{simplices}:

\begin{example}\label{exm:simplicesaffine}
  For any $n : \N$, let $\Delta^n \subseteq \I^n$ be the following subset,
  \[ \Delta^n := \scomp{i : n \to \I}{i_1 \ge \cdots \ge i_n}. \]
  This type is indeed a spectrum, since by definition we have
  \[ \Delta^n = \spec\I[i_1,\cdots,i_n]/i_1\ge\cdots\ge i_n. \]
\end{example}

The constant $1 : \I$, which is the top element in $\I$, induces a predicate
\[ \ms t : \I \to \pp, \]
which takes $i : \I$ to the proposition $i = 1$. The first observation is that $\ms t$ takes $i : \I$ to a spectrum:

\begin{lemma}\label{lem:openpropaffine}
  For any $i : \I$, $\ms ti = \spec \I/i$.
\end{lemma}
\begin{proof}
  By definition, $\spec \I/i = \I\alg(\I/i,\I)$. Since $\I \surj \I/i$ is a quotient, there is a homomorphism from $\I/i$ to $\I$ iff $i = 1$, and in this case the map is unique.
\end{proof}

To say something about the propositions that lies in $\ms t$, we assume a minimal amount of axioms:

\begin{axiom}[SQCI]
  $\I$ is stably quasi-coherent.
\end{axiom}

We already know that $\I$ is quasi-coherent. If it is stably so, then in particular for any $i : \I$, the principle congruence $\I/i$ will also be (stably) quasi-coherent. This implies the following result:

\begin{lemma}[SQCI]\label{lem:intconserve}
  The interval $\I$ is conservative,
  \[ \fa{i,j}{\I} (\ms ti \eq \ms tj) \eq i = j. \]
\end{lemma}
\begin{proof}
  It suffices to show
  \[ \fa{i,j}{\I}(\ms ti \to \ms tj) \to i \le j. \]
  Take $i,j$ with $\ms ti \to \ms tj$. By Lemma~\ref{lem:openpropaffine} and (SQCI), each $\ms ti$ and $\ms tj$ will be affine. Then we get a restriction map between $\I$-algebras,
  \[ \I/j = \I^{\ms tj} \to \I^{\ms ti} = \I/i. \]
  By the universal property of the quotient and the fact that $\T$ is a propositional theory, such a map exists iff $i \le j$, which is the desired result.
\end{proof}

Thus, under (SQCI), we may view the generic algebra $\I$ as a subuniverse of \emph{open} propositions via the embedding $\ms t$:

\begin{definition}
  Given a proposition $p$, we say it is \emph{open} if $p$ is of the form $\ms ti$ for some $i$ in $\I$,
  \[ \ms{isopen}(p) := \sum_{i:\I}p \eq \ms ti. \]
\end{definition}

By Lemma~\ref{lem:intconserve}, the $i$ that $p \eq \ms ti$ will be unique, thus being open is a proposition. Also, open propositions are evidently closed under finite conjunctions, since $\ms t$ preserves them. More generally, we may define the notion of open subtypes:

\begin{definition}
  A subtype $U$ of $X$ is \emph{open}, if for any $x:X$ the proposition $U(x)$ is open,
  \[ \ms{isOpen}(U) := \fa xX \ms{isopen}(U(x)). \]
\end{definition}

We write $\ms{Open}(X)$ as the type of open subtypes of $X$, which by definition is equivalent to $\I^X$. Note that this crucial depends on conservativity of $\I$.

\begin{lemma}[SQCI]\label{lem:openofaffinegivesalgebra}
  If $X = \spec A$ is affine, then the type of open subsets of $X$ is equivalent to $A$.
\end{lemma}
\begin{proof}
  Since $\ms{Open}(X)$ is simply $\I^X$, this is implied by the quasi-coherence principle.
\end{proof}

To account for the above equivalence in more detail, let $A$ be a quasi-coherent $\I$-algebra. For any $a : A$, the open subset of $\spec A$ induced by $a$ is given by
\[ D_a := \scomp{x : \spec A}{\ms t(xa)}. \]
$D_a$ is again a spectrum, since by definition the quotient $A \surj A/a$ induces an embedding $\spec A/a \inj \spec A$, given by
\[ \spec A/a = \scomp{x : \spec A}{\ms t(xa)} = D_a. \]
Indeed, if $A$ is furthermore stably quasi-coherent, then so is $D_a$.

One observation from~\cite{Cherubini_Coquand_Hutzler_2024} is that the type of open propositions forms a \emph{dominance} in the sense of~\cite{rosolini1986continuity}. In other words, open propositions are closed under truth and dependent product. The \emph{loc. cit.} uses certain choice principle on affine spaces to show this. However, in our situation the conservativity of the interval $\I$ allows direct computation of the type of open subsets of affine spaces according to Lemma~\ref{lem:openofaffinegivesalgebra}. Hence, our proof does not require any additional choice principle:

\begin{proposition}[SQCI]\label{prop:Idominance}
  The type of opens $\I$ forms a dominance.
\end{proposition}
\begin{proof}
  Suppose $p$ is an open proposition. By definition $p = \ms ti$ for some $i:\I$. Now if $q$ is an open subset of $p$, since $p = \ms ti$ is affine, Lemma~\ref{lem:openofaffinegivesalgebra} implies that $q = D_j$ for some $j : \I/i$. Since $\T$ is propositional, equivalently $j$ can be viewed as an element $j : \I$ with $j \le i$. This way, $q = \ms tj$, which is again open.
\end{proof}


\section{Lifting}\label{sec:lifting}

Given then dominance structure $\I$, we can construct internally the lifting functor. For any type $X$, its lift is given by
\[ X\prt := \sum_{i:\I}X^{\ms ti}. \]
The functoriality is easy to express: For any $f : X \to Y$, we have
\[ f\prt(i,x) := (i,\ld{w}{\ms ti}fxw). \]
There is an evident unit $\eta\prt : X \to X\prt$, where
\[ \eta := \ld x X(1,\ld\hp 1 x). \]
The dominance structure on $\I$ also gives a multiplication $\mu : (X\prt)\prt \to X\prt$, where $\mu$ takes any $(i,u)$ with $i : \I$ and $u : \ms ti \to X\prt$ first to $(j,x)$, where $j$ is the dependent sum
\[ j := \sum_{w:\ms ti} (uw)_0, \]
and $x : \ms tj \to X$ is the partial element such that for $w : \ms ti$ and $v : (uw)_0$
\[ x(w,v) := (uw)_1(v). \]

\begin{example}
  By definition, it is easy to see that
  \[ 1\prt = \sum_{i:\I}\ms ti = \I. \]
\end{example}

For synthetic domain theory, the object of particular importance is the lift of the type of open propositions $\I$ itself. In fact, the lifting of any stably quasi-coherent $\I$-algebra can be computed fairely explicitly:

\begin{lemma}\label{lem:liftingofalgebra}
  If $A$ is stably quasi-coherent, then we have
  \[ A\prt = \scomp{i : \I, a : A}{a \le i}. \]
\end{lemma}
\begin{proof}
  Note for any $i : \I$, the quotient $A/i$ is again quasi-coherent. Now notice that we do have
  \[ A/i = A \otimes \I/i. \]
  By Corollary~\ref{cor:tensorasspace} and quasi-coherence of $A/i$, 
  \[ A/i = A \otimes \I/i = A^{\spec \I/i} = A^{\ms ti}. \]
  This way, it follows that 
  \[ A\prt = \sum_{i:\I}A^{\ms ti} = \sum_{i:\I}A/i = \scomp{i : \I, a : A}{a \le i}. \qedhere \]
\end{proof}

\begin{corollary}[SQCI]
  For the interval $\I$, $\I\prt = \Delta^2$.
\end{corollary}
\begin{proof}
  By Lemma~\ref{lem:liftingofalgebra}, 
  \[ \I\prt = \scomp{i,j : \I}{i \ge j} = \Delta^2. \]
  The second equivalence again uses the fact that $\T$ is propositional.
\end{proof}

\begin{remark}
  One interesting thing to notice here is that, though $\I\prt$ by computation is a dependent sum of algebras, it is naturally equivalent to a \emph{spectrum}, which is a geometric object. In some sense the source is the assumption that $\T$ is propositional, which allows us to identify the algebraic object $\I/i$ as a subset $\scomp{j : \I}{j \le i}$.
\end{remark}


More generally, for domain theoretic applications we would want to compute the liftings of the simplices introduced as spectrums in Example~\ref{exm:simplicesaffine}. It is first of all easy to see that the lifting of any affine space is also straight forward to compute by duality:

\begin{lemma}\label{lem:liftofaffine}
  If $X = \spec A$ is affine, then the lifting of $X$ is given by
  \[ X\prt = \sum_{i:\I}\I\alg(A,\I/i). \]
\end{lemma}
\begin{proof}
  By Proposition~\ref{prop:duality}, since $X = \spec A$ is affine,
  \[ X\prt = \sum_{i:\I}X^{\ms ti} = \sum_{i:\I}\spec A^{\spec \I/i} = \sum_{i:\I}\I\alg(A,\I/i). \qedhere \]
\end{proof}

Thus, motivated by domain theory, we find ourselves naturally move towards the following axiomatisation:

\begin{axiom}[SQCF]
  All finitely generated free $\I$-algebras, i.e. $\I[n]$ for $n : \N$, are stably quasi-coherent.
\end{axiom}

Of course, (SQCF) implies (SQCI) when taking $n$ to be 0. Furthermore, (SQCF) implies that the simplices $\Delta^n$ are now (stably) affine as well. This way, we can indeed compute their lifts:

\begin{lemma}[SQCF]
  For any $n : \N$, we have
  \[ \Delta^n\prt = \Delta^{n+1}. \]
\end{lemma}
\begin{proof}
  By Lemma~\ref{lem:liftofaffine}, for the affine space $\Delta^n = \spec\I[n]/i_1 \ge \cdots \ge i_n$,
  \begin{align*}
    \Delta^n\prt 
    &= \sum_{i:\I}\I\alg(\I[n]/i_1\ge\cdots\ge i_n,\I/i) \\
    &= \scomp{i,i_1,\cdots,i_n:\I}{i \ge i_1 \ge \cdots \ge i_n} \\
    &= \Delta^{n+1}
  \end{align*}
  The second equality again uses the fact that $\T$ is propositional.
\end{proof}

In particular, from a geometric perspective, the unit
\[ \eta : \Delta^n \inj \Delta^n\prt = \Delta^{n+1} \]
takes $i_1 \ge \cdots \ge i_n$ in $\Delta^n$ to $1 \ge i_1 \ge \cdots \ge i_n$ in $\Delta^{n+1}$. 

\section{Local geometry}\label{sec:nullstellensatz_and_local_geometry}

Besides the quasi-coherence principle, motivated by applications in domain theory, we would like to build models where the interval $\I$ might have more structures. For instance, to account for divergency in computation, one minimal assumption is to require that the type of open propositions $\I$ is also closed under \emph{falsum}. For some other purposes, we might want the dominance $\I$ to be closed under finite \emph{disjunctions} as well.

For this reason, we will make additional assumptions on our theory $\T$:

\begin{definition}
  We say a theory $\T$ is \emph{bipropositional}, if it extends the theory of distributive lattices, such that truth of an element is computed by slicing and falsity of an element is computed by coslicing.
\end{definition}

In other words, a bipropositional theory is a propositional theory which also has a distributive lattice structure, and furthermore the quotients of the form $A/a=0$ are computed by,
\[ a \vee - : A \surj a/A = A/a=0. \]
For instance, the theory $\mbb D$ of distributive lattices and $\mbb B$ of Boolean algebras are both bipropositional. Again, given a distributive lattice or a Boolean algebra $D$, the theory of $D$-algebras is bipropositional. However, the theory of Heyting algebras is \emph{not}. 

By our assumption, many of the results developped earlier will also be applicable in the dual statement, by exchaning $1$ to $0$ and $\wedge$ to $\vee$. For instance, the constant $0$ now allows us to define another predicate,
\[ \ms f : \I \to \pp, \]
which takes $i : \I$ to $i = 0$. We might similarly call propositions in the image of $\ms f$ as \emph{closed propositions}. Notice that the same proof of Lemma~\ref{lem:intconserve} still applies in the case of $\ms f$ by the assumption that $\T$ is bipropositional. Hence, (SQCI) still implies $\ms f$ is an embedding.

Similarly, following the development in Section~\ref{sec:dominance} and Section~\ref{sec:lifting}, we observe that under (SQCI), $\ms f$ will also makes the universe of closed propositions a dominance, and there is an accompanying \emph{colifting} monad $(-)\cprt$, defined by
\[ X\cprt := \sum_{i:\I} X^{\ms fi}. \]
And similarly, under (SQCF), we can explicitly compute the colifting of the simplices $\Delta^n$,
\[ \Delta^n\cprt = \Delta^{n+1}, \]
where now the unit $\eta : \Delta^n \inj \Delta^n\cprt = \Delta^{n+1}$ takes a sequence $i_1 \ge \cdots \ge i_n$ to $i_1 \ge \cdots \ge i_n \ge 0$ in $\Delta^{n+1}$. To distinguish the two inclusions $\Delta^n \inj \Delta^n\prt = \Delta^{n+1}$ and $\Delta^n \inj \Delta^{n}\cprt = \Delta^{n+1}$, we will denote the former as $\eta\prt$ and the latter as $\eta\cprt$.

\begin{remark}
  It is also interesting to look at the case of the theory $\mbb H$ of Heyting algebras. Though the above argument does not work, we do know that the quotient $\I/i=0$ is equivalent to $\I/\neg i$. In particular, any closed proposition is also open, but the converse is not true. Hence, by (SQCI), $\ms fi \eq \ms fj$ will imply $\neg i = \neg j$, but this does not mean $i = j$.
\end{remark}

To make the dominace $\I$ closed under falsum, we exploit the minimal element $0$. We may further assume the following non-triviality axiom:

\begin{axiom}[NT]\label{ax:nt}
  For $\I$, $0 \neq 1$.
\end{axiom}

\begin{remark}
  The above axiom semantically corresponds to working with certain subtopoi of the classifying topos $\Set[\T]$. For instance, the minimal topology on the underlying site $\mmod\T\fp$ we might choose for (NT) to hold is to assert that the trivial $\T$-model is covered by the empty sieve. Since the trivial $\T$-model is a strict terminal object, this topology will be subcanonical, thus (SQCF) still holds.
\end{remark}

As a first consequence, if (NT) holds, then the lifting monad $(-)\prt$ defined in Section~\ref{sec:lifting} is also pointed, where for any $X$ we can define
\[ \bot := (0,?) : X\prt, \]
where $? : \ms t0 = \emp \to X$ is the unique map from $\emp$. Similarly, the colifting monad is also pointed, with
\[ \top := (1,?) : X\cprt. \]

In particular, now $\emp = \ms t0 = \ms f1$ is affine, in fact both an open and closed proposition. For any $\I$-algebra $A$, we say $A$ is \emph{trivial} if it is equivalent to the trivial $\I$-algebra, which we denote as $0$. Equivalently, this is saying that $0 = 1$ in $A$. In the case of (NT), we have the following weak form of nullstellensatz:

\begin{lemma}[NT]\label{lem:nulls}
  For any affine $X = \spec A$, $X = \emp$ iff $A$ is trivial.
\end{lemma}
\begin{proof}
  By assumption, $A = \I^{\spec A} = \I^\emp = 0$. 
\end{proof}

Assuming (NT) allows us to infer much more structure of the interval. For instance, $\I$ satisfies the following field axioms as a consequence of nullstellensatz:

\begin{lemma}[NT, SQCI]\label{lem:field}
  $\I$ is a field in the sense that
  \[ \fa i{\I} \neg \ms ti \eq \ms fi, \quad \fa i\I \neg\ms fi \eq \ms ti. \]
  In particular, the embedding $2 \inj \I$ induced by $0,1$ is $\neg\neg$-dense,
  \[ \fa i\I \dneg(\ms ti \vee \ms fi). \]
\end{lemma}
\begin{proof}
  If $\ms fi$ then $\neg\ms ti$ by (NT). On the other hand, by Lemma~\ref{lem:intconserve} and (NT), if $i \neq 1$ then $i = 0$ since $0 \neq 1$. The dual case is completely similar. Now suppose $\neg(\ms ti \vee \ms fi)$, then $\neg\ms ti \wedge \neg \ms fi$, which equals $\ms fi \wedge \ms ti$, contradictory. Thus, we have $\dneg(\ms ti\vee\ms fi)$.
\end{proof}

This allows us to observe that the open and closed propositions are exactly complementary to each other:

\begin{corollary}[NT, SQCI]\label{cor:opendnegclose}
  For any proposition $p$, $p$ is open iff $\neg p$ is closed and vice versa. Furthermore, open and closed propositions are $\dneg$-stable.
\end{corollary}

To make the dominance $\I$ closed under finite joins, we exploit the join operator. We might impose the following locality principle on $\I$:

\begin{axiom}[L]
  $\I$ is non-trivial, and for $i,j : \I$, $\ms t(i \vee j) \eq (\ms ti \vee \ms tj)$.
\end{axiom}

\begin{lemma}[L, SQCI]
  The dominance $\I$ is closed under finite disjunctions.
\end{lemma}
\begin{proof}
  (L) is exactly saying the embedding $\ms t$ preserves finite joins.
\end{proof}

\begin{remark}
  Similarly to the case for (NT), assuming (L) semantically amounts to working in a suitable subtopos of $\Set[\T]$.
\end{remark}

Though the set up of this section also works for the theory $\mbb B$ of Boolean algebras, the content is in some sense trivialised in that case. Since by the presence of a classical negation, a proposition is open iff it is closed. Thus, for domain theoretic applications, our main focus will be a theory of distributive algebra, and from now on we do assume we work specifically in such a theory.

However, the structure of distributive lattices not only allows us to build more structures into the domiance $\I$, it is absolutely crucial for higher order axioms of domain theory, at least in the approach of modelling synthetic domain theory using classifying topoi of algebraic theories. The next section will showcase the first important example of how the algebraic structure affects the internal logic of the classifying topos.

\section{Intrinsic order and Phoa's principle}

Following~\cite{hyland2006first}, given the dominance $\I$ we may define an intrinsic order on each type as follows:

\begin{definition}
  The \emph{intrinsic order} on a type $X$ is defined as follows:
  \[ x \preceq y := \fa{U}{X\to\I} U(x) \le U(y). \]
\end{definition}

By definition, the intrinsic order is reflexive and transitive. As already observed in \emph{loc. cit.}, one important property of the intrinsic order is that \emph{every} map is monotone w.r.t. this order:

\begin{lemma}\label{lem:anymapmonotoneintriscorder}
  For $f : X \to Y$, $x \preceq y$ in $X$ implies $fx \preceq fy$ in $Y$.
\end{lemma}
\begin{proof}
  This simply follows from compositionality of functions.
\end{proof}

Just as how we have computed the lifting of affine spaces in Section~\ref{sec:lifting}, the quasi-coherence principle also determines the intrinsic order on affine spaces. For instance, if $X = \spec A$ is affine, then for any $x,y : X$,
\[ x \preceq y \eq \fa aA xa \le ya. \]
In other words, the order $\preceq$ on an affine space is the point-wise order as functions on $A$. In practice, we usually have an explicit presentation of the algebra $A$, and in this case we can more explicitly characterise the intrinsic order on $\spec A$:

\begin{lemma}
  Let $A = \I[X]/R$ be a quasi-coherent $\I$-algebra for some generating set $X$ and relation $R$. Then the intrinsic order on $\spec A$ is induced as a subspace of the point-wise order on $\spec\I[X] = \I^X$.
\end{lemma}
\begin{proof}
  By the previous discussion, for any $x,y : \spec A$ we have
  \[ x \preceq y \eq \fa p{\I[X]/R} xp \le yp. \]
  Now if we view $x,y$ as objects in $\spec \I[X] = \I^X$, then for any $p : \I[X]/R$, $xp$ is simply the evaluation of $p$ on $x$. Of course then $xp \le yp$ for all $p$ iff $x \le y$ point-wise, since $p$ can be chosen as the generating variables.
\end{proof}

\begin{example}[SQCF]
  For instance, the above result implies that the intrinsic order on the cubes $\I^n$ and on the simplices $\Delta^n$ are exactly the one we expect.
\end{example}

An interesting consequence of Lemma~\ref{lem:anymapmonotoneintriscorder} is that, all the maps between the above spaces are automatically \emph{monotone}. In particular, we have:

\begin{corollary}[SQCF]\label{cor:syntheticorderinterval}
  Any map $f : \I \to \I$ is monotone.
\end{corollary}

To characterise the intrinsic order, an important aspect involves the socalled \emph{Phoa's principle}, which relates it with functions out of $\I$:

\begin{definition}
  We say a type $X$ satisfies the \emph{Phoa's principle}, if its intrinsic order is classified by $\pair{\ev_0,\ev_1} : X^\I \to X \times X$.
\end{definition}

The Phoa's principle has many consequences. For instance, if $X$ satisfies the Phoa's principle, since the intrinsic order on a type is in particular a relation, the evaluation pair $X^\I \to X \times X$ will be an \emph{embedding}. Intuitively, it means that any map from the interval $f : \I \to X$ is determined by its boundary. Furthermore, since $\preceq$ is also \emph{transitive}, there will then be a well-defined \emph{composition} on functions from $\I$ to $X$. These are indeed certain internal \emph{orthogonality} classes that can be formulated in our type system. We will consider them in more detail in Section~\ref{sec:synposet}.

Another consequence of the Phoa's principle is that, it also implies an extensionality principle on intrinsic orders on function spaces:

\begin{lemma}[SQCF]\label{lem:phoaexponential}
  For any type $X,Y$, if $X$ satisfies the Phoa's principle, then the intrinsic order on $X^Y$ is induced point-wise by the intrinsic order on $X$, and also satisfies the Phoa's principle.
\end{lemma}
\begin{proof}
  Since $\I \to X^Y$ is equivalent to $Y \to X^\I$, it suffices to show that the intrinsic order on $X^Y$ is point-wise. For any $f,g : X^Y$, suppose $f \preceq g$. For any $y : Y$, since the evaluation map $\ev_y : X^Y \to X$ is monotone by Lemma~\ref{lem:anymapmonotoneintriscorder}, we have $fy \preceq gy$. On the other hand, suppose for any $y : Y$ we have $fy \preceq gy$. This way, by the Phoa's principle on $X$, we get a map $Y \to X^\I$, whose transpose 
  \[ [f,g] : \I \to X^Y, \]
  satisfies $[f,g](0) = f$ and $[f,g](1) = g$. Then consider any $U : X^Y \to \I$. Notice that by Corollary~\ref{cor:syntheticorderinterval} the composite $U[f,g]$ is monotone, thus
  \[ Uf = U[f,g](0) \le U[f,g](1) = Ug. \]
  Hence, indeed we have $f \preceq g$.
\end{proof}

Our first goal is to show that, indeed, any affine space will satisfy the Phoa's principle. We remark that the result below does \emph{not} involve the quansi-coherence assumption (SQCF). Rather, it holds purely by algebraic properties of distributive lattices:

\begin{theorem}\label{thm:phoaaffine}
  Any affine space $X$ satisfies the Phoa's principle.
\end{theorem}
\begin{proof}
  By Proposition~\ref{prop:duality}, if $X = \spec A$ is affine, then we have
  \[ X^\I = \I\alg(A,\I[i]). \]
  To this end, we observe that for distributive lattices, the following pair of evaluation maps on free algebra
  \[ \pair{\ev_0,\ev_1} : \I[i] \to \I \times \I \]
  classifies the order on $\I$. This follows from a normal form result of polynomials for distributive lattices, i.e. any polynomial $p$ is of the form 
  \[ p = p(0) \vee x \wedge p(1). \]
  See e.g.~\cite[Thm. 10.11]{lausch2000algebra}. This way, a map $\I \to X$ is exactly described as two points $x,y : X$ such that $\fa aA xa \le ya$, which coincides with $x \preceq y$.
\end{proof}

\begin{remark}
  The above proof crucially relies on the fact that for any distributive lattice $A$, the free algebra $A[i]$ is equivalent to the order on $A$. This is a perfect example of how an algebraic property of a theory has a non-trivial effect on the internal logic of its classifying topos.
\end{remark}

Just as the case for lifting in Section~\ref{sec:lifting}, our next goal is to characterise the intrinsic order on $\I$-algebras. However, this is automatic now given the previous results:

\begin{theorem}[SQCF]\label{them:phoaalgebra}
  For any quasi-coherent $\I$-algebra $A$, the intrinsic order on $A$ coincides with its usual order, and furthermore it also satisfies the Phoa's principle.
\end{theorem}
\begin{proof}
  Since $A$ is quasi-coherent, we have $A = \I^{\spec A}$. Now by Theorem~\ref{thm:phoaaffine}, the intrinsic order on $A$ coincides with the point-wise induced order on $\I$, which is indeed the usual order on $A$ because the equivalence $A = \I^{\spec A}$ is algebraic. Lemma~\ref{lem:phoaexponential} then also implies $A$ satisfies the Phoa's principle.
\end{proof}

As a consequence, we get some partial information on \emph{arbitrary} maps between $\I$-algebras:

\begin{corollary}[SQCF]
  If $A,B$ are quasi-coherent $\I$-algebras, then any map $f : A \to B$ is monotone.
\end{corollary}

At the end of this section, we describe another interesting perspective arising from the proof of Theorem~\ref{thm:phoaaffine}. We have seen that the dualising object $\I$ has a double role: It is both an algebra and a spectrum. The proof of Phoa's principle gives us many more such examples. For instance, $\I[i]$ classifies the order on $\I$, which by definition is also a spectrum. Now since for us the simplice $\Delta^2$ is the dual order, we will then identify it with $\I[i]$ along the following map,
\[ \pair{\ev_1,\ev_0} : \I[i] \to \Delta^2. \]
In fact, \emph{all} the simplices in this case have an algebraic description:

\begin{proposition}\label{prop:simplicesasalgebra}
  For any $n : \N$ and $\I$-algebra $A$, for any $a_0 \ge \cdots \ge a_n$ in $A$, define a polynomial in $A[i_1,\cdots,i_n]$ as follows,
  \[ a_0 \wedge i_1 \vee a_1 \wedge \cdots \wedge i_n \vee a_n. \]
  This is well-defined in the quotient $A[n]/i_1 \le \cdots \le i_n$, in the sense that no matter how you arrange the parenthesis there is a unique value in the quotient $A[n]/i_1 \le \cdots \le i_n$. In fact this gives us an equivalence
  \[ \Delta[A]^{n+1} = A[n]/i_1 \le \cdots \le i_n, \]
  where we write $\Delta[A]^{n+1}$ as $\scomp{a_0,\cdots,a_n : A}{a_0 \ge \cdots \ge a_n}$.
\end{proposition}
\begin{proof}
  This again follows from a normal form result on algebras of the form $A[n]/i_1 \le \cdots \le i_n$, which is a consequence of the general normal form for multivariable polynomials on distributive lattices; see~\cite[Thm. 10.21]{lausch2000algebra}.
\end{proof}

Now we can identify the simplices $\Delta^n$ as a finitely presented $\I$-algebra as shown above. Under such equivalences, the inclusion $\eta\prt$ simply corresponds to the canonical inclusion as shown below,
\[
  \begin{tikzcd}
    \Delta^{n+1} \ar[r, tail, "\eta\prt"] \ar[d, "\simeq"'] & \Delta^{n+2}  \ar[d, "\simeq"] \\ 
    \I[n]/i_1 \le \cdots \le i_n \ar[r, tail] & \I[n\!+\!1]/i_0 \le \cdots \le i_{n} 
  \end{tikzcd}
\]

% \section{Rice's theorem}

% In this section we will mainly consider \emph{algebras} for the lifting monad $(-)\prt$, which are affine spaces or $\I$-algebras. 

% \begin{lemma}
%   For any $X$, there is an equivalence
%   \[ \I^{X\prt} = (\I^X)\cprt. \]
% \end{lemma}
% \begin{proof}
%   First we define $\Phi : \I^{X\prt} \to (\I^X)\cprt$, for any $f : X\prt \to \I$, 
%   \[ \Phi f := (f\bot,). \]
% \end{proof}

% For starters, we consider the \emph{free} algebras of the form $X\prt$ for some type $X$. We will first characterise the intrinsic order on $X\prt$ following~\cite{hyland2006first}:

% \begin{lemma}[SQCF]\label{lem:orderonfreealgebra}
%   For any type $X$, the intrinsic order on $X\prt$ is given by
%   \[ (i,x) \preceq (j,y) \eq \ms ti \to (\ms tj \wedge x \preceq_X y). \]
% \end{lemma}
% \begin{proof}
%   For the left to right direction, suppose $(i,x) \preceq (j,y)$. Since we have a projection function $X\prt \to \I$, by monotonicity in Lemma~\ref{lem:anymapmonotoneintriscorder}, we indeed have $i \le j$, which by the proof of Lemma~\ref{lem:intconserve} is equivalent to $\ms ti \to \ms tj$. Now the remaining follows from the fact that $\I = 1\prt$ is an algebra for the lifting functor, thus any map $X \to \I$ extends to one $X\prt \to \I$. 

%   For the right to left direction, consider any $U : X\prt \to \I$. To show $U(i,x) \le U(j,y)$, it suffices to show $\ms tU(i,x) \to \ms tU(j,y)$. 
% \end{proof}

% The most crucial technical result in this section is that, the types satisfying Phoa's principle are closed under direct sums:

% \begin{lemma}
%   If $X$ satisfies Phoa's principle, and so is any $Y_x$ for a type family $Y : X \to \tp$ and $x : X$, then $\sum_{x:X}Y_x$ also satisfies the Phoa's principle.
% \end{lemma}
% \begin{proof}
%   We 
% \end{proof}



% Recall from Section~\ref{sec:nullstellensatz_and_local_geometry}, under the assumption (NT), the lifting monad is pointed. The interesting observatino is that this point in fact gives a bottom element in the intrinsic order:

% \begin{lemma}[NT]
%   For any type $X$, the point $\bot : 1 \to X\prt$ is a minimal element in the intrinsic order of $X\prt$.
% \end{lemma}
% \begin{proof}
%   We can directly compute the function type as follows,
%   \[ X\prt \to \I = \prth{\sum_{i:\I}X^{\ms ti}} \to \I = \prod_{i:\I} X^{\ms ti} \to \I. \]
%   Now suppose we have $(i,x) : X\prt$. For any $U : X\prt \to \I$,
%   \[ U(i,x) \]
% \end{proof}

% \begin{lemma}[NT, SQCF]
%   Let $X$ satisfies the Phoa's principle. Then an algebra structure on $X$ for the lifting monad is equivalently a point $x : 1 \to X$ which is a minimal element in the intrinsic order.
% \end{lemma}
% \begin{proof}
  
% \end{proof}

% \section{Synthetic posets}\label{sec:synposet}

% \begin{definition}
%   We say a type $X$ is \emph{$\I$-separated}, if $X^\I \to X^2$ is an embedding.
% \end{definition}  

% In type theory we can also define general shapes of boundaries $\partial\Delta^n$ and horns $\Lambda^n_k$ are definable. For instance, the horn $\Lambda^2_1$ will be constructed as the following pushout,
% \[
%   \begin{tikzcd}
%     1 \ar[d, "\bot"'] \ar[r, "\top"] & \I \ar[d] \\
%     \I \ar[r] & \Lambda^2_1
%     \arrow["\lrcorner"{anchor=center, pos=0.125, rotate=180}, draw=none, from=2-2, to=1-1]
%   \end{tikzcd}
% \]
% This gives us an embedding $\Lambda^2_1 \to \Delta^2$, which we can identify with
% \[ \Lambda^2_1 = \scomp{j \ge i : \I}{\ms tj \vee \ms fi}. \]
% A more complex construction is the walking equivalence, which we will denote as $E$. 

% The notion of synthetic posets is formulated internally as certain orthogonality conditions:

% \begin{definition}
%   For any type $X$, we say it is 
%   \begin{itemize}
%     \item \emph{$\I$-separated}, if $X^\I \to X^2$ is an embedding;
%     \item \emph{Segal}, if $X^{\Delta^2} \to X^{\Lambda^2_1}$ is an equivalence.
%     \item \emph{Rezk}, if $X \to X^E$ is an equivalence.
%   \end{itemize}
%   $X$ is a (synthetic) category, if $X$ Segal and Rezk. $X$ is a (synthetic) poset, if it is furthermore $\I$-separated. These are propositions.
% \end{definition}

% For $\I$-separatedness, it is in fact equivalently to being separated for the double negation topology, at least for sets:

% \begin{proposition}\label{prop:isepiffdnegclosed}
%   A set $X$ is $\I$-separated iff it is separated, i.e. for any $x,y$, $x = y$ is $\dneg$-closed.
% \end{proposition}
% \begin{proof}
%   By Corollary~\ref{lem:field} the inclusion $\mb 2 \inj \I$ is $\dneg$-dense, thus being separated implies being $\I$-separated. On the other hand, 
% \end{proof}

% From the above definition, it immediately follows that posets are closed under limits and retracts, and in fact forms an exponential ideal. For a non-trivial example, let us first show that $\I$ is a poset:

% \begin{lemma}\label{lem:intervalposet}
%   $\I$ is a poset.
% \end{lemma}
% \begin{proof}
%   By Phoa's principle, $\I$ is $\I$-separated. To show it is segal, consider a map $[f,g] : \Lambda^2_1 \to \I$, which is equivalently two maps $f : \I \to \I$ and $g : \I \to \I$ with $f(\bot) = g(\top)$. By Phoa's principle again, this is equivalently a sequence $g(\bot) \le g(\top) = f(\bot) \le f(\top)$. Now the pair $g(\bot) \le f(\top)$ defines a map $\Delta^2 \to \I$, which is easily seen to be unique.
% \end{proof}

% \begin{corollary}
%   All the simplices $\Delta^n$ are posets.
% \end{corollary}
% \begin{proof}
%   They are retracts of cubes $\I^n$, and since posets form an exponential ideal, $\I^n$ are posets due to Lemma~\ref{lem:intervalposet}.
% \end{proof}


% For another type of objects, all the algebraic objects we care about will be posets, and in fact they also satisfies the Phoa's principle:

% \begin{corollary}
%   Any f.p. $\I$-algebra $A$ is a poset, and in fact the canonical order on $A$ coincide with $A^\I$.
% \end{corollary}
% \begin{proof}
%   By (QC), for any f.p. $\I$-algebra $A$ we have $A = \I^{\spec A}$, which is a poset since they are closed under exponentials. 
% \end{proof}

% However, perhaps the more interesting examples of posets are affine spaces. For instance, $\I$ is the canonical example of an affine space being a poset. In fact, again by (QC), we can show \emph{all} affine spaces are posets:

% \begin{proposition}
%   Any affine space $\spec A$ is a poset.
% \end{proposition}
% \begin{proof}
%   By the duality given in Proposition~\ref{prop:duality}, we have
%   \[ \spec A^\I = \spec A^{\spec \I[x]} = \I\alg(A,\I[x]), \]
%   and under this equivalence, it is easy to see that the boundary map is now given by
%   \[ \pair{\ev_\bot,\ev_\top} : \I\alg(A,\I[x]) = \spec A^\I \to \spec A^2 = \I\alg(A,\I)^2. \]
%   But then $\spec A$ being a poset trivially follows from the Phoa's principle of $\I$, since $\I\alg(A,\I[x])$ now exactly classifies maps $f,g : A \to \I$ such that $f \le g$, because as the order $\I[x] \inj \I \times \I$, $\I[x]$ is in fact a \emph{subalgebra}.
% \end{proof}

% For all the above examples, they are indeed special cases of \emph{schemes}, which arise as glueing of affine spaces along open subsets. Intuitively, since $\I$ is tiny enough to make being a poset a local property, schemes would themselves again be posets.

\section{Infinitay domain theory}

Untill this point, we have seen that all the finitary axioms for synthetic domain theory is a consequence of (SQCF), with potentially the assumption of (NT) or (L), depending on which closure properties one want for the dominance $\I$. 

However, one of the most crucial axioms of synthetic domain theory is an infinitary one. From the observation in~\cite{hyland2006first,JIBLADZE1997185}, we can define internally in type theory the initial algebra and final coalgebra for the functor $(-)\prt$, which we denote as $\omega$ and $\ov\omega$, respectively. There is a canonical inclusion $\omega \inj \ov\omega$, and the final axiom for synthetic domain theory states that the interval $\I$ is \emph{complete}, in the sense that the induced map $\I^{\ov\omega} \to \I^\omega$ is an \emph{equivalence}. The most crucial observation in this section is that, is again a consequence of quasi-coherence. 

Since $(-)\prt$ by construction preserves all connected limits, the final coalgebra $\ov\omega$ can be easily characterised as a limit. As shown in~\cite{hyland2006first}, it can be identified as the object of infinite descending sequences in $\I$,
\[ \ov\omega := \scomp{i : \N \to \I}{\fa n\N i_n \ge i_{n+1}}, \]
which can be viewed as the following sequential limit,
\[
\begin{tikzcd}
  \cdots \ar[r] & \Delta^2 \ar[r, "!\prt"] & \I \ar[r, "!"] & 1
\end{tikzcd}
\]
where the transition map $\Delta^{n+1} \to \Delta^n$ takes the sequence $i_0 \ge \cdots \ge i_n$ to the final segment $i_1 \ge \cdots \ge i_n$. 

According to the above description, $\ov\omega$ is in fact a \emph{spectrum},
\[ \ov\omega = \spec(\I[\N]/\forall n.\, i_n \ge i_{n+1}). \]
However, the corresponding algebra is no longer finitely presented, but countably presented. By a countably presented, or c.p. in short, $\I$-algebra, we mean an $\I$-algebra of the form $\I[\N]/s=t$ for some $s,t : \N \to \I[\N]$. Motivated by the above characterisation of $\ov\omega$, we naturally consider the following stronger quasi-coherence principle:

\begin{axiom}[SQCC]
  All c.p. $\I$-algebras are (stably) quasi-coherent.
\end{axiom}

\begin{remark}
  The quasi-coherence principle for countably presented algebras is investigated in~\cite{cherubini2024foundation}. Just like the finitary version (SQCF), which is true in the classifying topos $\Set[\T] = [\mmod\T\fp,\Set]$, the countable version (SQCC) will be valid in a larger presheaf topos $[\mmod\T\cp,\Set]$, or any subtopos induced by a subcanonical topology. For instance, the semantics of~\cite{cherubini2024foundation} is based on the topos of light condensed sets introduced by Clausen and Scholze, which is a subtopos of $[\mmod{\mbb B}\cp,\Set]$.
\end{remark}

As a first example of the usefulness of (SQCC), let us observe that it implies the following form of Markov principle:

\begin{lemma}[NT, SQCC]\label{lem:markov}
  For any $i : \ov\omega$, we have
  \[ \neg\fa{n}{\N}\ms ti_n \to \ex n\N\ms fi_n. \]
\end{lemma}
\begin{proof}
  Let $i : \ov\omega$. Notice that similar to Lemma~\ref{lem:openpropaffine}, the proposition $\fa n\N \ms ti_n$ by construction is the following affine space, 
  \[ \spec\I/i = \I\alg(\I/i,\I) = \fa n\N \ms ti_n, \]
  where we have abbreviated the c.p. $\I$-algebra $\I/\forall k.\, i_k = 1$ as $\I/i$. Now if we have $\neg\fa n\N \ms ti_n$, then $\spec\I/i = \emp$ which implies $\I/i$ is trivial by Lemma~\ref{lem:nulls}. But this algebra is trivial iff there merely exists $n : \N$ that $\ms fi_n$ holds, thus $\ex n\N \ms fi_n$.
\end{proof}

Now let us consider the initial algebra $\omega$ for the lifting functor. It is shown in~\cite{JIBLADZE1997185} that we can identify $\omega \inj \ov\omega$ as the following subset,
\[ \omega := \scomp{i : \ov\omega}{\fa\phi{\pp} (\fa n{\N} (\ms ti_n \to \phi) \to \phi) \to \phi}. \]
For another proof, see e.g.~\cite{VANOOSTEN2000233}. In the presence of the Markov principle above, this description can be drastically simplified:

\begin{lemma}[NT, SQCC]\label{lem:omegacolimit}
  $\omega$ is equivalent to the following subset of $\ov\omega$,
  \[ \omega = \scomp{i : \ov\omega}{\ex n{\N} \ms fi_n}. \]
\end{lemma}
\begin{proof}
  Let $i : \ov\omega$. It suffices to show that
  \[ \prth{\fa\phi{\pp} (\fa n{\N} (\ms ti_n \to \phi) \to \phi) \to \phi} \to \ex{n}\N \ms fi_n, \]
  We can take $\phi$ to be $\emp$. By assumption we have $\neg\fa n\N \neg\neg\ms ti_n$, which by Corollary~\ref{cor:opendnegclose} is equivalent to $\neg\fa n\N \ms ti_n$. Then Lemma~\ref{lem:markov} implies this is $\ex n\N \ms fi_n$.
\end{proof}

By~\cite[Cor. 1.10]{VANOOSTEN2000233}, the above result exactly means that $\omega$ is indeed the internal colimit of the following sequence,
\[ 
\begin{tikzcd}
  \emp \ar[r, "?"] & 1 \ar[r, "?\prt = \eta\cprt"] & \I \ar[r, "\eta\cprt"] & \Delta^2 \ar[r] & \cdots
\end{tikzcd}
\]

Now we can show the most important infinitary axiom for synthetic domain theory:

\begin{theorem}[NT, SQCC]
  The canonical map $\I^{\ov\omega} \to \I^\omega$ is an equivalence.
\end{theorem}
\begin{proof}
  Since $\ov\omega$ is now affine, we have
  \[ \I^{\ov\omega} = \I[\N]/\forall n.\,i_n\ge i_{n+1}. \]
  On the other hand, since $\omega$ is internally the colimit of $\Delta^n$, we have
  \[ \I^\omega = \lt_{n:\N}\I^{\Delta^n} = \lt_{n:\N} \I[n]/i_1 \ge \cdots \ge i_n. \]
  Note that the transition maps induced by $\eta\cprt : \Delta^n \to \Delta^{n+1}$ under quasi-coherence gives us the following maps on algebras:
  \[
  \begin{tikzcd}
    \I^{\Delta^{n+1}} \ar[r, "\I^{\eta\cprt}"] & \I^{\Delta^n} \\ 
    \I[n\!+\!1]/i_1 \ge \cdots \ge i_{n+1} \ar[u, "\simeq"] \ar[r, "i_{n+1} \mapsto 0"'] & \I[n]/i_1 \ge \cdots \ge i_n \ar[u, "\simeq"']
  \end{tikzcd}
  \]
  Now taking the limit of the above sequence of algebras gives us
  \[ \lt_{n:\N}\I[n]/i_1 \ge \cdots \ge i_n = \I[\N]/\forall n.\,i_n \ge i_{n+1} = \I^{\ov\omega}. \]
  This shows that $\I^{\omega} = \I^{\ov\omega}$.
\end{proof}











\bibliographystyle{apalike} 
\bibliography{mybib}


\end{document}


