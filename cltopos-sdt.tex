\documentclass[12pt]{amsart}

\usepackage{amsmath,amssymb,amsthm,amsfonts}
\usepackage[numbers]{natbib}
\usepackage[english]{babel}
\usepackage[X2,T1]{fontenc}
\usepackage{libertine}
\usepackage{libertinust1math}
\usepackage[utf8]{inputenc} %'utf8' instead of 'latin1'
\usepackage{enumitem} 
\usepackage{tikz-cd}
\usepackage{hyperref}
\usepackage{stmaryrd}
\usepackage{appendix}
\usepackage{mathtools}
\usepackage{cleveref}
\usepackage{quiver}


\newtheorem{theorem}{Theorem}[section]
\newtheorem{fact}[theorem]{Fact}
\newtheorem{lemma}[theorem]{Lemma}
\newtheorem{conjecture}[theorem]{Conjecture}
\newtheorem{corollary}[theorem]{Corollary}
\newtheorem{claim}[theorem]{Claim}
\newtheorem{proposition}[theorem]{Proposition}
\newtheorem{observation}[theorem]{Observation}

\theoremstyle{definition}
\newtheorem{example}[theorem]{Example}
\newtheorem{definition}[theorem]{Definition}
\newtheorem{remark}[theorem]{Remark}
\newtheorem{question}[theorem]{Open Question}
\newtheorem{assumption}[theorem]{Assumption}
\newtheorem{qquestion}[theorem]{Question}
\newtheorem*{axiom}{Axiom}

\newcommand{\mc}[1]{\mathcal{#1}}
\newcommand{\mb}[1]{\mathbf{#1}}
\newcommand{\mbb}[1]{\mathbb{#1}}
\newcommand{\T}{\mbb T}
\newcommand{\I}{\mbb I}
\newcommand{\gn}[1]{\ulcorner\! #1 \!\urcorner}
\newcommand{\mr}[1]{\mathrm{#1}}
\newcommand{\mf}[1]{\mathfrak{#1}}
\newcommand{\ms}[1]{\mathsf{#1}}
\newcommand{\Ind}{\mathbf{Ind}}
\newcommand{\Pro}{\mathbf{Pro}}
\newcommand{\Set}{\mb{Set}}
\newcommand{\Prop}{\mb{Prop}}
\newcommand{\sSet}{\mb{sSet}}
\newcommand{\sSeto}{\mb{sSet}_{\le 1}}
\newcommand{\End}{\operatorname{End}}
\newcommand{\Hom}{\operatorname{Hom}}
\newcommand{\Fin}{\mb{Fin}}
\newcommand{\Cnt}{\mb{Cnt}}
\newcommand{\Enm}{\mb{Enm}}
\newcommand{\Grp}{\mb{Grp}}
\newcommand{\sub}{\mr{Sub}}
\newcommand{\Lex}{\mb{Lex}}
\newcommand{\Pos}{\mb{Pos}}
\newcommand{\alg}{\text{-}\mb{Alg}}
\newcommand{\Var}{\mb{Var}}
\newcommand{\Aff}{\mb{Aff}}
\newcommand{\Vect}{\mb{Vect}}
\newcommand{\CRing}{\mb{CRing}}
\newcommand{\DL}{\mb{DL}}
\newcommand{\BA}{\mb{BA}}
\newcommand{\HA}{\mb{HA}}
\newcommand{\JL}{\mb{JL}}
\newcommand{\ML}{\mb{ML}}
\newcommand{\ProMon}{\mb{ProMod}}
\newcommand{\CTopoi}{\mb{CTopoi}}
\newcommand{\PTc}{\mb{PT}_c}
\newcommand{\Topoi}{\mb{Topoi}}
\newcommand{\sh}{\mb{Sh}}
\newcommand{\psh}{\mb{Psh}}
\newcommand{\Cont}{\mb{Cont}}
\newcommand{\Cart}[1]{#1^\to_{\text{cart}}}
\newcommand{\Str}{\mb{Str}}
\newcommand{\op}{^{\mathrm{op}}}
\newcommand{\inv}{^{\mathrm{-1}}}
\newcommand{\pf}[1]{\widehat{#1}}
\newcommand{\qsi}[1]{\tilde{#1}}
\newcommand{\cob}{\vartriangleleft}
\newcommand{\other}{\mathrm{otherwise}}
\newcommand{\geo}[1]{\left|#1\right|}
\newcommand{\ov}[1]{\overline{#1}}
\newcommand{\set}[1]{\{\,#1\,\}}
\newcommand{\eff}{\Leftrightarrow}
\newcommand{\conjt}{\;\&\;}
\newcommand{\pair}[1]{\left\langle#1\right\rangle}
\newcommand{\id}{\mathrm{id}}
\newcommand{\ev}{\mathrm{ev}}
\newcommand{\elem}{\int\!\!}
\newcommand{\nt}{\Rightarrow}
\newcommand{\scomp}[2]{\{\,#1\mid#2\,\}}
\newcommand{\yon}{\mathtt{y}}
\newcommand{\surj}{\twoheadrightarrow}
\newcommand{\hook}{\hookrightarrow}
\newcommand{\cg}{\operatorname{\sim}}
\newcommand{\im}{\operatorname{img}}
\newcommand{\cgf}[2]{\leftindex_{#1}{\cg}_{#2}}
\newcommand{\coten}{\pitchfork}
\newcommand{\ppr}{\operatorname{\hat\times}}
\newcommand{\rl}{^{\perp}}
\newcommand{\llo}[1]{\leftindex_{}^{\perp} {#1}}
\newcommand{\dl}{^{\circ}}
\newcommand{\prth}[1]{\left(#1\right)}
\newcommand{\fn}{_{\mr{f.}}}
\newcommand{\fp}{_{\mr{f.p.}}}
\newcommand{\fpp}{_{\mr{f.p.,+}}}
\newcommand{\cp}{_{\mr{c.p.}}}
\newcommand{\cpp}{_{\mr{c.p.,+}}}
\newcommand{\can}{_{\mr{can}}}
\newcommand{\pls}{^+}
\newcommand{\mns}{^-}
\newcommand{\dv}{\operatorname{\uparrow}}
\newcommand{\cv}{\operatorname{\downarrow}}
\newcommand{\et}{_{\text{\'et}}}
\newcommand{\N}{\mb N}
\newcommand{\Q}{\mbb Q}
\newcommand{\Z}{\mbb Z}
\newcommand{\Deltao}{\Delta_{\le 1}}
\newcommand{\Deltaw}{\Delta_{\omega}}
\newcommand{\sk}{\ms{sk}}
\newcommand{\csk}{\ms{csk}}
\newcommand{\sInt}{\mb{sInt}}
\newcommand{\jcan}{J_{\mr{can}}}
\newcommand{\wCPO}{\omega\mb{CPO}}
\newcommand{\shape}{\operatorname{\smallint}}
\newcommand{\dneg}{\neg\neg}
\newcommand{\prt}{_{\bot}}
\newcommand{\cprt}{_{\top}}
\newcommand{\fa}[2]{\forall #1\!\colon\!\!#2.\ }
\newcommand{\ex}[2]{\exists #1\!\colon\!\!#2.\ }
\newcommand{\exu}[2]{\exists_! #1\!\colon\!\!#2.\ }
\newcommand{\ld}[2]{\lambda #1\!\colon\!\!#2.\ }
\newcommand{\subopen}{\subseteq_{\mbb \I}}
\newcommand{\emp}{\emptyset}
\newcommand{\eq}{\leftrightarrow}
\newcommand{\ass}[1]{\llbracket#1\rrbracket} %\usepackage{stmaryrd}
\newcommand{\pss}[1]{||#1||} %\usepackage{stmaryrd}
\newcommand{\pt}{\ms{pt}}
\newcommand{\tp}{\ms{Type}}
\newcommand{\pp}{\ms{Prop}}
\newcommand{\st}{\ms{Set}}
\newcommand{\cnt}{\ms{Cnt}}
\newcommand{\gp}{\ms{Grpd}}
\newcommand{\pcat}{\ms{PCat}}
\newcommand{\cat}{\ms{Cat}}
\newcommand{\pcatt}{\ms{PCAT}}
\newcommand{\catt}{\ms{CAT}}
\newcommand{\PCat}{\mb{PCat}}
\newcommand{\Cat}{\mb{Cat}}
\newcommand{\sFrm}{\sigma\mb{Frm}}
\newcommand{\Frm}{\mb{Frm}}
\newcommand{\Loc}{\mb{Loc}}
\newcommand{\PCAT}{\mb{PCAT}}
\newcommand{\CAT}{\mb{CAT}}
\newcommand{\Catt}{\mf{Cat}}
\newcommand{\CATT}{\mf{CAT}}
\newcommand{\Topp}{\mb{Top}}
\newcommand{\Top}{\mf{Top}}
\newcommand{\wTop}{\omega\mb{Top}}
\newcommand{\Tp}{\ms{TYPE}}
\newcommand{\Pp}{\ms{PROP}}
\newcommand{\St}{\ms{SET}}
\newcommand{\Gp}{\ms{GRPD}}
\newcommand{\fst}{\ms{Fin}}
\newcommand{\Mod}{\mb{Mod}}
\newcommand{\Spec}{\mb{Spec}}
% \newcommand{\wTop}{\omega\mb{Spec}}
\newcommand{\quot}[1]{/_{\pair{#1}}}
\newcommand{\List}{\ms{List}}
\newcommand{\hp}{\text{-}}
\newcommand{\PG}{\ms{PG}}
\newcommand{\uv}[1]{\underline{#1}}
\newcommand{\mmod}[1]{#1\text{-}\mathbf{Mod}}
\newcommand{\func}{\mb{Func}}
\newcommand{\tm}[1]{#1\text{-}\mathrm{Term}}
\newcommand{\eqn}[1]{#1\text{-}\mathrm{Eqn}}
\newcommand{\horn}[1]{#1\text{-}\mathrm{Horn}}
\newcommand{\gr}[2]{[#1|#2]}
\newcommand{\VT}{\mbb V_\T}
\newcommand{\spec}{\operatorname{Spec}}
\newcommand{\El}{\mr{El}}
\newcommand{\lan}{\ms{lan}}
\newcommand{\ran}{\ms{ran}}
\newcommand{\upp}{_{\ms U}}
\newcommand{\dsg}[1]{\!\pair{#1}}
\newcommand{\co}{\mr{Co}}
\DeclareFontFamily{U}{dmjhira}{}
\DeclareFontShape{U}{dmjhira}{m}{n}{ <-> dmjhira }{}
\DeclareRobustCommand{\yon}{\text{\usefont{U}{dmjhira}{m}{n}\symbol{"48}}}
\DeclareRobustCommand{\noy}{\text{\reflectbox{\yon}}\!}

\NewDocumentCommand\Jon{m}{\textcolor{red}{\textbf{Jon:~}#1}}

\makeatletter
\newcommand{\ct@}[2]{%
  \vtop{\m@th\ialign{##\cr
    \hfil$#1\operator@font lim$\hfil\cr
    \noalign{\nointerlineskip\kern1.5\ex@}#2\cr
    \noalign{\nointerlineskip\kern-\ex@}\cr}}%
}
\newcommand{\ct}{%
  \mathop{\mathpalette\ct@{\rightarrowfill@\textstyle}}\nmlimits@
}
\makeatother
\makeatletter
\newcommand{\lt@}[2]{%
  \vtop{\m@th\ialign{##\cr
    \hfil$#1\operator@font lim$\hfil\cr
    \noalign{\nointerlineskip\kern1.5\ex@}#2\cr
    \noalign{\nointerlineskip\kern-\ex@}\cr}}%
}
\newcommand{\lt}{%
  \mathop{\mathpalette\lt@{\leftarrowfill@\textstyle}}\nmlimits@
}
\makeatother

\title{Domains and classifying topoi}
\author{Lingyuan Ye}
\address{
Lingyuan \textsc{Ye} \newline
Department of Computer Science and Technology\newline
University of Cambridge\newline
Cambridge, UK\newline
\href{mailto:ye.lingyuan.ac@gmail.com}{\sf ye.lingyuan.ac@gmail.com}
}

\begin{document}
%

%
%\titlerunning{Abbreviated paper title}
% If the paper title is too long for the running head, you can set
% an abbreviated paper title here
%

%
% \Endhorrunning{L. Ye}
% First names are abbreviated in the running head.
% If there are more than two authors, 'et al.' is used.
%
% \institute{New College\\
\begin{abstract}
  This paper produces models of synthetic domain theory via classifying topoi of algebraic theories. Using the internal logic of a large family of classifying topoi of theories based on $\sigma$-frames, especially the \emph{quasi-coherence principle}, we prove the axioms of synthetic domain theory hold in such topoi. This provides a new set of techniques reasoning about the behaviour of domains in a synthetic context, which allows us to establish a family of important orthogonality conditions for some special types of domains.
\end{abstract}
%
\maketitle              % typeset the header of the contribution
%

\section{Introduction}\label{sec:intro}

\subsection{Synthetic domain theory}\label{subsec:sdt}

The proposal to develop a synthetic theory of domains was raised by Dana Scott in the 1980s. The thesis is that (pre)domains should be viewed simply as some special sets in a suitable universe, and any function between them should automatically respect the computational data associated to domains. Later Hyland~\cite{hyland2006first} gives an extensive list of the properties such a universe should satisfy.

Firstly, a universe for synthetic domains should contain an interval object $\I$ which induces an information order (or specialisation order from a topological perspective) on domains, and every function will automatically be monotone. \Jon{Let's be a little careful here; although in the classic SDT, the information order played an important role, it seems to me that without a deeper change to the architecture of SDT (e.g. augmenting $\I$ to something that captures non-propositional measurements), this viewpoint will not scale to higher domains. Can we be a little more agnostic?} An internal axiom that characterises the synthetic nature of this fact is the so-called \emph{Phoa's principle}: The function space $\I^\I$ should classify the order on $\I$. In other words, the functions from $\I$ to itself are completely determined by the order structure on $\I$. 

Additionally, $\I$ should also form a subuniverse of propositions closed under dependent sums, so as to form a dominance in the sense of~\cite{rosolini1986continuity}. The dominance structure is used to parametrise partial functions through a lifting operator $(-)\prt$ constructed from $\I$. 

\Jon{The completeness of the interval is not, in the usual sense, an “infinitary” axiom (e.g. in contrast to the axiom that $\omega$ is inductive, which is infinitary)—it is elementary.}
Besides these elementary axioms, the main infinitary axiom of synthetic domain theory is the \emph{completeness} of the interval: Let $\omega,\ov\omega$ be the carriers of the internal initial algebra and final coalgebra to the lifting functor $(-)\prt$, respectively. \Jon{I would like to reserve the $(-)\prt$ notation for the actual Sierpi\'nski cone in our research, as one day the two investigations may collide fruitfully; can we use $L$ or something instead?} There is a canonical inclusion $\omega \hook \ov\omega$, and the completeness axiom states that $\I$ is \emph{right orthogonal} to this inclusion in the sense that the canonically induced map $\I^{\ov\omega} \to \I^{\omega}$ should be an equivalence. This can be viewed as the synthetic version of the $\omega$-completeness of the Sierpi\'nski space in traditional domain theory.

\subsection{Classifying topoi and Phoa's principle}\label{subsec:classtopphoa}

The connection between Phoa's principle and classifying topoi is first observed in a recent work on synthetic category theory~\cite{gratzer2024directed}, and this work is inspired by this observation. \Jon{We should indeed cite that paper, but I believe that I was the one who suggested this to those guys. I think instead of attributing the idea to someone, which was kind of in the air, we can just explain the connection more straightforwardly.} \emph{A posteriori}, the crucial step for this observation is already contained in the work of Blass~\cite{RN879}.

To recap, let $\T$ be a (finitary) Horn theory. Recall that the classifying topos of $\T$ is given by the following presheaf category (cf.~\cite[D3.1]{johnstone2002sketches}),
\[ \Set[\T] := [\mmod\T\fp,\Set], \]
where $\mmod\T\fp$ is the category of finitely presented $\T$-algebras. In particular, there is a generic $\T$-model $U_\T$, such that any $\T$-model in a topos $\mc E$ is realised as the image of $U_\T$ under the inverse image part of a geometric morphism $\mc E \to \Set[\T]$. The observation of~\cite{RN879} is that, internally in $\Set[\T]$, the function space of $U_\T$ is completely characterised by polynomials,
\[ U_\T^{U_\T} \cong U_\T[x], \]
where $U_\T[x]$ denotes the free $\T$-model generated by an additional element, which is a  polynomial algebra on $U_\T$.

The observation of Gratzer~et~al.~\cite{gratzer2024directed} is then that, when $\T$ is taken to be the theory $\mbb D$ of \emph{distributive lattices} and $\I$ is the generic $\mbb D$-model, the equivalence $\I^\I \cong \I[x]$ implies the Phoa's principle. This is because for distributive lattices, $\I[x]$ classifies the order on $\I$. This is a perfect example of how the properties of a theory affects the internal logic of its classifying topos. We will see many more examples of this form in this paper.

\subsection{Quasi-coherence}\label{subsec:qc}

Recently, the work of Blass has been greatly generalised by Blechschmidt in his PhD thesis~\cite{blechschmidt2021using} (also see the unpublished note~\cite{blechschmidt2020general} by the same author), which identifies a stronger property satisfied by the generic model $U_\T$ in $\Set[\T]$, termed \emph{quasi-coherence}. The terminology is motivated by its application in algebraic geometry.

Let $A$ be a $U_\T$-algebra in $\Set[\T]$, i.e. it is an internal $\T$-model, equipped with a homomorphism $U_\T \to A$. The quasi-coherence principle in $\Set[\T]$ states that, for any \emph{finitely presented}, or in short \emph{f.p.}, $U_\T$-algebra $A$, if we define its \emph{spectrum} to be the internal type of $U_\T$-algebra homomorphisms
\[ \spec A := U_\T\alg(A,U_\T), \]
then the exponential transpose of the canonical evaluation map \[ A\times \spec A\to U_\T \]  in $\Set[\T]$ is an equivalence, i.e. we have $A \cong U_\T^{\spec A}$.
Types of the form $\spec A$ are also called \emph{affine}. This in fact induces an internal duality between f.p.\ $U_\T$-algebras and affine types. In particular, it implies Blass's result mentioned earlier. 

In fact, the appearance of finite presentation in the above formulation of quasi-coherence is not an essential limitation. Following the development in~\cite{blechschmidt2021using,blechschmidt2020general}, it is clear that the quasi-coherence principle for finitary Horn theories can be generalised to larger cardinality, if we enlarge the site of the classifying topos $\Set[\T]$. For instance, one can consider
\[ \Set[\T]_\omega := [\mmod\T\cp,\Set], \]
where $\mmod\T\cp$ denotes the category of \emph{countably presented}, or in short \emph{c.p.}, $\T$-models. In this case, countably presented $U_\T$-algebras will enjoying quasi-coherence in $\Set[\T]_\omega$. For instance, this is the theoretical basis in a recent approach for synthetic topology~\cite{cherubini2024foundation}, which exploits the quasi-coherence principle for countably presented Boolean algebras. Horn theories $\T$ whose operations have \emph{countable} arity can also be considered in this case, so long as the base topos $\Set$ satisfies countable choice.

Blechschmidt also points out that quasicoherence descends from the classifying (presheaf) topos $\Set[\T]$ to any subtopos containing the generic model $U_\T$; see Blechschmidt~\cite[Cor. 7.7]{blechschmidt2020general}. Logically, this corresponds to working in a geometric quotient of the Horn theory $\T$. \Jon{\emph{Any} geometric quotient?} Hence, in such a subtopos the generic model $U_\T$ will validates more geometric sequents as specified by the topology, which should be viewed as certain \emph{local properties}. For instance, the generic ring in~\cite{Cherubini_Coquand_Hutzler_2024} is assumed to be a \emph{local ring}, which is true in the Zariski topology.\Jon{This is already in Blechschmidt, so we should refer to it there.} In the case of synthetic Stone duality, Cherubini~et~al.\cite{cherubini2024foundation} assume the generic Boolean algebra is given by the discrete set 2, which is again only true in a suitably chosen topology. \Jon{Rather than saying things like ``... assume that the generic thing is ...'', which is not really what is happening, we should say something like ``force the generic thing to be ...''. In other word, nobody is assuming that the generic ring is local, but rather they are moving from looking at the generic ring to the generic local ring. (I want to avoid confusing this with times when you really can assume wlog that the ``generic thing is blah'', as in the case of Kock's observation that the generic local ring is a field, etc.)}


\subsection{Main contribution}

The aim of this paper is to explain the connection between quasi-coherence for \emph{distributive lattices} and \emph{$\sigma$-frames} with synthetic domain theory. The theory of distributive lattices is  As a first indication of why quasi-coherence for these theories might be useful for domain theory, we have observed that by taking the interval $\I$ to be the generic distributive lattice or $\sigma$-frame, many important objects for domain theory can be written as \emph{spectrums}. This include the cubes $\I^n$ (\Cref{exm:cubeaffine}), the simplices $\Delta^n$ (\Cref{exm:simplicesaffine}), and the final coalgebra $\ov\omega$ (\Cref{exm:ovomegaaffine}). The initial algebra $\omega$ will also be realised as a sequential colimit of spectrums (\Cref{prop:omegacolimit}). Here we outline the main results of this paper, and discuss some of their potential applications.

\emph{1) Quasi-coherence produces synthetic domain theory}: We show the principle of quasi-coherence suffices to account for \emph{all} essential axioms for synthetic domain theory. Specifically:

\emph{\ \ 1.1)} Work with distributive lattices or $\sigma$-frames, the quasi-coherence for f.p.\ algebras (in fact a weaker assumption; cf. \Cref{ax:sqci}) implies that the generic interval $\I$ forms a dominance (\Cref{prop:Idominance}). It also leads to an explicit computation of the lifting of quasi-coherent $\I$-algebras (\Cref{prop:liftingofalgebra}), and spectrums (\Cref{prop:liftofaffine}).

\emph{\ \ 1.2)} As mentioned, quasi-coherence for f.p.\ algebras implies the generic interval $\I$ satisfies the Phoa's principle. In fact, we can show more generally that any affine space will also satisfy the Phoa's principle (\Cref{thm:phoaaffine}), and similarly for quasi-coherent algebras (\Cref{thm:algebraphoa}).

\emph{\ \ 1.3)} Most importantly, working with $\sigma$-frames, the quasi-coherence for c.p.\ algebras (plus the minor assumption that $\I$ is non-trivial) implies completeness of $\I$ (\Cref{thm:complete}). This is the only place where we have to work with $\sigma$-frames and not distributive lattices (cf.~\Cref{rem:whynotdis}) This gives a full account of the axioms for synthetic domain theory as specified in~\cite{hyland2006first}. Additionally, we can show the initial algebra $\omega$ is characterised by a sequential colimit (\Cref{prop:omegacolimit}).

\emph{2) Local properties of spectrums}: Though in general we do not assume the generic interval $\I$ to satisfy any local property such as linearity, we do discuss many instances of them and investigate their consequences.

\emph{\ \ 2.1)} \Cref{sec:locality} introduces various local conditions for $\I$, all of which will be compatible with quasi-coherence. This showcases the flexibility of this framework to encompass different flavours of domain theory.

\emph{\ \ 2.2)} More interestingly, without assuming the local properties globally, one can still show that $\I$ will be \emph{orthogonal} to the maps that classifies these local properties (\Cref{specisnontrivial,specissimplicial,specis1t}). In general, any limiting diagram of quasi-coherent algebras will induce a localisation class containing $\I$; cf. \Cref{rem:limofalgloc}. This exhibits a new type of techniques in reasoning about domains in this framework.

\emph{\ \ 2.3)} As a special case of locality, we also connect to the recent approach of synthetic (higher) category theory~\cite{riehl2017type,buchholtz2021synthetic,gratzer2024directed}. In particular, we show spectrums will be synthetic categories (\Cref{thm:affineposet}). As another example, we also show $\omega$ is a synthetic category, in fact it satisfies \emph{all} the orthogonality conditions discussed in this paper (\Cref{thm:omegaortho}). 

\emph{3) New models for synthetic domain theory}: As a consequence, a large family of new models for synthetic domain theory based on classifying topoi for $\sigma$-frames emerges. We will discuss these models in \Cref{sec:model}, and compare them to the existing sheaf models for synthetic domain theory~\cite{FIORE1997151} in \Cref{subsec:compare}.

\subsection{Style and notation of the paper}

Our motivation for domain theory encourages us to work in a context as general as possible, so as to allow future developments that connect more specific flavours of domain theory existing in the literature. This means that besides \Cref{sec:model} where we discuss models, throughout the paper we will work constructively in a type theory enriched with a generic model $\I$. The base type system we work with is intensional type theory with a universe satisfying function extensionality, which will be valid in any ($\infty$-)topos. 

We do not assume any additional assumption globally. Whenever we introduce an assumption under the \textbf{Axiom} environment, it should be viewed as introducing the \emph{content} of that assumption, rather than assuming it directly afterwards. In particular, the development is completely modular, and any additional assumption will be explicitly mentioned in each result.

In fact, we even take a step further. For the first half of this paper we will not work with distributive lattices specifically. Instead, we assume to work with an arbitrary \emph{propositional} theory (\Cref{defn:propositional}). In particular, the construction of the dominance and the computation of lifting mentioned in 1.1) works in this generality. Only starting from \Cref{sec:locality} will we then work more specifically with theories based on distributive lattices.

\subsection*{Notation and preliminaries}

By \emph{proposition} and \emph{set}, we will always use them in the sense of~\cite{hottbook}, i.e. $-1$-types and $0$-types. The subuniverse of propositions and sets will be denoted as $\pp$ and $\st$, respectively. Propositions are used to define subtypes $P : X \to \pp$, and we also write the dependent sum $\sum_{x:X}P(x)$ suggestively as $\scomp{x:X}{P(x)}$. Notice that function extensionality implies $\pp$ and $\st$ are both closed under dependent product, and $\st$ furthermore is closed under dependent sums. To emphasis, we will also use $\fa xXP(x)$ to denote the dependent product $\prod_{x:X}P(x)$. We will also write $\ex xXP(x)$ for the propositional truncation $\pss{\sum_{x:X}P(x)}$. In this case we say there \emph{merely} exists $x:X$ that $P(x)$. Similarly, we use $P \vee Q$ to mean the truncated proposition $\pss{P + Q}$. We will use $\N$ to denote the type of natural numbers. For $n:\N$, we will also abuse notation and treat $n$ as the finite type of $n$ elements. 

Given an $\I$-algebra $A$, we will abuse notation by identifying elements of $\I$ as elements of $A$ via the structural map $\I \to A$. For an algebra $A$ and two lists of terms $a,b : I \to A$, we will write $A/a = b$ as the \emph{quotient algebra} identifying $a_i$ with $b_i$ for all $i : I$. We also write $A[I]$ for the free algebra over $A$ generated from $I$, or we also explicitly mention the generators $A[i_1,\cdots,i_n]$. We write the coproduct of $\I$-algebras as $A \otimes B$.

\section{Quasi-coherence and affine spaces}\label{sec:basics}

As mentioned, we will work with a type system enriched with a generic model $\I$ of some Horn theory. In this section we will first describe what is quasi-coherence, and briefly recap some of its elementary consequences. The results in this section applies to any Horn theory $\T$, thus at this state we do not assume $\T$ to satisfy any additional properties.

We view quasi-coherence as a \emph{property} of $\I$-algebras. As mentioned in \Cref{subsec:qc}, an $\I$-algebra is a $\T$-model $A$ equipped with a homomorphism $\I \to A$. Here by a $\T$-model we always mean a \emph{set} (in the sense of~\cite{hottbook}) equipped with a family of operations satisfying the axioms specified by $\T$. For any $\T$-algebras $A,B$, we will use $\I\alg(A,B)$ to denote the set of homomorphisms of $\I$-algebras from $A$ to $B$. For instance, we have mentioned the \emph{spectrum} $\spec A$ of $A$ is the set $\I\alg(A,\I)$.

Before discussing quasi-coherence, we observe that many sets can be naturally viewed as the spectrum of some $\I$-algebra. For the simplest example, since $\I$ is the initial $\I$-algebra, $\spec \I \cong \I\alg(\I,\I) \cong 1$. Similarly, $\I$ itself as a spectrum, $\spec \I[i] \cong \I\alg(\I[i],\I) \cong \I$. More generally, we have:

\begin{example}[Cubes]\label{exm:cubeaffine}
  For any $n:\N$, let $\I[n]$ be the free $\I$-algebra on $n$ generators. The cube $\I^n$ is a spectrum,
  \[ \spec \I[n] \cong \I\alg(\I[n],\I) \cong \I^n. \]
\end{example}

Evidently, the above construction not only applies to finite types, but arbitrary set $X$ where $\spec \I[X] \cong \I^X$. Notice $\I[X]$ is the $X$-indexed coproduct of $\I[i]$ for $\I$-algebras, and $\spec$ takes it to the $X$-indexed product of $\spec\I[i] \cong \I$. 

More generally, notice that $\spec$ is in fact a contravariant functor from $\I$-algebras to sets: For any $f : A \to B$, we have a map $\spec f : \spec B \to \spec A$ by precomposing with $f$. It indeed turns many colimits to limits, and as a consequence:

\begin{proposition}
  Every spectrum is a equaliser of powers of $\I$. 
\end{proposition}
\begin{proof}
  Let $A$ be \emph{any} $\I$-algebra. Notice that it can be written as a coequaliser of free $\I$-algebras, e.g. by taking $X$ as $A$ and $R$ as $\I[X] \times \I[X]$,
  \[
  \begin{tikzcd}
    \I[R] \ar[r, shift left, "s"] \ar[r, shift right, "t"'] & \I[X] \ar[r, two heads] & A
  \end{tikzcd}
  \]
  By the universal property, it follows that $\spec A$ can be identified as
  \[ \spec A \cong \scomp{x : \spec\I[X]}{xs = xt}, \]
  which implies it is the following equaliser,
  \[
  \begin{tikzcd}
    \spec A \ar[r, hook] & \spec\I[X] \cong \I^X \ar[r, shift left, "\spec s"] \ar[r, shift right, "\spec t"'] & \spec\I[R]\cong\I^R
  \end{tikzcd}
  \]
  This completes the proof.
\end{proof}

\begin{remark}[Spectrums are replete]\label{rem:specarereplete}
  Recall the notion of \emph{replete} type given in~\cite{hyland2006first}: $X$ is replete if it is orthogonal to any map $f$ that $\I$ is orthogonal to. Through out the paper, by orthogonality we always mean the \emph{internal} notion: $X$ is orthogonal to a map $f$ iff $X^f$ is an equivalence. Thus, an object is replete iff it belongs to the smallest internal localisation class containing $\I$. In particular, replete objects form an exponential ideal, which by the previous result it follows that all spectrums will be replete.
\end{remark}

Among all $\I$-algebras, some will be special:

\begin{definition}[Quasi-coherent algebra]
  An $\I$-algebra $A$ is \emph{quasi-coherent}, if the canonical evaluation map $A \to \I^{\spec A}$ is an equivalence of $\I$-algebras, where $\I^{\spec A}$ has the pointwise $\I$-algebra structure.
\end{definition}

We emphasis that the equivalence $A \cong \I^{\spec A}$ is not only an equivalence of types, but an equivalence of $\I$-algebras. This implies that the algebraic structure on $A$ can be viewed as pointwise induced by that on $\I$. 

% As a first consequence, the spectrum of a quasi-coherent algebra $A$ always \emph{has enough points}, in the sense that it distinguishes elements in $A$:

% \begin{lemma}\label{lem:completeness}
%   For a quasi-coherent $\I$-algebra $A$ and for $a,b : A$,
%   \[ a \le b \eq \fa x{\spec A} xa \le xb. \]
% \end{lemma}
% \begin{proof}
%   This follows from $A \cong \I^{\spec A}$ is an $\I$-algebra equivalence.
% \end{proof}

\begin{remark}[Quasi-coherent algebras are replete]\label{rem:qcreplete}
  If $A$ is quasi-coherent, then since $A \cong \I^{\spec A}$ it is also replete.
\end{remark}

\begin{example}\label{exm:intervalqc}
  $\I$ itself by definition is quasi-coherent. We have seen that $\spec \I \cong 1$. Under this equivalence, the canonical map 
  \[ \I \to \I^{\spec \I} \cong \I \]
  is exactly the identity on $\I$, hence is an equivalence.
\end{example}

Just like quasi-coherence is a property on $\I$-algebras, being affine will be a property on spectrums:

\begin{definition}[Affine]
  We say a type $X$ is \emph{affine}, if $\I^X$ is quasi-coherent, and the canonical evaluation map $X \to \spec \I^X$ is an equivalence.
\end{definition}

\begin{remark}[Dual algebra for affine space is unique]
  If $X$ is affine, it must be a set, and any $X \cong \spec A$ for a quasi-coherent $\I$-algebra $A$ will induce an equivalence $\I^X \cong A$. This justifies the above definition. In the future whenever we indicate $X$ is affine with $X \cong \spec A$, we will always assume $A$ is quasi-coherent.
\end{remark}

\begin{remark}[Affineness as a general property]
  Our notion of affine space already indicates an important difference between our approach and the ones taken in the related works~\cite{Cherubini_Coquand_Hutzler_2024,cherubini2024foundation}. There, the notion is directly connected to a notion of size, i.e. they are the spectrum of f.p.\ or c.p.\ algebras. Since we aim for a completely modular development, our notion of affine is more general, which a priori does not favour a class of size-related algebras. Many of the results in this section are essentially developed in~\cite{Cherubini_Coquand_Hutzler_2024}. However, we include them here for the precise reason that we need a version of them that applies to our slightly more general set up.
\end{remark}

For quasi-coherent algebras, the contravariant functor $\spec$ now induces a duality between quasi-coherent algebras and affine spaces:

\begin{proposition}\label{prop:duality}
  If $B$ is quasi-coherent, then for any $A$, the canonical map is an equivalence,
  \[ \I\alg(A,B) \cong \spec A^{\spec B}. \]
\end{proposition}
\begin{proof}
  The proof is essentially contained in~\cite[Prop. 2.2.1]{Cherubini_Coquand_Hutzler_2024}. We include the proof here for completeness since our assumption is slightly more general. By quasi-coherence of $B$, we have the following equivalences,
  \[ \I\alg(A,B) \cong \I\alg(A,\I^{\spec B}) \cong \I\alg(A,\I)^{\spec B} \cong \spec A^{\spec B}. \]
  The second step holds since $\I^{\spec B}$ has pointwise $\I$-algebra structure.
\end{proof}

The above duality result tells us that we can infer equivalence of algebras via equivalence of their spectrums:

\begin{corollary}\label{cor:dualityeqv}
  For any $f : \I\alg(A,B)$, if $A,B$ are quasi-coherent, then $f$ is an equivalence iff it induces an equivalence $\spec B \to \spec A$. 
\end{corollary}

At the end of this section, following~\cite{Cherubini_Coquand_Hutzler_2024} we describe an extremely useful formula computing coproducts of $\I$-algebras:

\begin{proposition}\label{prop:tensorasspace}
  For any $\I$-algebra $A,B$, 
  \[ \spec A \times \spec B \cong \spec(A \otimes B). \]
  If $B$ is quasi-coherent, then there is an equivalence
  \[ \I^{\spec A \otimes B} \cong B^{\spec A}. \]
  Of course, this computes $A \otimes B$ when $A \otimes B$ is itself quasi-coherent.
\end{proposition}
\begin{proof}
  This is essentially~\cite[Lem. 3.2.1 \& Lem. 3.3.1]{Cherubini_Coquand_Hutzler_2024}.
\end{proof}

Under the above identification, if $B$ is quasi-coherent, then the evaluation map $A \otimes B \to \I^{\spec A \otimes B}$ can be equivalently viewed as
\[ c \mapsto \ld{x}{\spec A} (x\otimes B)(c) : A \otimes B \to B^{\spec A}. \]

% \begin{proposition}\label{prop:pullbackofaffine}
%   Let $A,B,C$ be $\I$-algebras. Given $f : \I\alg(B,A)$ and $g : \I\alg(B,C)$, the pullback of $\spec f$ and $\spec g$ is given by the spectrum
%   \[ \spec A \times_{\spec B} \spec C \cong \spec (A \otimes_B C), \]
%   where $A \otimes_B C$ is the following pushout of $\I$-algebras,
%   \[
%     \begin{tikzcd}
%       B \ar[r, "f"] \ar[d, "g"'] & A \ar[d] \\
%       C \ar[r] & A \otimes_B C
%       \arrow["\lrcorner"{anchor=center, pos=0.125, rotate=180}, draw=none, from=2-2, to=1-1]
%     \end{tikzcd}
%   \]
% \end{proposition}
% \begin{proof}
%   The proof is essentially contained in~\cite[Lem. 3.2.2]{Cherubini_Coquand_Hutzler_2024}. The only subtlety to note is that, the use of quasi-coherence can be eliminated by assuming $\spec f,\spec g$ are induced by maps between $\I$-algebras. 
% \end{proof}

% \begin{remark}
%   Of course, when $A,C$ are quasi-coherent, by \Cref{prop:duality} any map from $\spec A$ to $\spec B$ is induced by an $\I$-algebra morphism, and similarly for $\spec C$. In this case, any such pullback can be computed as a spectrum.
% \end{remark}




\section{Open propositions and dominance}\label{sec:dominance}

Notice that up till this point we have not used any special property of $\T$ rather than the fact that it is a Horn theory. However, to move closer to the intended application in domain theory, we start by assuming our theory $\T$ is \emph{propositional}: 

\begin{definition}[Propositional theory]\label{defn:propositional}
  We say a Horn theory $\T$ is \emph{propositional}, if it extends the theory of meet-semi-lattices (1, $\wedge$), and truth of an element is computed by slicing: For any $\T$-model $A$ and $a:A$, the quotient $A/a=1$ is given by
  \[ a \wedge - : A \surj A/a, \]
  where $A/a$ by definition is $\cv a := \scomp{b:A}{b\le a}$, where $\le$ is the partial order on $A$ induced by the meet-semi-lattice structure.
\end{definition}

\begin{remark}[Examples of propositional theories]
  The theory $\mbb M$ of meet-semi-lattices, $\mbb D$ of distributive lattices, $\mbb H$ of Heyting algebras, and $\mbb B$ of Boolean algebras are all examples of propositional theories in the above sense. In fact, all finitary quotients of $\mbb H$ or $\mbb B$ will again be propositional. More generally, for any propositional theory $\T$ and any $\T$-model $D$, the theory of $D$-algebras will again be propositional, because quotients of $D$-algebras are computed the same as quotients of models of $\T$.
\end{remark}

\begin{remark}[The theory of $\sigma$-frames]\label{rem:sigmaframe}
  We also consider theory of $\sigma$-frames. A $\sigma$-frame $A$ is a lattice with finite meets and countable joins, such that they satisfy the distributivity axiom 
  \[ a \wedge \bigvee_{n:\N} b_n = \bigvee_{n:\N} a \wedge b_n, \]
  for any $a$ in $A$ and $b : \N \to A$. We can axiomatise $\sigma$-frames as an \emph{infinitary} algebraic theory $\mbb S$, extending the theory of meet-semi-lattices with a constant $0$ and a function symbol $\bigvee$ with \emph{countable} arity. Such an axiomatisation can be found e.g. in~\cite[Exa. 3.26]{adamek1994locally}. It is easy to see $\mbb S$ is propositional. 

  A notable fact about $\sigma$-frames is that, their finitary behaviours are exactly the same as distributive lattices, in the sense that for any $\sigma$-frame $A$, a f.p.\ $\sigma$-frame $A[n]/R$ generated from $A$ coincides with the f.p.\ \emph{distributive lattice} $A[n]/R$. Thus, there will be no difference between working with $\sigma$-frames or distributive lattice when we work with f.p.\ $\I$-algebras. The distinguished feature of $\sigma$-frames allowing countably joins will only shine in \Cref{sec:infdomain} when we prove completeness of $\I$.
\end{remark}

For a propositional theory $\T$, we think of the generic model $\I$ as certain interval object, since it is equipped with a partial order. In this case, more types can be realised as spectrums. The important examples are \emph{simplices}:

\begin{example}\label{exm:simplicesaffine}
  For any $n : \N$, let $\Delta^n \hook \I^n$ be the following subset,
  \[ \Delta^n := \scomp{i : n \to \I}{i_1 \ge \cdots \ge i_n}. \]
  This type is indeed a spectrum, since by definition we have
  \[ \Delta^n \cong \spec\I[i_1,\cdots,i_n]/i_1\ge\cdots\ge i_n. \]
  To simplify later discussion, we might also introduce the following types isomorphic to simplices above,
  \[ \Delta_n := \scomp{i : n \to \I}{i_1 \le \cdots \le i_n}. \]
\end{example}

The constant $1 : \I$, which is the top element in $\I$, induces a predicate
\[ \ms t : \I \to \pp, \]
which takes $i : \I$ to the proposition $i = 1$. The first observation is that $\ms t$ takes $i : \I$ to a spectrum:

\begin{lemma}\label{lem:openpropaffine}
  For any $i : \I$, $\spec\I/i$ is a proposition, and $\ms ti \eq \spec \I/i$.
\end{lemma}
\begin{proof}
  By definition, $\spec \I/i \cong \I\alg(\I/i,\I)$. Since $\I$ is the initial $\I$-algebra and $\I/i$ is a quotient of $\I$, any homomorphism $\I/i \to \I$ will be unique, hence $\spec\I/i$ is a proposition. It thus suffices to show $\spec\I/i \to \ms ti$. By $\T$ being propositional, an $\I$-algebra morphism $f : \I/i \to \I$ is a commutative diagram as follows,
  \[
  \begin{tikzcd}
    \I/i \ar[rr, dashed, "f"] & & \I \\ 
    & \I \ar[ul, two heads, "i \wedge -"] \ar[ur, equal]
  \end{tikzcd}
  \]
  In this case, we have
  \[ i = f(i \wedge i) = f(i) = 1. \]
  The final identity holds since $f$ preserves the top element. 
\end{proof}

The goal of this section is to show that, under a suitable quasi-coherence assumption, $\ms t : \I \to \pp$ makes $\I$ a \emph{dominance} in the sense of~\cite{rosolini1986continuity}. It turns out that it suffices to require all principle quotients of $\I$ to be quasi-coherent.

\begin{definition}[Stable quasi-coherence and stably affine]
  An $\I$-algebra $A$ is \emph{stably quasi-coherent} if all finitely generated principle congruences of $A$ are also quasi-coherent. An affine $X \cong \spec A$ is \emph{stably affine}, if $A$ is stably quasi-coherent.
\end{definition}

In other words, if $A$ is stably quasi-coherent, then for any $a,b:A$, the quotient $A/a=b$ is also stably quasi-coherent. It turns out the following minimal assumption suffices:

\begin{axiom}[SQCI]\label{ax:sqci}
  $\I$ is stably quasi-coherent.
\end{axiom}

As a first consequence, we show the following important lemma:

\begin{lemma}[SQCI]\label{lem:intconserve}
  The interval $\I$ is \emph{conservative}, that for $i,j : \I$,
  \[ (\ms ti \eq \ms tj) \eq i = j. \]
\end{lemma}
\begin{proof}
  If $i \le j$ then trivially $\ms ti \to \ms tj$. Thus it suffices to show
  \[ (\ms ti \to \ms tj) \to i \le j. \]
  Take $i,j$ with $\ms ti \to \ms tj$. By \Cref{lem:openpropaffine} and (SQCI), each $\ms ti$ and $\ms tj$ will be affine. Then we get a restriction map between $\I$-algebras,
  \[ \I/j \cong \I^{\ms tj} \to \I^{\ms ti} \cong \I/i. \]
  Similar to the proof of \Cref{lem:openpropaffine}, the existence of such an $\I$-algebra homomorphism implies $i \le j$. 
\end{proof}

Thus, under (SQCI), we may view the generic algebra $\I$ as a subuniverse of \emph{open} propositions via the embedding $\ms t$:

\begin{definition}[Open proposition]
  Given a proposition $p$, we say it is \emph{open} if $p$ is of the form $\ms ti$ for some $i$ in $\I$,
  \[ \ms{isopen}(p) := \sum_{i:\I}p \eq \ms ti. \]
\end{definition}

Under (SQCI), by \Cref{lem:intconserve} the $i$ that $p \eq \ms ti$ will be \emph{unique}, thus being open is itself a proposition. Also, open propositions are evidently closed under finite conjunctions, since $\ms t$ preserves them. 

More generally, we may define the notion of open subtypes:

\begin{definition}[Open subtype]
  A subtype $U$ of $X$ is \emph{open}, if for any $x:X$ the proposition $U(x)$ is open,
  \[ \ms{isOpen}(U) := \fa xX \ms{isopen}(U(x)). \]
\end{definition}

If $X$ is affine, open subtypes are easy to classify:

\begin{lemma}[SQCI]\label{lem:openofaffinegivesalgebra}
  Let $X \cong \spec A$ be affine. Then any open subset of $X$ is of the form $D_a$ for some $a:A$, where
  \[ D_a := \scomp{x : X}{\ms t(xa)} \cong \spec A/a. \]
\end{lemma}
\begin{proof}
  By definition, if $U$ is an open subtype of $X$, then 
  \[ \fa xX\sum_{i:\I}U(x) \eq \ms ti. \] 
  This way, we get a map $\qsi a : X \to \I$ with $U(x) \eq \ms t(\qsi ax)$. By $X$ being affine, $\qsi a$ can be identified with an element $a:A$, where $\qsi ax = xa$. This way,
  \[ \fa xX U(x) \eq \ms t(xa), \]
  thus $U$ is the subtype $D_a$.
\end{proof}

\begin{proposition}[SQCI]\label{prop:Idominance}
  The type of opens $\I$ forms a dominance.
\end{proposition}
\begin{proof}
  Since $\ms t : \I \hook \pp$ by definition is closed under finite meets, it suffices to show that an open subtype of an open proposition is again an open proposition. Suppose $p$ is an open proposition and $q$ is an open subtype of $p$. By definition $p \eq \ms ti$ for some $i:\I$. Since $\ms ti$ is affine, \Cref{lem:openofaffinegivesalgebra} implies that $q$ is $D_j$ for some $j : \I/i$. Since $\T$ is propositional, equivalently $j$ can be viewed as an element $j : \I$ with $j \le i$. This way, 
  \[ q \eq D_j \eq \spec(\I/i)/j \eq \spec\I/j, \]
  which implies $q$ is also an open proposition.
\end{proof}

\begin{remark}[Dominance without choice]\label{rem:dominancewithoutchoice}
  The related work \cite{Cherubini_Coquand_Hutzler_2024} contains a similar construction of a dominance from a generic \emph{ring}. However, the proof there relies on certain classical axiom, which they call \emph{Zariski local choice}. The reason that the construction here does not require additional assumption is precisely due to conservativity of $\I$: The dominance is $\I$ itself, rather than an image of $\I$. This way, it allows us to transform an open subtype of $X$ to a map $X \to \I$, which by quasi-coherence can be further identified as an element of the algebra if $X$ is affine. 
\end{remark}


\section{Lifting}\label{sec:lifting}

Given then dominance structure $\I$, we can construct internally the lifting functor. For any type $X$, its lift is given by
\[ X\prt := \sum_{i:\I}X^{\ms ti}. \]
The functoriality is easy to express: For any $f : X \to Y$, we have
\[ f\prt(i,x) := (i,\ld{w}{\ms ti}fxw). \]
There is an evident unit $\eta\prt : X \to X\prt$, where
\[ \eta := \ld x X(1,\ld\hp 1 x). \]
The dominance structure on $\I$ also gives a multiplication $\mu : (X\prt)\prt \to X\prt$, where $\mu$ takes any $(i,u)$ with $i : \I$ and $u : \ms ti \to X\prt$ first to $(j,x)$, where $j$ is the dependent sum
\[ j := \sum_{w:\ms ti} (uw)_0, \]
and $x : \ms tj \to X$ is the partial element such that for $w : \ms ti$ and $v : (uw)_0$
\[ x(w,v) := (uw)_1(v). \]

\begin{example}
  By definition, it is easy to see that
  \[ 1\prt \cong \sum_{i:\I}\ms ti \cong \I. \]
\end{example}

For synthetic domain theory, the object of particular importance is the lift of the interval $\I$ itself. The following computation works more generally for all stably quasi-coherent algebras:

\begin{proposition}\label{prop:liftingofalgebra}
  If $A$ is stably quasi-coherent, then we have
  \[ A\prt \cong \sum_{i:\I}A/i. \]
\end{proposition}
\begin{proof}
  Note for any $i : \I$, the quotient $A/i$ is again quasi-coherent. Now notice that we do have
  \[ A/i \cong A \otimes \I/i. \]
  By \Cref{prop:tensorasspace} and quasi-coherence of $A/i$, 
  \[ A/i \cong A \otimes \I/i \cong A^{\spec \I/i} \cong A^{\ms ti}. \]
  This way, it follows that 
  \[ A\prt \cong \sum_{i:\I}A^{\ms ti} \cong \sum_{i:\I}A/i. \qedhere \]
\end{proof}

In other words, an element $(i,a) : A\prt$ is simply a pair $i : \I$ and $a : A$ that $a \le i$. This way, we can easily compute $\I\prt$:

\begin{corollary}[SQCI]
  $\I\prt \cong \Delta^2$.
\end{corollary}
\begin{proof}
  By \Cref{prop:liftingofalgebra} and the assumption (SQCI),
  \[ \I\prt \cong \sum_{i:\I}\I/i \cong \scomp{i,j : \I}{i \ge j} \cong \Delta^2. \]
  The second equivalence again uses the fact that $\T$ is propositional.
\end{proof}

\begin{remark}[Algebra v.s. geometry, I]
  One interesting fact to note here is that, though $\I\prt$ by computation is a dependent sum of algebras, it is naturally equivalent to a \emph{spectrum}. In some sense the source is the assumption that $\T$ is propositional, which allows us to identify the algebraic object $\I/i$ as a subset $\scomp{j : \I}{j \le i}$. There will be more examples of such nature once we work more specifically with distributive lattices; cf. \Cref{prop:simplicesasalgebra}.
\end{remark}

More generally, for domain theoretic applications we would want to compute the lifting of $\Delta^2$, or $\Delta^n$ for any $n:\N$. For this purpose, we observe that we can more generally compute the lifting of any spectrum (not only affine spaces):

\begin{proposition}[SQCI]\label{prop:liftofaffine}
  For an $\I$-algebra $A$, the lifting of $\spec A$ is
  \[ (\spec A)\prt \cong \sum_{i:\I}\I\alg(A,\I/i). \]
\end{proposition}
\begin{proof}
  By (SQCI) and \Cref{lem:openpropaffine}, $\ms ti$ is affine for any $i:\I$. By the duality result in \Cref{prop:duality}, we have
  \[ (\spec A)\prt \cong \sum_{i:\I}\spec A^{\ms ti}\cong \sum_{i:\I}\I\alg(A,\I/i). \qedhere \]
\end{proof}

% Thus, motivated by domain theory, we find ourselves naturally move towards the following axiomatisation:

% \begin{axiom}[SQCF]
%   All finitely generated free $\I$-algebras, i.e. $\I[n]$ for $n : \N$, are stably quasi-coherent.
% \end{axiom}

% Equivalently, (SQCF) says that any finitely presented $\I$-algebra is (stably) quasi-coherent, where an $\I$-algebra is finitely presented if it is merely of the form $\I[n]/s=t$ with $s,t : m \to \I[n]$ for some finite $n,m$.

% Of course, (SQCF) implies (SQCI) when taking $n$ to be 0. Furthermore, (SQCF) implies that the simplices $\Delta^n$ are now (stably) affine as well. This way, we can indeed compute their lifts:

\begin{corollary}[SQCI]
  For any $n : \N$, we have
  \[ \Delta^n\prt \cong \Delta^{n+1}, \]
  and the unit $\eta : \Delta^n \hook \Delta^n\prt \cong \Delta^{n+1}$ takes $i_1 \ge \cdots \ge i_n$ in $\Delta^n$ to $1 \ge i_1 \ge \cdots \ge i_n$ in $\Delta^{n+1}$. 
\end{corollary}
\begin{proof}
  By \Cref{prop:liftofaffine}, for the spectrum $\Delta^n \cong \spec\I[n]/i_1 \ge \cdots \ge i_n$,
  \begin{align*}
    \Delta^n\prt 
    &\cong \sum_{i:\I}\I\alg(\I[n]/i_1\ge\cdots\ge i_n,\I/i) \\
    &\cong \scomp{i_1,\cdots,i_n:\I/i}{i_1 \ge \cdots \ge i_n} \\
    &\cong \scomp{i,i_1,\cdots,i_n:\I}{i \ge i_1 \ge \cdots \ge i_n} \\
    &\cong \Delta^{n+1}
  \end{align*}
  Again the third steps holds since $\T$ is propositional.
\end{proof}

\section{Distributive lattice and locality}\label{sec:locality}

One important finitary axiom for synthetic domain theory is the \emph{Phoa's principle}, which as an observation of~\cite{gratzer2024directed} is a consequence of the quasi-coherence principle for distributive lattices. Thus, we now work with the theory $\mbb D$ of distributive lattices, or more generally the theory of $D$-algebra for some distributive lattice $D$. As indicated in \Cref{rem:sigmaframe}, computation will not change if we replace distributive lattices by $\sigma$-frames. Hence, the results here and in \Cref{sec:intposet} will be applicable to $\sigma$-frames as well, where only the properties of f.p.\ $\I$-algebras matter.

In this case, $\I$ has a minimal element $0$. Notice that the constant $0$ also induces a predicate on $\I$,
\[ \ms f : \I \to \pp, \]
which takes any $i : \I$ to $i = 0$. The theory of distributive lattices, or the theory of $D$-algebras for any distributive lattice $D$, also satisfies the \emph{dual} property of being propositional: It extends the theory of join-semi-lattices ($0$, $\vee$), and the principle congruence $A/a = 0$ for any $a:A$ is computed by \emph{coslicing} $a \vee - : A \to a/A$ with $a/A := \scomp{b:A}{a \le b}$; similarly for $\sigma$-frames. Hence, dual versions of the previous results will also hold by symmetry, when we exchange $1$ for $0$ and $\wedge$ for $\vee$. For instance, the same proof as in \Cref{lem:intconserve} also implies the following:

\begin{corollary}[SQCI]
  $\ms f : \I \to \pp$ is an embedding.
\end{corollary}

We will now call propositions in the image of $\ms f$ \emph{closed}, and a subtype classified by $\ms f$ a \emph{closed subtype}. Completely similar to the previous \Cref{lem:openofaffinegivesalgebra}, closed subtypes of affine spaces are again of the following form:

\begin{corollary}[SQCI]\label{cor:closedaffinealgebra}
  If $X \cong \spec A$ is affine, then closed subtypes of $X$ are all of the form for some $a:A$,
  \[ C_a := \scomp{x : X}{\ms f(xa)} \cong \spec a/A. \]
\end{corollary}

\begin{corollary}[SQCI]
  $\ms f$ forms a dominance.
\end{corollary}

There is an accompanying \emph{colifting} monad $(-)\cprt$, defined by
\[ X\cprt := \sum_{i:\I} X^{\ms fi}. \]
And completely similar as \Cref{prop:liftingofalgebra}, under (SQCI) we can explicitly compute the colifting of the simplices $\Delta^n$,
\[ \Delta^n\cprt \cong \Delta^{n+1}, \]
where now the unit $\eta : \Delta^n \hook \Delta^n\cprt \cong \Delta^{n+1}$ takes a sequence $i_1 \ge \cdots \ge i_n$ to $i_1 \ge \cdots \ge i_n \ge 0$ in $\Delta^{n+1}$. To distinguish the two inclusions $\Delta^n \hook \Delta^n\prt \cong \Delta^{n+1}$ and $\Delta^n \hook \Delta^{n}\cprt \cong \Delta^{n+1}$, we will denote the former as $\eta\prt$ and the latter as $\eta\cprt$.

The remaining part of this section will introduce various locality axioms for the interval $\I$, and discuss some of their consequences with the quasi-coherence axiom.

As a start, a minimal requirement for synthetic domain theory to model divergent computation is for the dominance $\I$ to be closed under falsum. For this, we introduce the following minimal amount of non-triviality:

\begin{axiom}[NT]\label{ax:nt}
  For $\I$, $0 \neq 1$.
\end{axiom}

% Let us first observe some elementary consequence of (NT). For instance, the lifting monad $(-)\prt$ defined in \Cref{sec:lifting} is now also \emph{pointed}. For any $X$ we can define
% \[ \bot := (0,?) : X\prt, \]
% where $? : \ms t0 \cong \emp \to X$ is the unique map from $\emp$.

This way, $\emp \eq \ms t0$ now will be affine, and it is both an open and closed proposition. As mentioned in introduction, (NT) together with a strong enough quasi-coherence principle will imply all the axioms for synthetic domain theory. Before we discuss that in \Cref{sec:infdomain}, a minimal amount of quasi-coherence provided by (SQCI) already implies a lot of elementary properties. As a first consequence of (NT) with (SQCI), we can show $\I$ is almost the Boolean algebra $2$ in the following sense:

\begin{proposition}[NT, SQCI]\label{prop:filed}
  For any $i : \I$ 
  \[ \neg \ms ti \eq \ms fi, \quad \neg\ms fi \eq \ms ti. \]
  In particular, the embedding $2 \hook \I$ induced by $0,1$ is $\neg\neg$-dense,
  \[ \fa i\I \dneg(\ms ti \vee \ms fi). \]
\end{proposition}
\begin{proof}
  If $\ms fi$ then $\neg\ms ti$ by (NT). On the other hand, by the conservativity result in \Cref{lem:intconserve} and (NT), if $i \neq 1$ then $i = 0$ since $0 \neq 1$. The dual case also follows by symmetry. Now given $i :\I$, if $\neg(\ms ti \vee \ms fi)$, which implies $\neg\ms ti \vee \neg\ms fi$, equivalently $\ms fi \wedge \ms ti$, contradictory to (NT). Hence, $\neg\neg(\ms ti \wedge \ms fi)$.
\end{proof}

This allows us to observe that the open and closed propositions have a bijective correspondence between them:

\begin{corollary}[NT, SQCI]\label{cor:opendnegclose}
  For any proposition $p$, $p$ is open iff $\neg p$ is closed and vice versa. Furthermore, open and closed propositions are $\dneg$-stable.
\end{corollary}

However, under the non-triviality axiom (NT), quasi-coherence implies the interval $\I$ will \emph{never} be isomorphic to the discrete set $2$. One way to see this is to show $\I$ is always \emph{internally connected}. For this purpose, we recall the \emph{specialisation order} as well studied in synthetic domain theory~\cite{PhoaWesleyKym-Son1991DtiR}:

\begin{definition}[Specialisation order]\label{defn:specialisation}
  The \emph{specialisation order} on a type $X$ is defined as follows:
  \[ x \preceq y := \fa{U}{X\to\I} U(x) \le U(y). \]
\end{definition}

By definition, the specialisation order is reflexive and transitive. As already observed in \emph{loc. cit.}, one important property of the specialisation order is that \emph{every} map is monotone w.r.t. this order:

\begin{lemma}\label{lem:anymapmonotoneintriscorder}
  For $f : X \to Y$, $x \preceq y$ in $X$ implies $fx \preceq fy$ in $Y$.
\end{lemma}
\begin{proof}
  This simply follows from compositionality of functions.
\end{proof}

We can very easily classify the specialisation order on discrete sets, i.e. sets with decidable equality:

\begin{lemma}[NT]\label{lem:discretephoa}
  If $M$ is a discrete set, then the specialisation order on $M$ is discrete, in the sense that for $n,m : M$,
  \[ n \preceq m \to n = m. \]
\end{lemma}
\begin{proof}
  Suppose $n \preceq m$. If $n \neq m$, we can construct a map $f : M \to \I$ with 
  \[ f(x) =
  \begin{cases}
    1 & x = n \\ 
    0 & \other
  \end{cases}
  \]
  This is well-defined since $M$ has decidable equality. In particular, $f(n) = 1$ and $f(m) = 0$, contradictory to $n \preceq m$. Hence, $\neg n \neq m$, or $n = m$.
\end{proof}

\begin{corollary}[NT]\label{cor:connectedpreservediscrete}
  If the specialisation order on $X$ has a minimal or top element $x$, then $X$ is internally connected, in the sense that for any discrete set $M$, the constant map $M \to M^X$ is an equivalence.
\end{corollary}
\begin{proof}
  Since $x:X$ is inhabited, $M \to M^X$ is always an embedding. Thus it suffices to show for any $f : X \to M$, $f$ is a constant map. Since $x$ is minimal in $X$, for any $y:X$ by \Cref{lem:anymapmonotoneintriscorder} we have $f(x) \preceq f(y)$, which by \Cref{lem:discretephoa} implies $f(x) = f(y)$. Hence, $M \to M^X$ is also surjective. This shows it is an equivalence since it is a map between sets.
\end{proof}

\begin{remark}
  The above proof would also work with the weaker assumption that the specialisation order on $X$ is \emph{connected}. This generalises a similar result given in~\cite[Prop. 4.4.1]{hyland2006first}.
\end{remark}

Thus, it suffices to compute the specialisation order on $\I$. In fact, the specialisation order on any affine space is easy to characterise:

\begin{lemma}\label{lem:specorderofaffine}
  For any affine space $X \cong \spec A \cong \I\alg(A,\I)$, the specialisation order $\preceq$ on $X$ is induced by the pointwise order on $\I$,
  \[ x \preceq y \eq \fa aA xa \le ya. \]
\end{lemma}
\begin{proof}
  This holds by definition of $\preceq$, and the fact that $\I^X \cong A$.
\end{proof}

To apply the above result to $\I$, we would like to view $\I$ itself as an affine space. Since we know $\I \cong \spec\I[i]$, we would like $\I[i]$ to be a quasi-coherent $\I$-algebra. This goes beyond the assumption (SQCI), and we introduce the following stronger axiom:

\begin{axiom}[SQCF]
  For any $n : \N$, the finitely generated free $\I$-algebra $\I[n]$ is stably quasi-coherent. 
\end{axiom}

The above axiom is equivalent to saying that all f.p.\ $\I$-algebras are quasi-coherent. Under (SQCF), the cubes $\I^n$ and the simplices $\Delta^n$ will all be affine as shown by \Cref{exm:cubeaffine,exm:simplicesaffine}.

\begin{corollary}[SQCF]\label{cor:pointwisecubeorder}
  The specialisation order on $\I^n$ is the pointwise order induced by $\I$ for any $n : \N$.
\end{corollary}
\begin{proof}
  By assumption $\I^n \cong \spec\I[n]$ is affine. By viewing $p : \I[n]$ as a polynomial function on $\I^n$, we have
  \[ \fa{i,j}{\I^n} i \preceq j \eq \fa p{\I[n]} p(i) \le p(j). \]
  Since projections are in particular polynomials, we indeed have $i_m \le j_m$ for all $m : n$. On the other hand, since polynomial functions for distributive lattices are monotone, if $i \le j$ then indeed we have $p(i) \le p(j)$. 
\end{proof}

\begin{corollary}[NT, SQCF]\label{cor:intisnot2}
  $\I$ is not the discrete set $2$.
\end{corollary}
\begin{proof}
  By \Cref{cor:pointwisecubeorder}, the specialisation order on $\I$ coincides with its canonical order. Hence, it has a top element, thus by \Cref{cor:connectedpreservediscrete}, $2^\I \cong 2$. If $\I \cong 2$, then $2^\I \cong 2^2 \cong 4$, contradictory. Hence, $\I$ is not 2.
\end{proof}

We will continue the computation of specialisation order and investigating their consequences in \Cref{sec:intposet}. At this moment, we emphasis that all the previously established results are purely the consequences of the quasi-coherence assumption (SQCF), and the non-triviality axiom (NT). This showcases the strength and richness of the quasi-coherence axiom.

The remaining part of this section discusses further locality axioms on the interval $\I$, which would correspond to stronger properties of the dominance $\I$ and can be applied to different flavours of domain theory. 

A slightly stronger axiom which is also common in the practice of domain theory is that the dominance $\I$ should furthermore be closed under all finite disjunctions. For this, we consider the following axiom:

\begin{axiom}[L]\label{ax:L}
  $\I$ is local, i.e. $0 \neq 1$, and $\ms t(i\vee j) \eq \ms ti \vee \ms tj$ for $i,j : \I$. Geometrically, (L) states that the following two inclusions are jointly surjective.
  \[ 
  \begin{tikzcd}
    \I \ar[r, hook, "{i \mapsto (i,1)}"] & \scomp{i,j : \I}{\ms t(i\vee j)} & \I \ar[l, hook', "{j \mapsto (1,j)}"']
  \end{tikzcd}
  \]
\end{axiom}

Under (L), the dominance $\I$ will be closed under finite disjunctions. In this case, we can more generally characterise the space $2^X$ for \emph{any} affine space $X$, not just the ones having a minimal or maximal element according to \Cref{cor:connectedpreservediscrete}. For this we first observe a strengthening of \Cref{cor:intisnot2}:

\begin{lemma}[L, SQCI]\label{lem:intisnotBoolean}
  $\I$ has no non-trivial complemented elements, i.e. if $i : \I$ is complemented, then $\ms ti \vee \ms fi$. In particular, $\I$ cannot be Boolean.
\end{lemma}
\begin{proof}
  Suppose $i$ has a complement $j$. Then by (L) we have
  \[ 1 \eq \ms t(i \vee j) \eq \ms ti \vee \ms tj, \]
  and similarly,
  \[ \emp \eq \ms t(i \wedge j) \eq \ms ti \wedge \ms tj. \]
  It follows that $\ms tj \eq \neg \ms ti$, and by \Cref{prop:filed}, $\ms tj \eq \ms fi$. This way, $\ms ti \vee \ms fi$. If $\I$ is Boolean, it implies $\I \cong 2$, contradictory to \Cref{cor:intisnot2}.
\end{proof}

\begin{proposition}[L, SQCI]\label{prop:complementinalgebra}
  If $X \cong \spec A$ is affine, then $2^X$ is equivalent to the set of complemented elements in $A$.
\end{proposition}
\begin{proof}
  The crucial observation is that in this case, $2$ is itself a \emph{spectrum}. Consider the algebra $B := \I[i,j]/\pair{i\wedge j =0,i\vee j = 1}$. By construction, it classifies complemented objects. Hence, an element in $\spec B$ is a complemented element in $\I$, which implies $\spec B \cong 2$ by \Cref{lem:intisnotBoolean}. This way, by \Cref{prop:duality}, $2^X \cong \I\alg(B,A)$, which is precisely the set of complemented elements in $A$.
\end{proof}

As another example of a new spectrum (L) allows us to define:

\begin{example}\label{exm:hornaffine}
  Consider the right outer horn $\Lambda^2_2$ as the following pushout,
  \[
    \begin{tikzcd}
      1 \ar[d, "1"'] \ar[r, "1"] & \I \ar[d] \\
      \I \ar[r] & \Lambda^2_2
      \arrow["\lrcorner"{anchor=center, pos=0.125, rotate=180}, draw=none, from=2-2, to=1-1]
    \end{tikzcd}
  \]
  By viewing $\Lambda^2_2$ as a subspace of $\I^2$, we may identify it as follows,
  \[ \Lambda^2_2 \cong \scomp{i,j : \I}{\ms ti \vee \ms tj}. \]
  Assuming (L), $\ms ti \vee \ms tj \eq \ms t(i\vee j)$, it follows that 
  \[ \Lambda^2_2 \cong \spec \I[i,j]/i \vee j. \]
\end{example}

Axiom (L) clearly has a dual counterpart:

\begin{axiom}[cL]\label{ax:cL}
  $\I$ is colocal, i.e. $0 \neq 1$ and $\ms f(i \wedge j) \eq \ms fi \vee \ms fj$ for all $i,j : \I$.
\end{axiom}

Similarly to \Cref{exm:hornaffine}, this allows one to identify the left outer horn $\Lambda^2_0$ as a spectrum. Of course we can combine the axioms (L) and (cL). 

An axiom even stronger than both (L) and (cL) is the simplicial axiom:

\begin{axiom}[SL]\label{ax:SL}
  $\I$ is \emph{linear}, i.e. $0 \neq 1$ and for $i,j : \I$, $i \le j \vee j \le i$. Geometrically, the simplicial axiom states that the two simplices $\Delta^2,\Delta_2$ covers the square $\Delta^2 \cup \Delta_2 \cong \I^2$.
\end{axiom}

The strongest locality axiom states that the simplicial structure is truncated to level 1:

\begin{axiom}[1T]\label{ax:1T}
  $\I$ is \emph{1-truncated}, i.e. (SL) holds and for $i,j : \I$, 
  \[ i \le j \eq \ms fi \vee i = j \vee \ms tj. \]
  Geometrically, (1T) additionally requires the canonical inclusion $\partial\Delta_2 \hook \Delta_2$ from the boundary $\partial\Delta_2$ to the 2-simplice $\Delta_2$ is an equivalence, where $\partial\Delta_2$ type-theoretically is simply
  \[ \partial\Delta_2 := \scomp{i,j : \I}{\ms fi \vee i = j \vee \ms ti}. \]
\end{axiom}

In general we will at most work with the weakest locality axiom (NT). However, we will show in \Cref{sec:local} that all affine spaces will be \emph{orthogonal} to the maps that define these local principles as shown above.


\section{Phoa's principle}\label{sec:intposet}

As mentioned in the introduction, the Phoa's principle states that the order on $\I$ is classified by the function space $\I^\I$. The goal of this section is to show that this statement holds not only for $\I$, but also more generally for all affine spaces and quasi-coherent algebras.

Before we proceed, we need to be precise what we are going to show, since there are two competing notions of order on an affine space or an $\I$-algebra. On one hand we have introduced a specialisation order in \Cref{defn:specialisation}. On the other hand, both affine spaces and $\I$-algebras have a canonical order on them: The former is induced pointwise by the order on $\I$, and the latter is induced from its distributive lattice structure. 

From \Cref{cor:pointwisecubeorder} we have seen that for cubes $\I^n$, there is no ambiguity between the two notions. The following result will show that in practice, this more generally holds for all affine spaces.

\begin{lemma}\label{lem:orderonaffine}
  Let $A,B$ be quasi-coherent and $A \surj B$ a surjection. The induced map $\spec B \hook \spec A$ is an order embedding for $\preceq$, i.e. for $x,y : \spec B$, $x \preceq y$ in $\spec B$ iff $x \preceq y$ in $\spec A$.
\end{lemma}
\begin{proof}
  By monotonicity in \Cref{lem:anymapmonotoneintriscorder}, it suffices to show for $x,y : \spec B$, if $x \preceq y$ in $\spec A$, then $x \preceq y$ in $\spec B$. However, this follows from the fact that $\I$ is weakly orthogonal to the inclusion $\spec B \hook \spec A$,
  \[
  \begin{tikzcd}
    \spec B \ar[d, hook] \ar[r, "b"] & \I \\ 
    \spec A \ar[ur, dashed, "\exists a"'] 
  \end{tikzcd}
  \]
  because by quasi-coherence $\I^{\spec A} \cong A \surj B \cong \I^{\spec B}$ is a surjection. 
\end{proof}

For instance, under (SQCF), the above result applies to the quotient 
\[ \I[n] \surj \I[n]/R \]
for any finitely generated equivalence relation $R$. In particular, the specialisation order on simplices $\Delta^n$ coincides with its pointwise order as a subspace of $\I^n$.

This way, we can generalise the Phoa's principle to the following definition that applies to an arbitrary type:

\begin{definition}[Phoa's principle]\label{def:phoa}
  A type $X$ satisfies \emph{Phoa's principle}, if 
  \[ \pair{\ev_0,\ev_1} : X^\I \to \operatorname{\preceq} \]
  is an equivalence over $X \times X$.
\end{definition}

% \begin{remark}[Linked spaces]
%   In~\cite{PhoaWesleyKym-Son1991DtiR}, a type $X$ is called \emph{linked} if $X^\I \to \operatorname{\preceq}$ is a surjection, and $X$ is called a \emph{space} if $\preceq$ is antisymmetric. The \emph{loc. cit.} establishes a large collection of properties for linked spaces to study general domains. From a current perspective, we believe our definition above is the better behaved notion.

In other words, a type $X$ satisfies Phoa's principle iff its specialisation order is classified by maps from the interval $\I$.

\begin{example}
  For instance, under (NT) and (SQCF), \Cref{lem:discretephoa} together with \Cref{cor:connectedpreservediscrete,cor:pointwisecubeorder} imply all discrete sets satisfy Phoa's principle.
\end{example}

The first important implication of working with distributive lattices, besides the symmetry between meets and joins, is the following fact:

\begin{theorem}[SQCF]\label{thm:phoaaffine}
  Any affine space $\spec A$ satisfies Phoa's principle.
\end{theorem}
\begin{proof}
  By assumption $\I$ is affine, thus from \Cref{prop:duality} we have
  \[ \spec A^\I \cong \I\alg(A,\I[i]). \]
  We observe that the pair of evaluation maps
  \[ \pair{\ev_0,\ev_1} : \I[i] \to \I \times \I \]
  classifies the order on $\I$. This follows from a normal form result of polynomials for distributive lattices, i.e. any polynomial $p:\I[i]$ is of the form 
  \[ p = p(0) \vee i \wedge p(1). \]
  See e.g.~\cite[Thm. 10.11]{lausch2000algebra}. This way, a map $\I \to \spec A$ is exactly described as two points $x,y$ such that $\fa aA xa \le ya$, which coincides with $x \preceq y$ by \Cref{lem:specorderofaffine}.
\end{proof}

\begin{remark}[Algebraic properties in classifying topoi]\label{rem:normalalgebra}
  We emphasis that the above proof crucially depends on the fact that for any distributive lattice $A$, the free algebra $A[i]$ is equivalent to the order on $A$, due to the normal form for $p$ in $A[i]$ as 
  \[ p = p(0) \vee i \wedge p(1). \]
  This is a perfect example of how an algebraic property of a theory has a non-trivial effect on the internal logic of its classifying topos.
\end{remark}

At this point, the situation for quasi-coherent algebras is easy to see:

\begin{corollary}[SQCF]
  For a quasi-coherent $\I$-algebra $A$, the specialisation order coincides with its canonical order.
\end{corollary}
\begin{proof}
  If $A$ is quasi-coherent, then $A \cong \I^{\spec A}$ as an $\I$-algebra, and its canonical order thus coincides with the pointwise order on $\I$. On the other hand, it is well-known that the Phoa's principle for the interval $\I$ implies the specialisation order on any type $\I^X$ is pointwise; see e.g.~\cite[Thm. 4.2.1]{hyland2006first}. Hence, the two orders on $A$ indeed coincide.
\end{proof}

\begin{theorem}[SQCF]\label{thm:algebraphoa}
  If both $A$ and $A[i]$ are quasi-coherent, then $A$ satisfies Phoa's principle.
\end{theorem}
\begin{proof}
  By assumption and \Cref{prop:tensorasspace}, we have
  \[ A^\I \cong A^{\spec \I[i]} \cong \I^{\spec A[i]} \cong A[i]. \]
  Since the specialisation order on $A$ coincides with its canonical order, and as mentioned $A[i]$ is equivalent to the canonical order on $A$, $A$ indeed satisfies Phoa's principle.
\end{proof}

At the end of this section, we describe another interesting perspective arising from the proof of \Cref{thm:phoaaffine}. We have seen that the dualising object $\I$ has a double role: It is both an algebra and a spectrum. The proof of Phoa's principle gives us many more such examples. For instance, $\I[i]$ classifies the order on $\I$, which by definition is the spectrum $\Delta_2$. In fact, \emph{all} the simplices is classified by some algebra.

For any $n : \N$ and $\I$-algebra $A$, we write $\Delta[A]^{n}$ as the lists of decreasing elements of length $n$ in $A$,
\[ \Delta[A]^{n} := \scomp{a_1,\cdots,a_n : A}{a_1 \ge \cdots \ge a_n}. \]
Of course $\Delta[\I]^n$ is simply $\Delta^n$. We have the following general algebraic description of $\Delta[A]^n$:

\begin{proposition}\label{prop:simplicesasalgebra}
  For any $n : \N$ and $\I$-algebra $A$, there is an equivalence 
  \[ \Delta[A]^{n+1} \cong A[n]/i_1 \le \cdots \le i_n, \]
  sending any $a_0 \ge \cdots \ge a_n$ in $A$ to the polynomial 
  \[ a_0 \wedge i_1 \vee a_1 \wedge \cdots \wedge i_n \vee a_n. \]
\end{proposition}
\begin{proof}
  The fact that the polynomial is well-defined in $A[n]/i_1 \le \cdots \le i_n$, in the sense that the value does not depend on how one arranges parenthesis, and that this gives an equivalence, again follows from a normal form result on algebras of the form $A[n]/i_1 \le \cdots \le i_n$, which is a consequence of the more general normal form for polynomials on distributive lattices; see~\cite[Thm. 10.21]{lausch2000algebra}.
\end{proof}

\section{Infinitary domain theory}\label{sec:infdomain}

Until this point, we have seen that elementary axioms for synthetic domain theory follow from (SQCF) for distributive lattices ($\sigma$-frames). In this section we show that the infinitary axiom of synthetic domain theory, viz. completeness of the interval $\I$, is also a consequence of quasi-coherence for $\sigma$-frames. In fact, the only place we have used the essentially the properties of $\sigma$-frames is in the proof of \Cref{thm:complete}, and we will also indicate why that fails for distributive lattices in \Cref{rem:whynotdis}.

The connection between the infinitary aspect of synthetic domain theory and quasi-coherence starts from the observation that the final coalgebra $\ov\omega$ for the lifting functor $(-)\prt$ can be described as a \emph{spectrum}:

\begin{example}[$\ov\omega$ is a spectrum]\label{exm:ovomegaaffine}
  Since $(-)\prt$ by construction is a polynomial functor, it preserves all connected limits. Hence, the final coalgebra $\ov\omega$ can be easily characterised as the following sequential limit, 
  \[
  \begin{tikzcd}
    \ov\omega \cong \lt_{n:\N}\Delta^n \ar[r] & \cdots \ar[r] & \Delta^2 \ar[r, "!\prt"] & \I \ar[r, "!"] & 1
  \end{tikzcd}
  \]
  where the transition map $\Delta^{n+1} \to \Delta^n$ takes the sequence $i_0 \ge \cdots \ge i_n$ to the final segment $i_1 \ge \cdots \ge i_n$. As observed in~\cite[Sec. 5.2]{hyland2006first}, it also has an equivalent type-theoretic description as the object of infinite descending sequences in $\I$,
  \[ \ov\omega \cong \scomp{i : \N \to \I}{\fa n\N i_n \ge i_{n+1}}. \]
  This way, we have a natural equivalence
  \[ \ov\omega \cong \spec\I[\N]/\pair{i_n \ge i_{n+1}}_{n:\N}. \]
  Here now $\I[\N]/\pair{i_n \ge i_{n+1}}_{n:\N}$ is a \emph{countably presented} $\I$-algebra.
\end{example}

In general, by a c.p.\ $\I$-algebra we mean one of the form $\I[\N]/\pair{s_n = t_n}_{n:\N}$ for some $\N$-indexed lists of terms $s,t : \N \to \I[\N]$. In particular, all f.p.\ $\I$-algebra will also be c.p.. Motivated by the above characterisation of $\ov\omega$, we naturally consider the following stronger quasi-coherence principle:

\begin{axiom}[SQCC]
  All c.p.\ $\I$-algebras are quasi-coherent.
\end{axiom}

\begin{remark}
  Our modular development in the previous sections can be now immediately applied to $\ov\omega$ if we assume (SQCC). For instance, the description of the specialisation order for affine spaces in \Cref{lem:orderonaffine} now applies to $\ov\omega$, which shows this again coincides with its pointwise order as a subspace of $\I^\N$. In particular, it also has both a top and bottom element, thus \Cref{cor:connectedpreservediscrete} now implies $\ov\omega$ is also internally connected, i.e. $M \to M^{\ov\omega}$ will be an equivalence for any discrete set $M$. 
\end{remark}

(SQCC) combined with the non-triviality axiom (NT) has many logical consequences. The crucial observation is that (NT) implies a weak form of \emph{nullstellensatz} result, as already noted in~\cite{blechschmidt2021using,blechschmidt2020general,Cherubini_Coquand_Hutzler_2024}:

\begin{lemma}[NT]\label{lem:nulls}
  For an affine space $X \cong \spec A$, $X \cong \emp$ iff $A$ is trivial, viz. $0=1$ in $A$.
\end{lemma}
\begin{proof}
  The if direction holds since by (NT) $\I$ is non-trivial, thus there is no homomorphism from a trivial algebra to $\I$. For the only if direction, by assumption $A \cong \I^{\spec A} \cong \I^\emp$, which implies $A$ is trivial. 
\end{proof}

Together with (SQCC), this implies the following form of Markov principle; a similar result is shown in~\cite{cherubini2024foundation}:

\begin{lemma}[NT, SQCC]\label{lem:markov}
  For any $i : \ov\omega$, we have
  \[ \neg\fa{n}{\N}\ms ti_n \to \ex n\N\ms fi_n. \]
\end{lemma}
\begin{proof}
  Let $i : \ov\omega$. Notice that similar to \Cref{lem:openpropaffine}, the proposition $\fa n\N \ms ti_n$ by construction is the following affine space, 
  \[ \spec\I/i \cong \I\alg(\I/i,\I) \cong \fa n\N \ms ti_n, \]
  where we have abbreviated the c.p.\ $\I$-algebra $\I/\pair{i_n=1}_{n:\N}$ as $\I/i$. Now if we have $\neg\fa n\N \ms ti_n$, then $\spec\I/i \cong \emp$ which by \Cref{lem:nulls} implies $\I/i$ is trivial. But this algebra is trivial iff $\ex n\N \ms fi_n$.
\end{proof}

Equipped with the above result, we can now proceed to study the initial algebra $\omega$ for the lifting functor. \cite{JIBLADZE1997185} has given a beautiful formula for a type-theoretic description of $\omega$ as the following subset of $\ov\omega$, 
\[ \omega := \scomp{i : \ov\omega}{\fa\phi{\pp} (\fa n{\N} (\ms ti_n \to \phi) \to \phi) \to \phi}. \]
For another proof, see e.g.~\cite{VANOOSTEN2000233}. In the presence of the Markov principle above, this description can be greatly simplified:

\begin{proposition}[NT, SQCC]\label{prop:omegacolimit}
  $\omega$ is equivalent to the following subset of $\ov\omega$,
  \[ \omega \cong \scomp{i : \ov\omega}{\ex n{\N} \ms fi_n}. \]
  In particular, $\omega$ is the colimit of the following sequence,
  \[ 
  \begin{tikzcd}
    \emp \ar[r, "?"] & 1 \ar[r, "?\prt \cong \eta\cprt"] & \I \ar[r, "\eta\cprt"] & \Delta^2 \ar[r] & \cdots \ar[r] & \ct_{n:\N}\Delta^n \cong \omega
  \end{tikzcd}
  \]
\end{proposition}
\begin{proof}
  Let $i : \ov\omega$. It suffices to show that
  \[ \prth{\fa\phi{\pp} (\fa n{\N} (\ms ti_n \to \phi) \to \phi) \to \phi} \to \ex{n}\N \ms fi_n. \]
  Assume the premise. We can instantiate $\phi$ to $\emp$. By assumption, we have $\neg\fa n\N \neg\neg\ms ti_n$, which by \Cref{cor:opendnegclose} is equivalent to $\neg\fa n\N \ms ti_n$. Then \Cref{lem:markov} implies this is $\ex n\N \ms fi_n$. The fact that this makes $\omega$ into the above sequential colimit follows from~\cite[Cor. 1.10]{VANOOSTEN2000233}.
\end{proof}

\begin{remark}\label{rem:omegaalwayscolim}
  The above characterisation of $\omega$ as a sequential colimit is still true if we work with distributive lattices, rather than $\sigma$-frames.
\end{remark}

Using this colimit description of $\omega$ and the fact that $\ov\omega$ is affine, by working with $\sigma$-frames, we can show $\I$ is complete:

\begin{theorem}[NT, SQCC]\label{thm:complete}
  $\I$ is orthogonal to the inclusion $\omega\hook\ov\omega$.
\end{theorem}
\begin{proof}
  Since $\ov\omega$ is now affine, we have
  \[ \I^{\ov\omega} \cong \I[\N]/\pair{i_n \ge i_{n+1}}_{n:\N}. \]
  On the other hand, since $\omega$ is the colimit of $\Delta^n$ and they are affine, we have
  \[ \I^\omega \cong \lt_{n:\N}\I^{\Delta^n} \cong \lt_{n:\N}\I[n]/i_0\ge\cdots\ge i_{n-1}. \]
  Note that the transition maps induced by $\eta\cprt : \Delta^n \to \Delta^{n+1}$ under quasi-coherence gives us the following maps on algebras:
  \[
  \begin{tikzcd}
    \I^{\Delta^{n+1}} \ar[r, "\I^{\eta\cprt}"] & \I^{\Delta^n} \\ 
    \I[n\!+\!1]/i_0 \ge \cdots \ge i_{n} \ar[u, "\cong"] \ar[r, "i_{n} \mapsto 0"'] & \I[n]/i_1 \ge \cdots \ge i_{n-1} \ar[u, "\cong"']
  \end{tikzcd}
  \]
  Hence, it suffices to show that $\I[\N]/\pair{i_n \ge i_{n+1}}_{n:\N}$ is indeed the sequential limit of the above $\I$-algebras,
  \[\begin{tikzcd}
    \cdots & {\I[n\!+\!1]/i_0 \ge \cdots \ge i_{n}} & {\I[n]/i_0 \ge \cdots \ge i_{n-1}} & \cdots \\
    & {\I[\N]/\pair{i_n \ge i_{n+1}}_{n:\N}}
    \arrow[from=1-1, to=1-2]
    \arrow["{i_{n} \mapsto 0}", from=1-2, to=1-3]
    \arrow[from=1-3, to=1-4]
    \arrow[dashed, from=2-2, to=1-1]
    \arrow["{f_{n+1}}"{description}, from=2-2, to=1-2]
    \arrow["{f_n}"{description}, from=2-2, to=1-3]
    \arrow[curve={height=15pt}, dashed, from=2-2, to=1-4]
  \end{tikzcd}\]
  where the map $f_n$ takes $i_k$ to itself for $k\le n$, and takes $i_k$ to $0$ for $k \ge n$.
  
  This is where it is crucial to work with $\sigma$-frames. We can more directly compute the above sequential limit using \Cref{prop:simplicesasalgebra} as follows,
  \[\begin{tikzcd}
    \cdots & {\I[n\!+\!1]/i_0 \ge \cdots \ge i_{n}} & {\I[n]/i_0 \ge \cdots \ge i_{n-1}} & \cdots \\
    \cdots & {\Delta_{n+2}} & {\Delta_{n+1}} & \cdots
    \arrow[from=1-1, to=1-2]
    \arrow["{{i_{n} \mapsto 0}}", from=1-2, to=1-3]
    \arrow["\cong", from=1-2, to=2-2]
    \arrow[from=1-3, to=1-4]
    \arrow["\cong", from=1-3, to=2-3]
    \arrow[from=2-1, to=2-2]
    \arrow["\pi_{n}", from=2-2, to=2-3]
    \arrow[from=2-3, to=2-4]
  \end{tikzcd}\]
  where $\pi_{n} : \Delta_{n+1} \to \Delta_{n+1}$ forgets the last entry, i.e. it takes $i_0 \le \cdots \le i_{n+1}$ to $i_0 \le \cdots \le i_{n}$. Hence, the sequential limit is given by
  \[ \lt_{n:\N} \Delta_{n} \cong \Delta_\infty := \scomp{i : \N \to \I}{\fa n\N i_n \le i_{n+1}}. \]
  Now it suffices to show that $\Delta_\infty$ gives a normal form for $\I[\N]/\pair{i_n \ge i_{n+1}}_{n:\N}$. This is indeed true for $\sigma$-frames: By a generalisation or the results in \Cref{prop:simplicesasalgebra}, an element $p$ in $\I[\N]/\pair{i_n \ge i_{n+1}}_{n:\N}$ can be uniquely written as the \emph{countable join},
  \[ p = \bigvee_{n:\N}p_n \wedge i_n, \]
  with $p_n \le p_{n+1}$ for all $n:\N$. A more detailed proof is given in \Cref{app:normalsigma}. This identifies $\I[\N]/\pair{i_n \ge i_{n+1}}_{n:\N}$ with $\Delta_\infty$, and completes the proof.
\end{proof}

\begin{remark}
  As indicated in \Cref{rem:qcreplete,rem:specarereplete}, both spectrums and quasi-coherent $\I$-algebras are replete. The above result then implies that they are also complete, i.e. orthogonal to $\omega\hook\ov\omega$.
\end{remark}

\begin{corollary}[NT, SQCC]
  $\I^{\ov\omega} \cong \I^\omega \cong \Delta_\infty$. In particular, $\omega$ is not affine.
\end{corollary}
\begin{proof}
  By the above proof, we have $\I^{\ov\omega} \cong \I[\N]/\pair{i_n \ge i_{n+1}}_{n:\N} \cong \Delta_\infty$.
\end{proof}

\begin{remark}[$\sigma$-frames v.s. distributive lattices]\label{rem:whynotdis}
  The completeness result shown in \Cref{thm:complete} is the only place in this paper where it works for $\sigma$-frames while not for distributive lattices. The reason is exactly because the c.p.\ distributive lattice $\I[\N]/\pair{i_n \ge i_{n+1}}_{n:\N}$ will \emph{not} be isomorphic to $\Delta_\infty$, thus the cone $\set{f_n}_{n:\N}$ displayed in the proof of \Cref{thm:complete} will \emph{not} be a limiting diagram of distributive lattices. The fact that we have \emph{all} infinite increasing sequences is clearly due to the fact that we have countable disjunctions in $\sigma$-frames, as when identifying $p$ in $\I[\N]/\pair{i_n \ge i_{n+1}}_{n:\N}$ with $p = \bigvee_{n:\N}p_n\wedge i_n$. More specifically, for the c.p.\ distributive lattice $\I[\N]/\pair{i_n \ge i_{n+1}}_{n:\N}$, since it is a sequential colimit of the following f.p.\ distributive lattices
  \[ 
  \begin{tikzcd}
    \cdots \ar[r] & \I[n]/i_{0} \ge \cdots \ge i_{n-1} \ar[r, hook] & \I[n\!+\!1]/i_0 \ge \cdots \ge i_n \ar[r, hook] & \cdots
  \end{tikzcd}
  \]
  and, due to the fact that the theory of distributive lattices is \emph{finitary}, this sequential colimit of \emph{algebras} is computed the same as their \emph{underlying sets}. Via \Cref{prop:simplicesasalgebra}, the result can be identified as the subtype of $\Delta_\infty$,
  \[ \Delta_\omega := \scomp{i : \Delta_\infty}{\ex n\N \ms ti_n}, \]
  where $\Delta_\omega \hook \Delta_\infty$ in some sense is the order-dual to the inclusion $\omega\hook\ov\omega$.
\end{remark}

\begin{remark}[Limits of algebras induce locality for $\I$]\label{rem:limofalgloc}
  By inspecting the proof of \Cref{thm:complete}, it is clear that the reason $\I$ is complete is because there is a specific \emph{limiting diagram} of quasi-coherent $\I$-algebras. In general, any such limit diagram will induce an orthogonality property satisfied by the all spectrums, and we will see many more examples in \Cref{sec:local}. From \Cref{rem:whynotdis} it is clear whether a diagram of $\I$-algebras is a limit heavily relies on the underlying algebraic theory. This is thus a perfect example of how the algebraic properties of a theory has significant consequence on the internal logic of its classifying topos.
\end{remark}

% \section{specialisation posets continued}\label{sec:synposet}

% The aim of this section is to first show that specialisation posets are exactly the \emph{replete objects} in the sense of~\cite{hyland2006first}, and that they satisfy some closure properties under certain colimits.

% We say a type $X$ is \emph{$f$-local} if $X^f$ is an equivalence. Dually, we also say $f$ is \emph{$X$-trivial}.

% \begin{definition}
%   A type $X$ is \emph{replete}, if it is $f$-local for all $\I$-trivial maps $f$.
% \end{definition}

% Though this is a very big orthogonality class, it is shown in~\cite{hyland2006first} that it is \emph{reflective}, and the reflector is the image of the canonical map $X \to \I^{\I^X}$.

% \begin{lemma}[SQCF]
%   $X$ is replete iff its specialisation order is anti-symmetric.
% \end{lemma}
% \begin{proof}
%   For any $x,y : X$, since $\I^{\I^X}$ is a set, they are equal in $\I^{\I^X}$ iff
%   \[ \fa U{X \to \I} Ux \cong Uy, \]
%   which is equivalent to $x \preceq y \wedge y \preceq x$. Hence, $X \to \I^{\I^X}$ is an embedding iff $x \cong y \to x \preceq y \wedge y \preceq x$ is an equivalence, which exactly means the specialisation order on $X$ is anti-symmetric.
% \end{proof}

% \begin{lemma}\label{lem:affspspecialisationposet}
%   All affine spaces are specialisation posets.
% \end{lemma}
% \begin{proof}
%   As shown in \Cref{thm:phoaaffine}, any affine space $X \cong \spec A$ is Phoa, and the specialisation order on $X$ is induced by the pointwise order on $\I\alg(A,\I)$, which is anti-symmetric.
% \end{proof}

% \begin{lemma}[SQCF]\label{lem:qcalgebraintposet}
%   All quasi-coherent $\I$-algebras are specialisation posets.
% \end{lemma}
% \begin{proof}
%   Again, by \Cref{them:phoaalgebra}, a quasi-coherent algebra $A$ is Phoa, and the specialisation order coincides with the order on $A$.
% \end{proof}

% specialisation posets satisfy the closure properties of an orthogonality class:

% \begin{lemma}[SQCF]
%   The specialisation order on limits of specialisation posets is pointwise, and are again specialisation posets. They in fact form an exponential ideal.
% \end{lemma}
% \begin{proof}
%   Suppose we have a limit $\lt_{r:I}X_r$ of specialisation posets. Given $(x_r)_{r:I}$ and $(y_r)_{r:I}$ in the limit, by monotonicity in \Cref{lem:anymapmonotoneintriscorder}, if $(x_r)_{r:I} \preceq (y_r)_{r:I}$, then the projection maps imply that $x_r \preceq y_r$ for all $r:I$. On the other hand, if $x_r \preceq y_r$ for all $r:I$, then since each $X_r$ is Phoa, we get a unique map $f_r : \I \to X_r$. We then need to show that this gives us $(f_r)_{r:I} : \I \to X_r$. This indeed follows from the Phoa's principle on each $X_r$. Hence, the order on the limit is again pointwise, and the limit is Phoa. Being an exponential ideal follows from~\ref{prop:phoaexponential}.
% \end{proof}

% Perhaps more surprisingly, specialisation posets will also be closed under various colimit constructions. To show this, we observe the following result:

% \begin{lemma}[NT, SQCF]
%   Let $X$ be a type with connected specialisation order. $(-)^X$ preserves dependent sums over discrete sets.
% \end{lemma}
% \begin{proof}
%   Suppose we have a discrete set $M$ and a family $P$ of types over $M$. Consider the canonical map
%   \[ \sum_{m:M} P_m^X \to \prth{\sum_{m:M}P_m}^X. \]
%   Since $X$ is connected, there merely is $x : X$. For any $f : X \to \sum_{m:M}P_m$, by \Cref{cor:connectedpreservediscrete}, $\pi_Mf(y) \cong \pi_Mf(x)$ for all $y : X$, thus $f$ factors through $P_m^X$, which then implies the fibres of the above map is contractible since $M$ is a discrete set.
% \end{proof}

% \begin{corollary}[NT, SQCF]
%   specialisation posets are closed under dependent sums over discrete sets.
% \end{corollary}

% \begin{definition}\label{def:projective}
%   A type $X$ is \emph{projective}, if for any family $P$ of sets
%   \[ \prod_{x:X}\pss{P_x} \to \pss{\prod_{x:X}P_x}. \]
% \end{definition}

% \begin{lemma}[NT, SQCF]
%   Let $X$ be a projective type with connected specialisation order. $(-)^X$ preserves sequential colimits.
% \end{lemma}
% \begin{proof}
%   For a family of sets $Y_n$ indexed by $\N$ with maps $f_n : Y_n \to Y_{n+1}$, recall from~\cite{10.1145/3373718.3394801} the colimit $\ct_{n:\N}Y_n$ is again a set, which is constructed as in usual set theory. Consider the canonical map
%   \[ \ct_{n:\N}Y_n^X \to \prth{\ct_{n:\N}Y_n}^X. \]
%   To show this is injective, it suffices to assume we have two elements $[(n,y)]$ and $[(m,z)]$. If they agree in the target, which means
%   \[ \fa xX \ex l\N f_{ln}(yx) \cong f_{lm}(zx). \]
%   By projectivity of $X$, we merely have a section $l : X \to \N$. By \Cref{cor:connectedpreservediscrete}, this section is indeed constant at $l \cong l(x)$ for the mere element $x:X$. It then follows that $[(n,y)] \cong [(m,z)]$.

%   On the other hand, by projectivity the above map is also surjective, since if we have $f : X \to \ct_{n:\N}Y_n$, then we have
%   \[ \fa xX \ex n\N\ex{y}{Y_n} fx \cong [(i,y)]. \]
%   By projectivity again, we merely have a section $X \to \N$, which is again constant by \Cref{cor:connectedpreservediscrete}. Thus we merely get a preimage.
% \end{proof}

% \begin{corollary}[NT, SQCF]\label{cor:intposetseqcolimit}
%   Assuming $\I$ is projective, specialisation posets are closed under sequential colimits.
% \end{corollary}

% \begin{remark}
%   This is the first place in this work where we require a form of projectivity of $\I$. $\I$ will indeed be projective, in fact \emph{projective} in the sense that \Cref{def:projective} not only applies to sets but arbitrary types, in all the models we care about.
% \end{remark}

% \begin{example}
%   Notice that as we have shown in \Cref{sec:infdomain} that, under (NT) and (SQCC), $\omega$ is the colimit of simplices $\Delta_n$. By \Cref{cor:intposetseqcolimit}, assuming $\I$ is projective, $\omega$ will also be a specialisation poset.
% \end{example}

% \section{Lifting continued}

% We would like to furthermore show that specialisation posets are closed under lifting. To this end, we introduce a stronger locality axiom:

% \begin{axiom}[L]
%   $\I$ is local, i.e. $0 \neq 1$ and $\ms t(i\vee j) \eq \ms ti \vee \ms tj$ for all $i,j : \I$.
% \end{axiom}

% We observe some elementary consequences of (L):

% \begin{lemma}[L, SQCF]
%   $\I$ is not Boolean.
% \end{lemma}
% \begin{proof}
%   If $\I$ is Boolean, then $\I \cong 2$ by \Cref{prop:filed}, since in that case we have for any $i : \I$,
%   \[ \ms ti \vee \ms fi \cong \ms ti \vee \ms t\neg i \cong \ms t(i \vee \neg i) \cong 1. \]
%   However, this contradicts \Cref{lem:discretephoa}, because if $\I \cong 2$ then $2^\I \cong 2^2$, which cannot be equivalent to $2$.
% \end{proof}

% Furthermore, (L) also allows us to define more affine spaces:

% \begin{example}[L]
%   Consider the right outer horn $\Lambda^2_2$ as a pushout,
%   \[
%     \begin{tikzcd}
%       1 \ar[d, "1"'] \ar[r, "1"] & \I \ar[d] \\
%       \I \ar[r] & \Lambda^2_2
%       \arrow["\lrcorner"{anchor=center, pos=0.125, rotate=180}, draw=none, from=2-2, to=1-1]
%     \end{tikzcd}
%   \]
%   By viewing $\Lambda^2_2$ as a subspace of $\I^2$, we may identify it as follows,
%   \[ \Lambda^2_2 \cong \scomp{i,j : \I}{\ms ti \vee \ms tj}. \]
%   Since $\ms ti \vee \ms tj \cong \ms t(i\vee j)$, it follows that 
%   \[ \Lambda^2_2 \cong \spec \I[i,j]/i \vee j. \]
% \end{example}

% The most important consequence of (L) is that it allows us to show specialisation posets are closed under dependent sums over certain affine spaces:

% \begin{definition}[SQCF]
%   We say an affine space $X$ is \emph{finitary}, if it is of the form $\spec\I[n]/s=t$ for some finitely presented $\I$-algebra. $X$ is thus a subspace of $\I^n$, and we say $X$ is \emph{topped}, if $1 : \I^n$ belongs to $X$.
% \end{definition}

% \begin{example}
%   All the finitary examples of affine spaces considered so far, including the cubes $\I^n$, the simplices $\Delta^n$ and $\Delta_n$, and the outer horn $\Lambda^2_2$, are all topped.
% \end{example}

% One interesting fact is that, since $\I$ is a dominance, it will always be orthogonal to the following family of maps:

% \begin{lemma}[SQCF]\label{lem:toppedfamilyortho}
%   Let $X$ be a finitary topped affine space, $Y : X \to \tp$ a type family. Then $\I$ is weakly orthogonal to the embedding $Y_1 \hook \sum_{i : X}Y_i$,
%   \[
%   \begin{tikzcd}
%     Y_1 \ar[d, hook] \ar[r, "f"] & \I \\ 
%     \sum_{i : X} Y_i \ar[ur, dashed, "\qsi f"']
%   \end{tikzcd}
%   \]
% \end{lemma}
% \begin{proof}
%   Using the dominance structure, we define $\qsi f$ as follows,
%   \[ \qsi f(i,y) := \sum_{*:\ms ti} f(y). \]
%   This is well-defined, and evidently extends $f$.
% \end{proof}

% For any finitary topped $X$, by viewing it as a subspace of $\I^n$, we may define the generalised lifting functor over $X$ as follows:
% \[ L_XY := \sum_{i:X}Y^{\ms ti}. \]
% Of course, $L_\I$ is the usual lifting functor $(-)\prt$. The most crucial observation of this section is that specialisation posets are closed under generalised lifting:

% \begin{theorem}[L, SQCF]\label{thm:specialisationordergenlifting}
%   Let $X$ be a finitary topped affine space and $Y$ a specialisation poset. Then the specialisation order on $L_XY$ is given by
%   \[ (i,x) \preceq (j,y) \eq \ms ti \to (\ms tj \wedge x \preceq_{Y} y). \]
%   In particular, $L_XY$ is again a specialisation order.
% \end{theorem}
% \begin{proof}
%   Recall from \Cref{lem:orderonaffine} the specialisation order on $X$ is induced from the pointwise order on $\I^n$. For the left to right direction, if $(i,x) \preceq (j,y)$, then by monotonicity $i \le j$. If $\ms ti$ holds, then $\ms tj$ holds, and $x \preceq_Y y$ since by \Cref{lem:toppedfamilyortho} any map $Y \to \I$ extends to one on $L_XY$.

%   On the other hand, suppose we have $(i,x) \preceq (j,y)$ in $L_XY$. We construct a map $f : \I \to L_XY$. On the base, it must take $k$ to $i \vee k \wedge j$. Hence, assuming $\ms t(i \vee k \wedge j)$, we need an element in $Y$. By (L), equivalently $\ms ti \vee \ms tk \wedge \ms tj$. Thus, we can define this by case distinction:
%   \begin{itemize}
%     \item If $\ms ti$ holds, then since $x \preceq_Y y$, by the fact that $Y$ is a specialisation poset, we get a unique map $g : \I \to X$, and we define $f(k) := g(k)$.
%     \item If $\ms tk \wedge \ms tj$ holds, we simply put $fk := y$.
%   \end{itemize}
%   This is evidently well-defined, thus we get a map $f : \I \to L_XY$.
% \end{proof}

% Of course, we can also consider the dual notion of \emph{bottomed} finitary affine spaces, which are affine subspaces of $\I^n$ containing $0$. However, to develop the same arguments, we would have to assume furthermore the \emph{colocality} axiom as follows,

% \begin{axiom}[cL]
%   $\I$ is colocal, i.e. $0 \neq 1$ and $\ms f(i \wedge j) \eq \ms fi \vee \ms fj$.
% \end{axiom}

% And in this case, for any bottomed finitary affine space $X$ we can construct the generalised colifting of a type $Y$ as
% \[ T_XY := \sum_{i:X}Y^{\ms fi}. \]
% We then similarly have the following result:

% \begin{theorem}[cL, SQCF]\label{thm:specialisationordergencolifting}
%   Let $X$ be a finitary bottomed affine space and $Y$ a specialisation poset. Then the specialisation order on $T_XY$ is given by
%   \[ (i,x) \preceq (j,y) \eq \ms fj \to (\ms fi \wedge x \preceq_{Y} y). \]
%   In particular, $T_XY$ is again a specialisation order.
% \end{theorem}

% Of course, it is perfectly fine to assume both (L) and (cL), and we may refer to the combination of them as (bL). Working in the context of (bL), we again recover the full duality between the meet and join structure as the presheaf geometry.

\section{Local properties for the interval}\label{sec:local}

In this section we review some of the locality axioms we have introduced in \Cref{sec:locality}. As mentioned in the introduction, the observation is that even if we do not assume them to be true globally for the interval $\I$, we can still show the maps representing the locality axiom to be \emph{orthogonal} to $\I$. As explained in \Cref{rem:limofalgloc}, the reason we have these orthogonality conditions is due to the fact that we have various \emph{limits} of $\I$-algebras. Unlike completeness in \Cref{thm:complete}, the local conditions in \Cref{sec:locality} are all induced by limits of \emph{f.p.} $\I$-algebras. Thus in this section, working with distributive lattices or $\sigma$-frames does not matter.

Starting from the simplest example, let us recall the local property (NT). In general, the proposition $0 = 1$ as the spectrum $\spec(\I/0=1)$ is \emph{not} $\emp$, if we do not assume (NT). However, we can look at the localisation class which thinks this map is invertible. The following fact shows that, from the perspective of $\I$, (NT) indeed holds:

\begin{proposition}[SQCI]\label{specisnontrivial}
  $\I$ is orthogonal to $\emp \hook \spec(\I/0=1)$.
\end{proposition}
\begin{proof}
  By assumption,
  \[ \I^{\spec(\I/0=1)} \cong (\I/0=1) \cong \I^\emp. \qedhere \]
\end{proof}

By inspecting the proof above, one can see that the reason it works is due to quasi-coherence, and the fact that $\I/0=1$ is the \emph{terminal} $\I$-algebra, which is the simplest example of a limit.

As another example, consider the simplicial axiom (SL). We skip the discussion of (L) and (cL), because (SL) is strictly stronger, and it has a better-known geometric meaning. Recall from \Cref{ax:SL} that (SL) can be represented geometrically as an embedding $\Delta^2\cup\Delta_2 \hook \I^2$.

\begin{definition}[Simplicial type]
  A type $X$ is \emph{simplicial}, if it is orthogonal to $\Delta^2 \cup \Delta_2 \hook \I^2$.
\end{definition}

Again, (SL) holds globally iff the above embedding is an equivalence, thus every type in this case will be simplicial. When it does not hold globally, we can still show:

\begin{proposition}[SQCF]\label{specissimplicial}
  $\I$ is simplicial.
\end{proposition}
\begin{proof}
  To show $\I$ is orthogonal to the embedding, equivalently it means for any $f,g$ indicated below making the solid diagram commute, there exists a unique lift $h$ making the whole diagram commute,
  \[\begin{tikzcd}
    \I \ar[r, "\delta"] \ar[d, "\delta"'] & \Delta^2 \ar[r, "f"] \ar[d] & \I \\
    \Delta_2 \ar[urr, "g"{description, near start}] \ar[r] & \I^2 \ar[ur, dashed, "h"']
    \arrow["\lrcorner"{anchor=center, pos=0.125}, draw=none, from=1-1, to=2-2]
  \end{tikzcd}\]
  Now by (SQCF), since the vertices of the left square are all affine, equivalently it suffices to show we have a pullback of $\I$-algebras,
  \[\begin{tikzcd}
    {\I[i,j]} \ar[r] \ar[d] & {\I[i,j]/i \ge j} \ar[d, "{i,j \mapsto k}"] \\
    {\I[i,j]/i \le j} \ar[r, "{i,j\mapsto k}"'] & {\I[k]}
  \end{tikzcd}\]
  Recall from \Cref{prop:simplicesasalgebra} the normal form of elements in $\I[i,j]/i\ge j$. An element $p$ there can be viewed as the polynomial 
  \[ p = p_{0,0} \vee i \wedge p_{1,0} \vee j \wedge p_{1,1}, \]
  with $p_{0,0} \le p_{1,0} \le p_{1,1}$. And similarly for $q$ in $\I[i,j]/i \le j$, 
  \[ q = q_{0,0} \vee j \wedge q_{0,1} \vee i \wedge q_{1,1}. \]
  They agree in $\I[k]$ iff 
  \[ p_{0,0} = q_{0,0}, \quad p_{1,1} = q_{1,1}. \]
  This then gives us a polynomial in $\I[i,j]$, which we write it as follows,
  \[ (p_{0,0} \vee i \wedge p_{1,0}) \vee j \wedge (q_{0,1} \vee i \wedge q_{1,1}) \]
  where $p_{0,0} = q_{0,0} \le q_{0,1}$ and $p_{1,0} \le p_{1,1} = q_{1,1}$. 
  By the general normal form for $A[i]$ for any $\I$-algebra $A$, if we view $\I[i,j]$ as $\I[i][j]$, the above exactly corresponds to the normal form of polynomials in $\I[i,j]$; also see~\cite[Thm. 10.21]{lausch2000algebra}. This means the above is a pullback of $\I$-algebras.
\end{proof}



Finally, we discuss the even stronger locality condition (1T). As mentioned in \Cref{ax:1T}, the additional property of (1T) is characterised by the embedding $\partial\Delta_2 \hook \Delta_2$.

\begin{definition}[1-truncated type]
  A type $X$ is \emph{1-truncated}, if it is simplicial, and it is also orthogonal to $\partial\Delta_2 \hook \Delta_2$.
\end{definition}

\begin{proposition}[SQCF]\label{specis1t}
  $\I$ is 1-truncated.
\end{proposition}
\begin{proof}
  Completely similar to the proof of \Cref{specissimplicial}, it suffices to show we have a limit diagram of $\I$-algebras as follows,
  \[\begin{tikzcd}
    & {\I[i,j]/i\le j} \\
    \I[i] & \I[k] & \I[j] \\
    \I & \I & \I
    \arrow["{j \mapsto 1}"', from=1-2, to=2-1]
    \arrow["{i,j \mapsto k}"{description}, from=1-2, to=2-2]
    \arrow["{i \mapsto 0}", from=1-2, to=2-3]
    \arrow["i \mapsto 1"', from=2-1, to=3-1]
    \arrow["i \mapsto 0"{description, pos=0.75}, from=2-1, to=3-2]
    \arrow["k \mapsto 1"{description, pos=0.75}, from=2-2, to=3-1]
    \arrow["k \mapsto 0"{description, pos=0.75}, from=2-2, to=3-3]
    \arrow["j \mapsto 1"{description, pos=0.75}, from=2-3, to=3-2]
    \arrow["j \mapsto 0", from=2-3, to=3-3]
  \end{tikzcd}\]
  Now by normal form $\I[i]$, an element in the limit consists of $p$ in $\I[i]$, $q$ in $\I[j]$ and $r$ in $\I[k]$, such that
  \[ p_1 = r_1, \quad p_0 = q_1, \quad r_0 = q_0. \]
  This exactly corresponds to a normal form in the algebra $\I[i,j]/i\le j$ with $q_0 \le q_1 = p_0 \le p_1$ as follows (cf. \Cref{prop:simplicesasalgebra})
  \[ q_0 \vee j \wedge q_1 \vee i \wedge p_1. \]
  Hence, the above is a limit diagram of $\I$-algebras.
\end{proof}

Besides the locality principles discussed in \Cref{sec:locality}, we also consider the orthogonality classes emerging from \emph{synthetic category theory}, as first introduced in~\cite{riehl2017type}. A synthetic category will be a type that satisfies the internal orthogonality condition of being \emph{simplicial}, \emph{Segal}, and \emph{Rezk}.

We start with the Segal condition. Besides the outer horn discussed in \Cref{exm:hornaffine}, we can also define the inner horn $\Lambda^2_1$ as a pushout,
\[
  \begin{tikzcd}
    1 \ar[d, "0"'] \ar[r, "1"] & \I \ar[d] \\
    \I \ar[r] & \Lambda^2_1
    \arrow["\lrcorner"{anchor=center, pos=0.125, rotate=180}, draw=none, from=2-2, to=1-1]
  \end{tikzcd}
\]
As a subtype of $\Delta_2$, it can be identified as $\scomp{(i,j) : \Delta_2}{\ms fi \vee \ms tj}$.

\begin{definition}[Segal types]
  $X$ is \emph{Segal} if it is orthogonal to $\Lambda^2_1 \hook \Delta_2$.
\end{definition}

\begin{proposition}[SQCF]
  $\I$ will be Segal.
\end{proposition}
\begin{proof}
  In this case, we need to show the following diagram is a pullback of $\I$-algebras, 
  \[
  \begin{tikzcd}
    \I[i,j]/i\le j \ar[r, "j \mapsto 1"] \ar[d, "i \mapsto 0"'] & \I[i] \ar[d, "i \mapsto 0"] \\
    \I[j] \ar[r, "j \mapsto 1"'] & \I
  \end{tikzcd}
  \]
  Again by the normal form theorem, an element in the pullback is given by $p$ in $\I[i]$ and $q$ in $\I[j]$ with $p_0 = q_1$. This way, it again corresponds to the following normal form in $\I[i,j]/i \le j$ by \Cref{prop:simplicesasalgebra},
  \[ q_0 \vee j \wedge q_1 \vee i \wedge p_1. \]
  Hence, the above is again a pullback.
\end{proof}

Next we consider the Rezk condition. Following~\cite{buchholtz2021synthetic}, we can define the type $\mbb E$ classifying categorical equivalences as the following colimit,
\[
\begin{tikzcd}
	& \I && \I && \I \\
	1 && {\Delta_2} && {\Delta_2} && 1
	\arrow[from=1-2, to=2-1]
	\arrow["{d_1}"{description}, from=1-2, to=2-3]
	\arrow["{d_2}"{description}, from=1-4, to=2-3]
	\arrow["{d_0}"{description}, from=1-4, to=2-5]
	\arrow["{d_1}"{description}, from=1-6, to=2-5]
	\arrow[from=1-6, to=2-7]
\end{tikzcd}
\]

\begin{definition}[Rezk types]
  We say $X$ is \emph{Rezk}, if it is orthogonal to $\mbb E \to 1$.
\end{definition}

\begin{proposition}[SQCF]
  $\I$ is Rezk.
\end{proposition}
\begin{proof}
  Notice by \Cref{thm:phoaaffine} $\I^\I$ classifies the canonical order on $\I$. This order is antisymmetric, thus $\I$ will be Rezk.
\end{proof}

In fact, $\I$ is slightly stronger than being a synthetic category, in the sense that they will be synthetic \emph{poset}. We construct $\I_{\rightrightarrows}$ as the following pushout,
\[
\begin{tikzcd}
  2 \ar[r] \ar[d] & \I \ar[d] \\ 
  \I \ar[r] & \I_{\rightrightarrows}
  \arrow["\lrcorner"{anchor=center, pos=0.125, rotate=180}, draw=none, from=2-2, to=1-1]
\end{tikzcd}
\]

\begin{definition}[$\I$-separated types]
  $X$ is \emph{$\I$-separated} if it is orthogonal to $\I_{\rightrightarrows} \to \I$.
\end{definition}

Equivalently, $X$ is $\I$-separated iff $X^\I \to X \times X$ is an embedding. The following is evident:

\begin{proposition}[SQCF]
  $\I$ is $\I$-separated. Any type $X$ satisfying Phoa's principle is also $\I$-separated.
\end{proposition}
\begin{proof}
  Since $\I^\I$ classifies the canonical order on $\I$ which is a partial order, it will be $\I$-separated. For types $X$ satisfying the Phoa's principle, since $X^\I$ classifies the specialisation order on $X$, which is again a partial order, they will be $\I$-separated.
\end{proof}

The notion of synthetic categories and synthetic posets are formulated as these orthogonality classes:

\begin{definition}[Synthetic categories and synthetic posets]
  A type $X$ is a \emph{synthetic category}, if it is simplicial, Segal, and Rezk. We say it is a \emph{synthetic poset}, if it is also $\I$-separated.
\end{definition}

\begin{theorem}[SQCF]\label{thm:affineposet}
  Any spectrum or quasi-coherent $\I$-algebra will be a synthetic poset.
\end{theorem}
\begin{proof}
  This follows from $\I$ being a synthetic poset, and spectrums and quasi-coherent algebras are \emph{replete} as indicated in \Cref{rem:qcreplete,rem:specarereplete}.
\end{proof}

At the end of this section, we also discuss the example of $\omega$, which as we have seen is \emph{not} affine. We show $\omega$ satisfies the Phoa's principle, and satisfies all the locality conditions discussed above. 

Recall from \Cref{prop:omegacolimit} that, under (NT) and (SQCC), $\omega$ is a colimit $\omega \cong \ct_{n:\N}\Delta^n$. As indicated in \Cref{rem:omegaalwayscolim}, this does not depend on working with distributive lattices or $\sigma$-frames. Type-theoretically, we have also shown that $\omega$ can be realised as the following subspace of $\ov\omega$,
\[ \omega \cong \scomp{i : \ov\omega}{\ex n\N \ms fi_n}. \]
This way, the inclusion $\Delta^n \hook \omega$ can be viewed as follows, 
\[ \Delta^n \cong \scomp{i : \omega}{\ms fi_{n}}, \]
which implies $\Delta^n \hook \omega$ is \emph{downward closed}. This allows us to directly compute the function space $\omega^\I$:

\begin{lemma}[NT, SQCC]\label{lem:intervalcommuteomega}
  We have a family of equivalences
  \[ \omega^\I \cong \prth{\ct_{n:\N}\Delta^n}^\I \cong \ct_{n:\N}(\Delta^n)^\I, \]
  and similarly by replacing $\I$ with $\I^n$ or $\Delta^n$.
\end{lemma}
\begin{proof}
  We show the following canonical map is an equivalence,
  \[ \ct_{n:\N}(\Delta^n)^\I \to \prth{\ct_{n:\N}\Delta^n}^\I. \]
  It is evident this map is an embedding, hence it suffices to show it is surjective. Given $f : \I \to \ct_{n:\N}\Delta^n$. By assumption, there merely exists $n:\N$ that $f(1)$ factors through $\Delta^n \hook \omega$. Now the claim is that the entire map $f$ factors through $\Delta^n$. This indeed holds, since we have shown in \Cref{cor:omegaphoa} that the specialisation order on $\omega$ is pointwise. By monotonicity, for any $i:\I$ we have $fi \preceq f1$, which implies $fi$ belongs to $\Delta^n$ as well, since $\Delta^n \hook \omega$ is a downward closed. The same holds for cubes or simplices since they all have a top element.
\end{proof}

As a first consequence, we show $\omega$ satisfies Phoa's principle:

\begin{proposition}[NT, SQCC]\label{cor:omegaphoa}
  The specialisation order on $\omega$ again coincides with its canonical pointwise order. In particular, $\omega$ also satisfies the Phoa's principle.
\end{proposition}
\begin{proof}
  For $i \le j$ in $\omega$, since $j$ merely belongs to $\Delta^n$ for some $n$. By downward closeness of $\Delta^n$ in $\omega$, $i,j$ both belongs to $\Delta^n$. Thus by \Cref{thm:phoaaffine} and \Cref{lem:intervalcommuteomega}, we get a map $f : \I \to \omega$ with $f(0) = i$ and $f(1) = j$. Hence, $i \preceq j$. Now if $i \preceq j$ in $\omega$, by \Cref{lem:anymapmonotoneintriscorder} we also have $i \preceq j$ in $\ov\omega$. Since $\ov\omega$ is affine, $i \le j$ in $\ov\omega$ by \Cref{thm:phoaaffine} again. This implies $i \le j$ in $\omega$. The uniqueness of $f : \I \to \omega$ follows from $\ov\omega$ satisfying the Phoa's principle and the fact that $\omega\hook\ov\omega$ is an embedding.
\end{proof}

As another consequence, we can also show the following general result establishing a large family of orthogonality conditions satisfied by $\omega$:

\begin{theorem}[NT, SQCC]\label{thm:omegaortho}
  Let $f : X \to Y$ be a map where $X,Y$ are finite colimits of cubes or simplices. If each $\Delta^n$ is $f$-local, then so is $\omega$.
\end{theorem}
\begin{proof}
  Let $Y \cong \ct_{r}Y_r$ be a finite colimit with each $Y_r$ a simplice or a cube.
  \[ \omega^Y \cong \lt_r\prth{\ct_{n:\N}\Delta^n}^{Y_r} \cong \lt_r\ct_{n:\N}(\Delta^n)^{Y_r} \cong \ct_{n:\N}\lt_r(\Delta^n)^{Y_r} \cong \ct_{n:\N}(\Delta^n)^Y \]
  The second equivalence holds by \Cref{lem:intervalcommuteomega}; the third holds since finite limits commutes with sequential colimits. Thus, if each $\Delta^n$ is $f$-local, i.e. $(\Delta^n)^Y \cong (\Delta^n)^X$, then so is $\omega$.
\end{proof}

\begin{remark}[Localisation classes closed under sequential colimits]
  Notice that in \Cref{lem:intervalcommuteomega} we do not need to use any form of choice principle to show the exponential $(-)^\I$ commutes with the sequential colimit $\ct_{n:\N}\Delta^n$, exactly because $\Delta^n \hook \omega$ is downward closed. However, for general sequential colimit, the same proof still goes through if we assume $\I$ satisfies a choice principle: For any type family $P$ over $\I$, we have
  \[ \prod_{i:\I}\pss{P(i)} \to \pss{\prod_{i:\I}P(i)}. \]
  Furthermore, the same holds for cubes and simplices since the types satisfying the choice principle are closed under finite products and retracts. Assuming $\I$ satisfies choice, following the proof of \Cref{thm:omegaortho}, one can show more generally that any orthogonality class specified by maps between finite colimits of simplices or cubes are always \emph{closed under sequential colimits}. However, not in every model of quasi-coherence $\I$ will satisfy the above choice principle. Hence, we do not include this result in the main text, as we tend to keep our assumptions as minimalistic as possible.
\end{remark}

% we would like to show that the previous orthogonality classes have various closure properties. In particular, we would like to show they are closed under \emph{dependent sums over discrete sets}, and \emph{sequential colimits}. 

% \begin{lemma}[NT, SQCF]\label{lem:disccoprod}
%   $\prod_{i:\N}$ commutes with $\sum_{m:M}$ with $M$ discrete, i.e. for any type family $P$ over $\I,M$, the canonical map
%   \[ \sum_{m:M}\prod_{i:\N}P_{i,m} \to \prod_{i:\I}\sum_{m:M}P_{i,m} \]
%   is an equivalences.
% \end{lemma}
% \begin{proof}
%   By definition, this map is an embedding, hence it suffices to show surjectivity. But this follows from \Cref{cor:connectedpreservediscrete}.
% \end{proof}

% \begin{proposition}[NT, SQCF]\label{prop:closuresum}
%   For an orthogonality condition $f : X \to Y$, if both $X,Y$ are connected colimits of cubes or simplices, then $f$-local objects will be closed under dependent sums indexed by discrete sets.
% \end{proposition}
% \begin{proof}
%   Let $Y \cong \ct_{r}Y_r$ be a connected colimit of cubes or simplices,
%   \[ \prth{\sum_{m:M}P_m}^Y \cong \lt_r\prth{\sum_{m:M}\lt_{r}P_m}^{Y_r} \cong \sum_{m : M}\lt_r P_m^{Y_r} \cong \sum_{m:M} P_m^Y. \]
%   The second equivalence uses the fact that finite connected limits commutes with dependent sums, and by \Cref{lem:disccoprod} $Y_r$ also commutes with dependent sums indexed by $M$. Similarly for $X$. Hence, if each $P_m$ is $f$-local, then so is $\sum_{m:M}P_m$.
% \end{proof}

% \begin{corollary}
%   All the orthogonality classes introduced in this section are closed under dependent sums indexed by discrete sets.
% \end{corollary}

% Completely similarly, we consider the case for sequential colimits. To show this closure property, we need to assume a choice principle:

% \begin{definition}
%   For any type $X$, we say it is projective if for any type family $P$ over $X$,
%   \[ \prod_{x:X} \pss{P(x)} \to \pss{\prod_{x:X}P(x)}. \]
% \end{definition}

% \begin{axiom}[PF]
%   $\I$ is projective.
% \end{axiom}

% \begin{remark}
%   The projective objects will be closed under finite products and retracts. Thus under (PF), the cubes $\I^n$ and the simplices $\Delta^n$ will also be projective.
% \end{remark}

% \begin{lemma}[NT, PF, SQCF]\label{lem:sequentialcolimit}
%   $\prod_{i:\N}$ commutes with sequential colimit for all $k$-truncated type families: If $P_{i,n}$ is a $k$-truncated type family with transition maps $\tau_{i,n} : P_{i,n} \to P_{i,n+1}$ for all $i:\I,n : \N$, then the canonical map
%   \[ \ct_{n:\N}\prod_{i:\I} P_{i,n} \to \prod_{i:\I}\ct_{n:\N}P_{i,n} \]
%   is an equivalence, and similarly for $\I^n,\Delta^n$.
% \end{lemma}
% \begin{proof}
%   We prove this by induction on the truncation level. For $P_{i,n}$ contractible, by~\citep[Lem. 7.2]{10.1145/3373718.3394801} both $\ct_{n:\N}\prod_{i:\I} P_{i,n}$ and $\prod_{i:\I}\ct_{n:\N}P_{i,n}$ are contractible, hence an equivalence.

%   Now suppose $P$ is a family of $k+1$-types. Given two canonical elements $[(0,f)],[(0,g)]$ in the colimit $\ct_{n:\N}\prod_{i:\I}P_{i,n}$, recall from~\cite{10.1145/3373718.3394801}, we have
%   \[ [(0,f)] \cong [(0,g)] \cong \ct_{n:\N} \tau_nf \cong \tau_ng \cong \ct_{n:\N} \prod_{i:\I} \tau_nfi \cong \tau_ngi. \]
%   On the other hand, we have
%   \[ \prod_{i:\I}[(0,fi)] \cong [(0,gi)] \cong \prod_{i:\I}\ct_{n:\N}\tau_nfi \cong \tau_ngi. \]
%   The two types are equivalent by induction hypothesis, since the identity types are $k$-truncated. This means the map is an embedding.

%   To show surjectivity, suppose we have $f : \prod_{i:\I}\ct_{n:\N}P_{i,n}$. Now by construction, 
%   \[ \fa i\I \ex n\N \ex y{P_{i,n}} f(i) \cong [(n,y)]. \]
%   By (PF), and \Cref{cor:connectedpreservediscrete}, we merely get a section $f : \prod_{i:\I} P_{i,n}$ for some $n$, which shows surjectivity.
% \end{proof}

% \begin{remark}
%   Notice that the first half of the proof of \Cref{lem:sequentialcolimit} does not rely on any properties of $\I$, hence this shows that for \emph{any} type $X$,
%   \[ \ct_{n:\N}\prod_{x:X} P_{x,n} \to \prod_{x:X}\ct_{n:\N}P_{x,n} \]
%   is an embedding, for $P$ a $k$-truncated family.
% \end{remark}

% \begin{proposition}[NT, PF, SQCF]
%   For $f : X \to Y$, if both $X,Y$ are finite colimits of cubes or simplices, then $k$-truncated $f$-local objects are closed under sequential colimits for all $k$.
% \end{proposition}
% \begin{proof}
%   The proof is similar to \Cref{prop:closuresum}, by using \Cref{lem:sequentialcolimit} and realising that finite limits commutes with sequential colimits.
% \end{proof}

% \begin{example}
%   As mentioned in \Cref{sec:infdomain}, $\omega \cong \ct_{n:\N}\Delta^n$ is the sequential colimit of the simplices. Hence, assuming (NT), (PF), and (SQCC), $\omega$ will also be a poset.
% \end{example}

\section{A discussion on the models}\label{sec:model}

It is now instructive to discuss models for the axioms we have used in the previous sections. These axioms can be organised into two classes:

\begin{enumerate}
  \item The quasi-coherence principle (SQCF), (SQCC);
  \item Various locality axioms discussed in \Cref{sec:locality}.
\end{enumerate}

For any Horn theory $\T$, through the work in~\cite{blechschmidt2020general,blechschmidt2021using} we know that the classifying topos $\Set[\T] \cong [\mmod\T\fp,\Set]$, with the generic $\T$-model $U_\T$, models quasi-coherence for finitely presented $U_\T$-algebras, viz. (SQCF). As mentioned in \Cref{subsec:qc}, the proof for quasi-coherence for finitely presented $U_\T$-algebras can be completely adapted to the countably presented case, provided we work with a larger site,
\[ \Set[\T]_\omega := [\mmod\T\cp,\Set], \]
where $\mmod\T\cp$ is the category of c.p.\ $\T$-models. In other words, (SQCC) holds in the larger topos $\Set[\T]_\omega$. And in this case, we also allow $\T$ to contain algebraic operations of countable arity.

These two topoi are intimately related. There is a fully faithful and left exact inclusion of site
\[ \mmod\T\fp\op \hook \mmod\T\cp\op, \]
Hence, it induces an adjoint triple (cf.~\cite[Thm. 7.20]{caramello2019denseness}),
\[\begin{tikzcd}
  {\Set[\T]_\omega} & {\Set[\T]}
  \arrow[""{name=0, anchor=center, inner sep=0}, "\Gamma"{description}, from=1-1, to=1-2]
  \arrow[""{name=1, anchor=center, inner sep=0}, "\Delta"', curve={height=18pt}, from=1-2, to=1-1]
  \arrow[""{name=2, anchor=center, inner sep=0}, "\nabla", curve={height=-18pt}, from=1-2, to=1-1]
  \arrow["\dashv"{anchor=center, rotate=-90}, draw=none, from=0, to=2]
  \arrow["\dashv"{anchor=center, rotate=-90}, draw=none, from=1, to=0]
\end{tikzcd}\]
equivalently, a \emph{local geometric morphism} $\Set[\T]_\omega \surj \Set[\T]$ as in~\cite[C3.6]{johnstone2002sketches}.

More generally, it is already observed in~\cite[Thm. 4.11.]{blechschmidt2020general} that if we have a topology $J$ on $\mmod\T\fp\op$ where $U_\T$ is a $J$-sheaf, then (SQCF) again holds in the sheaf subtopos $\sh(\mmod\T\fp\op,J)$. We call such a topology $J$ \emph{admissible}. For instance, since $U_\T$ is representable, any subcanonical topology will in particular be admissible. Completely similarly, for any admissible topology $J$ on $\mmod\T\cp\op$, the sheaf subtopos $\sh(\mmod\T\cp\op,J)$ also models (SQCC). As an example, this is the theoretical basis of the quasi-coherence principle for countably presented Boolean algebras in the topos of light condensed sets introduced by Clausen and Scholze, as shown in~\cite{cherubini2024foundation}.

An admissible topology $J$ on $\mmod\T\cp\op$ restricts to an admissible one on $\mmod\T\fp\op$. In this case, the adjoint triple mentioned above between the two presheaf topoi will restrict to the sheaf subtopoi,
\[\begin{tikzcd}
  {\sh(\mmod\T\cp\op,J)} & {\sh(\mmod\T\fp\op,J)}
  \arrow[""{name=0, anchor=center, inner sep=0}, "\Gamma"{description}, from=1-1, to=1-2]
  \arrow[""{name=1, anchor=center, inner sep=0}, "\Delta"', curve={height=18pt}, from=1-2, to=1-1]
  \arrow[""{name=2, anchor=center, inner sep=0}, "\nabla", curve={height=-18pt}, from=1-2, to=1-1]
  \arrow["\dashv"{anchor=center, rotate=-90}, draw=none, from=0, to=2]
  \arrow["\dashv"{anchor=center, rotate=-90}, draw=none, from=1, to=0]
\end{tikzcd}\]
which again identifies $\sh(\mmod\T\cp\op,J)$ a local topos over $\sh(\mmod\T\fp\op,J)$. These admissible topologies on $\mmod\T\fp$ or $\mmod\T\cp$ are exactly the required data to validate various local properties of the generic model $U_\T$. 

More specifically for us, let $\sFrm$ be the category of $\sigma$-frames, i.e. whose objects are posets with finite meets and countable joins, where they distribute through each other. We will also use $\sFrm\cp$ to denote the full subcategory of countably presented $\sigma$-frames, and $\sFrm\fp$ to denote the finitely presented ones. Since f.p.\ $\sigma$-frames are simply f.p.\ distributive lattices, we have an isomorphism $\sFrm\fp \cong \DL\fp$.

It is well-known the dual category of $\DL\fp \cong \sFrm\fp$ is $\Pos\fp$ of finite posets. In this case, we can also have a fairly explicit description of the larger dual category $\sFrm\cp$. The first observation is that any c.p.\ $\sigma$-frame $A$ is indeed a \emph{frame}, i.e. it has \emph{all} joins and finite meets, which distributes with each other. This is clear for all finitely presented $\sigma$-frames, since they are simply finite distributive lattices. To see this for the countable case, consider the countably generated free $\sigma$-frame $2[\N]$:

\begin{lemma}\label{lem:cgfreesframe}
  We have an isomorphism
  \[ 2[\N] \cong [P_f(\N),2], \]
  where $P_f(\N)$ is the poset of finite subsets of $\N$.
\end{lemma}
\begin{proof}
  The free $\sigma$-frame can be generated by first freely adding finite meets to the discrete poset $\N$, and then freely adding all countable joins. The first step results in the poset $P_f(\N)\op$. Now since $P_f(\N)\op$ is countable, freely adding all countable joins is equivalently freely adding \emph{all} joins, which is achieved by the presheaf construction. This way,
  \[ 2[\N] \cong [(P_f(\N)\op)\op,2] \cong [P_f(\N),2]. \qedhere \]
\end{proof}

\begin{corollary}\label{cor:dualsframe}
  There is a fully faithful embedding
  \[ \sFrm\cp \hook \Frm, \]
  preserving all countable colimits. This is again fully faithful when composed with $\ms{pt} : \Frm\op \to \Topp$. 
\end{corollary}
\begin{proof}
  Any c.p.\ $\sigma$-frame will be isomorphic to one of the form $2[\N]/R$ for some countably generated equivalence relation $R$. By \Cref{lem:cgfreesframe} $2[\N]$ is a frame, so is any of its quotient. By~\cite[Thm. 6.2.4]{makkai2006first}, such frames are indeed \emph{spatial}, hence they fully faithfully embed into the category of topological spaces via the functor $\pt$.
\end{proof}

This way, we can view $\sFrm\cp\op$ as a certain class of topological spaces. Notice that since all c.p.\ $\sigma$-frame will be a quotient of $2[\N]$, their dual spaces will be a subspace of $\pt(2[\N])$, which we can compute quite easily:

\begin{lemma}
  The space of points of $2[\N]$ is the Scott's graph model $G$, which is the countable product of the Sierpi\'nski space $G \cong \Sigma^\N$.
\end{lemma}
\begin{proof}
  Note $\pt$ takes colimits in $\Frm$ to limits in $\Topp$, since $\pt : \Loc \to \Topp$ is a right adjoint. Since $2[\N]$ is the countable coproduct of the free frame on one generator, we have
  \[ \pt(2[\N]) \cong \pt(2[x])^\N \cong \Sigma^\N, \]
  Here $\pt(2[x]) \cong \Sigma$ follows from a simple computation.
\end{proof}

This way, if we write $\wTop$ as the essential image of the fully faithful functor $\pt : \sFrm\cp\op \to \Topp$, its objects will all be subspaces of $G$. We would then have the following diagram,
\[
\begin{tikzcd}
  \sFrm\fp\op \ar[d, hook] \ar[r, "\simeq"] & \Pos\fp \ar[d, hook] \\
  \sFrm\cp\op \ar[r, "\simeq"'] & \wTop
\end{tikzcd}
\]
where here the inclusion $\Pos\fp \hook \wTop$ simply take each finite poset to its Alexandroff topological space. Hence, at the presheaf level, we have a local geometric morphism
\[ \psh(\wTop) \surj \psh(\Pos\fp). \]
where in each case, the generic $\sigma$-frame $\I$ as a presheaf is now the representable functor on the Sierpinski space $\Sigma$. 

Below we discuss the corresponding admissible topologies on $\wTop$, modelling the various locality principles we have considered in \Cref{sec:locality}. We encourage the readers to notice the connection between the topologies we discuss below, and the developments in \Cref{sec:local}.

\begin{example}[NT]
  To model (NT), we would want the empty sieve on $\pt(2/0=1) \cong \emp$ to be a covering. Since $\emp$ is a strict initial object in $\wTop$, this is a subcanonical topology. Thus, this gives us a least topology $J_{\ms{NT}}$ that models (NT), and we have
  \[ \sh(\wTop,J_{\ms{NT}}) \cong \psh(\wTop_+), \]
  where $\wTop_+$ is the full subcategory of $\wTop$ excluding $\emp$. The induced local geometric morphism now is given by 
  \[ \Gamma : \psh(\wTop_+) \surj \psh(\Pos_{\mr{f.p.,+}}), \]
  where, again $\Pos_{\mr{f.p.,+}}$ is the full subcategory of $\Pos_{\mr{f.p.,+}}$ consisting of non-empty posets, and $\psh(\Pos_{\mr{f.,+}})$ is the classifying topos for non-trivial distributive lattices.
\end{example}

\begin{example}[L]
  The additional axiom for (L) besides (NT) is that 
  \[ x \vee y = 1 \vdash x = 1 \vee y = 1, \] 
  This means the dual embeddings of the following quotients needs to be a covering family,
  \[ 2[y] \cong 2[x,y]/x \twoheadleftarrow 2[x,y]/x\vee y \surj 2[x,y]/y \cong 2[x]. \]
  It is easy to see that $\pt(2[x,y]/x \vee y)$ is the space $\Lambda^2_1$ obtained by glueing the open points of two copies of Sierpinski spaces together, i.e. it is the following pushout,
  \[
  \begin{tikzcd}
    1 \ar[d, "1"'] \ar[r, "1"] & \Sigma \ar[d, "l"] \\ 
    \Sigma \ar[r, "r"'] & \Lambda^2_1
    \arrow["\lrcorner"{anchor=center, pos=0.125, rotate=180}, draw=none, from=2-2, to=1-1]    
  \end{tikzcd}
  \]
  The least topology $J_{\ms L}$ is thus generated by the empty covering on $\emp$, and the covering $\set{l,r : \Sigma \to \Lambda^2_1}$. Since we have the pushout above, this topology is again subcanonical. 
  
  In this case, it is not hard to give an explicit description of a covering family in $\Pos\fp$: A family of maps is a $\ms J_{\ms L}$ covering on $P \in \Pos\fp$ iff it contains a (finite) subfamily of \emph{upward closed subsets} $\set{P_i\subseteq P}_{i\in I}$, where $P \cong \bigcup_{i\in I}P_i$; see a similar calculation for commutative rings in~\cite[VIII. 6]{maclane2012sheaves}. Following the terminology for algebraic geometry, this can be denoted as the \emph{Zariski topology}, and the local geometric morphism
  \[ \sh(\wTop,J_{\ms L}) \surj \sh(\Pos\fp,J_{\ms L}) \simeq \mb{Zar}(\mbb D), \]
  is over the \emph{Zariski topos} $\mb{Zar}(\mbb D)$ for distributive lattices.
\end{example}

\begin{example}[SL]
  An even stronger axiom is the linearity axiom, 
  \[ \top \vdash x \le y \vee y \le x, \]
  which requires the dual embeddings of the following quotients to form a covering family,
  \[ 2[x,y]/x \le y \twoheadleftarrow 2[x,y] \surj 2[x,y]/y \le x. \]
  We can compute that $\spec 2[x,y] \cong \Sigma^2$, and 
  \[ \pt(2[x,y]/x \le y) \cong \Sigma_2 \cong \set{0 < 1 < 2}. \]
  Again we have a pushout diagram,
  \[
  \begin{tikzcd}
    \Sigma \ar[r, "\partial"] \ar[d, "\partial"'] & \Sigma_2 \ar[d, "l"] \\ 
    \Sigma_2 \ar[r, "r"'] & \Sigma^2
    \arrow["\lrcorner"{anchor=center, pos=0.125, rotate=180}, draw=none, from=2-2, to=1-1]
  \end{tikzcd}
  \]
  where $\partial : \Sigma \to \Sigma_2$ maps $\Sigma$ to the end points of $\Sigma_2$, and $l,r : \Sigma_2 \hook \Sigma^2$ is the two $\Sigma_2$-chains in $\Sigma^2$. The least topology $\ms J_{\ms{SL}}$ for the linearity axiom is thus generated by the empty covering on $\emp$, and $\set{l,r : \Sigma_2 \hook \Sigma^2}$. Similarly, given this pushout, $J_{\ms{SL}}$ will be subcanonical.

  The topos $\sh(\wTop,J_{\ms{SL}})$ is closely related to simplicial sets. This is due to the fact that simplicial sets is the classifying topos of non-empty linear orders (cf. e.g.~\cite[VIII. 8]{maclane2012sheaves}), and thus we have an equivalence 
  \[ \sh(\Pos\fp,J_{\ms{SL}}) \cong \psh(\Delta). \]
  This makes $\sh(\wTop,J_{\ms{SL}}) \surj \psh(\Delta)$ a local topos over simplicial sets.
\end{example}

\begin{example}[1T]\label{exm:model1T}
  Similarly to the case of (SL), the 1-truncation axiom requires dual embeddings of the following quotient maps
  \[\begin{tikzcd}
    & {2[x,y]/x\le y} \\
    {2[x,y]/x=0} & {2[x,y]/x=y} & {2[x,y]/y=1}
    \arrow[two heads, from=1-2, to=2-1]
    \arrow[two heads, from=1-2, to=2-2]
    \arrow[two heads, from=1-2, to=2-3]
  \end{tikzcd}\]
  to be a covering, which translates to the fact that the three inclusions
  \[ d_0,d_1,d_2 : \Sigma \hook \Sigma_2 \]
  covers $\Sigma_2$. Again we have a colimit diagram as follows, 
  \[\begin{tikzcd}
    1 & 1 & 1 \\
    \Sigma & \Sigma & \Sigma \\
    & {\Sigma_2}
    \arrow["0"', from=1-1, to=2-1]
    \arrow["1"{description, pos=0.25}, from=1-1, to=2-2]
    \arrow["0"{description, pos=0.25}, from=1-2, to=2-1]
    \arrow["1"{description, pos=0.25}, from=1-2, to=2-3]
    \arrow["0"{description, pos=0.25}, from=1-3, to=2-2]
    \arrow["0", from=1-3, to=2-3]
    \arrow["{d_0}"', from=2-1, to=3-2]
    \arrow["{d_1}"{description}, from=2-2, to=3-2]
    \arrow["{d_2}", from=2-3, to=3-2]
  \end{tikzcd}\]
  which implies $J_{\ms{1T}}$ is subcanonical.
  
  Similar to the case of (SL), $\sh(\wTop,J_{\ms{1T}})$ is closely related to the topos of \emph{truncated simplicial sets} $\psh(\Delta_{\le 1})$. This is again induced by the fact that the classifying topos for 1-truncated distributive lattices is given by 
  \[ \sh(\Pos\fp,J_{\ms{1T}}) \cong \psh(\Delta_{\le 1}). \]
  Hence, this again gives us a local geometric morphism
  \[ \sh(\wTop,J_{\ms{1T}}) \surj \psh(\Delta_{\le 1}). \]
\end{example}


\section{Future directions}

\subsection{Domain theory with quasi-coherence}

In this paper we have explained the axioms of synthetic domain theory using the quasi-coherence principle for distributive lattices and $\sigma$-frames. As we have seen from various examples (\Cref{prop:liftingofalgebra,prop:liftofaffine,prop:complementinalgebra}, etc.) the quasi-coherence principle moreover helps with the actual \emph{computation} for operations on domains. Thus, the natural step next is to further develop domain theory within this framework.

% In particular, it will be extremely useful to investigate the orthogonality conditions satisfied by all spectrums (affine spaces), some of which have been observed in \Cref{sec:infdomain,sec:local}. In particular, in this paper we haven't found any orthogonality condition satisfied by $\I$ but not by general affine spaces. This allows us to conjecture that all (or at least some classes of) affine spaces will be \emph{replete}. This will be a non-trivial way of constructing new replete objects other than retracts of $\I^X$ for some type $X$.

\subsection{Quasi-coherence for related synthetic mathematics}

Both synthetic topology~\cite{bauer2009dedekind} and synthetic computability theory~\cite{RN552} involve similar structures as synthetic domain theory. In particular, an interval object $\I$ consisting of a dominance seems to be crucial in all three cases. We have showcased in \Cref{sec:dominance} that quasi-coherence for a wide variety of theories will produce such a structure, and it will be interesting to see connections with quasi-coherence to these two types of synthetic mathematics as well.

\subsection{Connection with existing models}\label{subsec:compare}

Finally, it will be instructive to compare the sheaf models for synthetic domain theory constructed in~\cite{FIORE1997151} with the models discussed in \Cref{sec:model}. Recall the topos $\mc H$ constructed in \emph{loc. cit.} is the following sheaf topos,
\[ \mc H := \sh(\mb P,J_{\ms{can}}), \]
where $\mb P$ be the full subcategory of the category $\wCPO$ of $\omega$-cpos consisting of retracts of the Scott's graph model $G$, and $J_{\ms{can}}$ is the canonical topology on $\mb P$. An $\omega$-cpo is simply a poset having countable joins, and a morphism between them are monotone maps preserving the countable joins.

Here we observe that the inclusion $\mb P \hook \wTop$ is \emph{fully faithful}, because
\begin{align*}
  &\wCPO(G,G) \\
  \cong& \Pos(P_f(\N)\op,[P_f(\N),2]) \\
  \cong& \sFrm\cp([P_f(\N),2],[P_f(\N),2]) \\ 
  \cong& \wTop(G,G) \qedhere
\end{align*}
This leads us to conjecture that $\mc H$ will be equivalent to the sheaf topos $\sh(\wTop,J_{\mr{can}})$. However, a detailed proof of this requires us to better understand the categorical properties of $\wTop$.

% \begin{theorem}\label{thm:comparemodel}
%   The embedding $\mb P \hook \wTop$ induces an equivalence
%   \[ \mc H \simeq \sh(\mb P,J_{\mr{can}}) \simeq \sh(\wTop,J_{\mr{can}}), \]
% \end{theorem}
% \begin{proof}
%   It suffices to show the inclusion induces a dense morphism of site 
%   \[ (\mb P,J_{\mr{can}}) \hook (\wTop,J_{\mr{can}}). \]
%   Following~\cite{caramello2019denseness},
%   Note that the direct image $i_*$ is computed by right Kan extension, thus $i$ is an embedding because $i$ is fully faithful, and $i^*i_* \cong 1$. The generic model is the corepresentable $\yon^{2[\N]}$, which when restricting along $i^*$ becomes the representable $\yon_G$. Hence, it suffices to show $i_*\yon_G \cong \yon^{2[\N]}$. 
  
%   For any $A\in\sFrm\cp$, we compute
%   \[ i_*\yon_G(A) \cong \lt_{X\in A/\mb P}\mb P(X,P) \cong \lt_{D\in A/\mb P}\sFrm(2[\N],D). \]
%   Now we show the canonical map is an isomorphism:
%   \[ \sFrm(2[\N],A) \to \lt_{D\in A/\mb P}\sFrm(2[\N],D). \]
  
%   \emph{Injectivity}: Suppose we have two maps $f,g : 2[\N] \to A$. Notice that the singleton $1\in\mb P$, which corresponds to the initial frame $2$. Since $A \in \sFrm$ is spatial, if for all $x : A \to 2$ we have $xf \cong xg$, then $f \cong g$.

%   \emph{Surjectivity}: Suppose we are given a compatible family $\set{f_D}_{D\in A/\mb P}$ where $f_D : 2[\N] \to D$. By the duality in \Cref{cor:dualsframe}, equivalently we have a compatible family
%   \[ \scomp{f_X : X \to G}{X \in \mb P/\pt(A)}. \]
%   Now there is essentially one way to define a map $f : \pt(A) \to G$, by letting $f(x)$ for $x\in\pt(A)$ to be $f_x$,
%   \[
%   \begin{tikzcd}
%     1 \ar[d, "f_x"'] \ar[r, "x"] & \pt(A) \ar[dl, dashed, "f"] \\ 
%     G
%   \end{tikzcd}
%   \]
% \end{proof}

% To simplify the comparison, following~\cite[Thm. 3]{FIORE1997151} we use the simpler presentation of $\mc H$ as $\sh(\mb L,J_{\mr{can}})$, where $\mb L$ is the full subcategory of $\wCPO$ consisting of retracts of $\ov\omega$. As a full subcategory $\mb L \hook \mb P$, we again 

\bibliographystyle{apalike} 
\bibliography{mybib}

\appendix

\section{A normal form result}

\begin{lemma}\label{app:normalsigma}
  For the countably presented $\sigma$-frame $\I[\N]/\pair{i_n \ge i_{n+1}}_{n:\N}$, an element $p$ can be uniquely written as 
  \[ p = \bigvee_{n:\N} p_n \wedge i_n, \]
  with $p_n \le p_{n+1}$ for all $n$. In particular, $\I[\N]/\pair{i_n \ge i_{n+1}}_{n:\N}$ is isomorphic to the $\sigma$-frame $\Delta_\infty$ with the pointwise order.
\end{lemma}
\begin{proof}
  We directly prove that $\Delta_\infty$ satisfies the universal property of the c.p.\ $\sigma$-frame $\I[\N]/\pair{i_n \ge i_{n+1}}_{n:\N}$. We pick the generators in $\Delta_\infty$ as follows,
  \[ i_n := (\underbrace{0,\cdots,0}_{n+1 \text{ times}},1,1,\cdots). \]
  For any $\sigma$-frame $A$ with $a_0 \ge a_1 \ge \cdots$, we define a map $f_a : \Delta_\infty \to A$ sending $j = (j^0,j^1,\cdots) : \Delta_\infty$ to
  \[ f_a(i) := j_0 \vee \bigvee_{n:\N} j^{n+1} \wedge a_{n}. \]
  By construction it is easy to see $f_a$ is a $\sigma$-frame morphism. Evidently for any $n : \N$, 
  \[ f_a(i_n) = \bigvee_{m:\N} i_n^{m+1} \wedge a_m = a_n. \]
  Furthermore, for any $\sigma$-frame map $f : \Delta_\infty \to A$ with $f(i_n) = a_n$, we must have $f = f_a$ because any $j$ in $\Delta_\infty$ can be written as
  \[ j = j^0 \vee \bigvee_{n:\N} j^{n+1} \wedge i_n, \]
  which implies that
  \[ f(j) = j^0 \vee \bigvee_{n:\N}j^{n+1} \wedge a_n = f_a(j). \]
  This completes the proof.
\end{proof}



% \section{Normal form results}

% \begin{lemma}
%   For a distributive lattice $A$, a polynomial $p$ in $A[i]$ is uniquely determined by two parameters $p_0 \le p_1$, with
%   \[ p = p_0 \vee i \wedge p_1. \]
% \end{lemma}

% \begin{lemma}
%   For a distributive lattice $A$, a polynomial $p$ in $A[n]/i_1 \ge \cdots \ge i_n$ is uniquely determined by $p_0 \le \cdots \le p_n$, with
%   \[ p = p_0 \vee i_1 \wedge \cdots \wedge i_n \vee p_n, \]
% \end{lemma}
% \begin{proof}
%   We directly show that $\Delta[A]_{n+1}$ with the pointwise distributive lattice structure satisfies the universal property of $A[n]/i_1 \ge \cdots \ge i_n$, where generators are given by 
%   \[ (0,1,\cdots,1), (0,0,1,\cdots,1), \cdots, (0,0,\cdots,1). \]
%   Notice that any element $p = (p_0,\cdots,p_n)$ in $\Delta[A]_{n+1}$ can be written as
%   \[ p = p_0 \vee (0,1,\cdots,1) \wedge (0,0,1,\cdots,1) \vee \cdots \vee (0,\cdots,1) \wedge p_n. \]
%   For any distributive lattice $B$ and a homomorphism $f : A \to B$, given $a \ge b$ in $B$ we construct a map
%   \[ f_{a,b} : \Delta[A]_3 \to B, \]
%   such that for $p_0 \le p_1 \le p_2$ in $\Delta[A]_3$, we define
%   \[ f_{a,b}(p_0,p_1,p_2) := f(p_0) \vee a \wedge f(p_1) \vee b \wedge f(p_2). \]
%   This is well-defined. Furthermore,
%   \begin{align*}
%     &f_{a,b}(0,1,1) = 0 \vee a \wedge 1 \vee b \wedge 1 = a, \\
%     &f_{a,b}(0,0,1) = 0 \vee a \wedge 0 \vee b \wedge 1 = b.
%   \end{align*}
%   Hence, it indeed maps the generators to $a,b$, respectively. Finally, to show it is unique, we observe that for any $p_0 \le p_1 \le p_2$, we have
%   \[ (p_0,p_1,p_2) = p_0 \vee (0,1,1) \wedge p_1 \vee (0,0,1) \wedge p_2. \]
%   Hence, $f_{a,b}$ is uniquely determined as above.
% \end{proof}

\end{document}


% However, to connect to models in~\cite{FIORE1997151}, we need to connect $\sFrm\cp$ with $\wCPO$. To do this, we first observe the following fact:

% \begin{proposition}
%   Let $\mb P$ be the full subcategory of $\wCPO$ consisting of retracts of Scott's graph model $G$. Then we have a fully faithful embedding 
%   \[ i : \mb P \hook \sFrm\cp\op. \]
% \end{proposition}
% \begin{proof}
%   Since both $\wCPO$ and $\Topp$ are Cauchy complete, it suffices to notice that for the graph model $G$, by viewing it as an $\omega$CPO or a topological space, they give the same endomorphisms,
%   \[ \wCPO(G,G) \cong \Topp(G,G). \]
%   Hence, this follows from \Cref{cor:dualsframe}.
% \end{proof}

% \begin{theorem}
%   The inclusion $i : \mb P \hook \sFrm\cp\op$ induces an embedding,
%   \[ i : \psh(\mb P) \hook [\sFrm\cp,\Set], \]
%   where the generic model in $[\sFrm\cp,\Set]$ is an $i$-sheaf.
% \end{theorem}
% \begin{proof}
%   Note that the direct image $i_*$ is computed by right Kan extension, thus $i$ is an embedding because $i$ is fully faithful, and $i^*i_* \cong 1$. The generic model is the corepresentable $\yon^{2[\N]}$, which when restricting along $i^*$ becomes the representable $\yon_G$. Hence, it suffices to show $i_*\yon_G \cong \yon^{2[\N]}$. 
  
%   For any $A\in\sFrm\cp$, we compute
%   \[ i_*\yon_G(A) \cong \lt_{X\in A/\mb P}\mb P(X,P) \cong \lt_{D\in A/\mb P}\sFrm(2[\N],D). \]
%   Now we show the canonical map is an isomorphism:
%   \[ \sFrm(2[\N],A) \to \lt_{D\in A/\mb P}\sFrm(2[\N],D). \]
  
%   \emph{Injectivity}: Suppose we have two maps $f,g : 2[\N] \to A$. Notice that the singleton $1\in\mb P$, which corresponds to the initial frame $2$. Since $A \in \sFrm$ is spatial, if for all $x : A \to 2$ we have $xf \cong xg$, then $f \cong g$.

%   \emph{Surjectivity}: Suppose we are given a compatible family $\set{f_D}_{D\in A/\mb P}$ where $f_D : 2[\N] \to D$. By the duality in \Cref{cor:dualsframe}, equivalently we have a compatible family
%   \[ \scomp{f_X : X \to G}{X \in \mb P/\pt(A)}. \]
%   Now there is essentially one way to define a map $f : \pt(A) \to G$, by letting $f(x)$ for $x\in\pt(A)$ to be $f_x$,
%   \[
%   \begin{tikzcd}
%     1 \ar[d, "f_x"'] \ar[r, "x"] & \pt(A) \ar[dl, dashed, "f"] \\ 
%     G
%   \end{tikzcd}
%   \]
% \end{proof}

% More specifically for us, let us focus on the case for $\mbb D$ the theory of distributive lattices. The same discussion more generally applies to the theory of $D$-algebras for any distributive lattice $D$. Let us first observe that we have a more detailed description of the dual category $\DL\cp\op$. Recall that there is a duality between distributive lattices and \emph{spectral spaces},\footnote{If one want a fully constructive development, it suffices to replace spectral spaces with coherent locales (cf.~\cite{johnstone1982stone}).}
% \[ \spec : \DL\op \cong \Spec : \Sigma. \]
% Here $\Spec$ is the category of spectral topological spaces and spectral maps. A spectral space $X$ is a sober topological space, such that its compact open subsets form a basis, i.e. they are closed under finite intersections and every open subset is a union of compact ones. A continuous map between spectral spaces $f : X \to Y$ is called \emph{spectral}, if $f\inv$ also preserves compact opens. For more information on this duality result, see e.g. Stone's original paper~\cite{stone1938topological} or the detailed exposition~\cite{bezhanishvili2010bitopological}.

% On one hand, $\spec$ takes a distributive lattice $A$ to its \emph{space of models}:
% \begin{itemize}
%   \item A point is a morphism $x : A \to 2$ in $\DL$;
%   \item Subsets of the form $D_a$ with $a\in A$ form a basis, with 
%   \[ D_a := \scomp{x \in \spec A}{x(a) \cong 1}. \]
% \end{itemize}
% It is easy to verify that the basis of the form $D_a$ are in fact compact, and hence $\spec A$ is a spectral space. Given a morphism $f : A \to B$, there is an induced map
% \[ f^* : \spec B \to \spec A, \]
% given by precomposition with $f$. One can verify $(f^*)\inv(D_a) \cong D_{fa}$, thus $f^*$ is spectral. We mention that though $\Spec$ is not a full subcategory of $\Topp$, it is still closed under limits in $\Topp$. 

% As an example, let us look at the space of models of the countably generated free distributive lattice $2[\N]$, where we have 
% \[ \spec 2[\N] \cong \Sigma^\N. \] 
% Here $\Sigma$ is the \emph{Sierpinski space}. This is due to the fact that
% \[ \spec 2[\N] \cong \spec\coprod_{\N} 2[x] \cong \prod_{\N} \spec 2[x] \cong \Sigma^\N, \]
% where it is easy to verify that $\spec 2[x] \cong \Sigma$. 

% On the other hand, the functor $\Sigma : \Spec \to \DL\op$ simply takes a spectral space $X$ to the distributive lattice 
% \[ \Sigma(X) := \Spec(X,\Sigma), \]
% In other words, $\Sigma(X)$ is the distributive lattice of compact opens of $X$.

% Let us now consider the essential image of $\spec : \DL\cp\op \to \Spec$, which we denote as $\wTop$. Notice that since $2[\N]$ is itself countable, a distributive lattice is countably presented iff it is a quotient of $2[\N]$. This way, by duality objects in $\wTop$ are simply spectral subspaces of $\Sigma^\N$, i.e. those subspaces $X \subseteq \Sigma^\N$, where for any compact open $U$ of $\Sigma^\N$, $X \cap U$ is again compact in $X$. In particular, not all subspaces of $\Sigma^\N$ will lie in $\wTop$. For instance, the discrete space $\N$ as a subspace of $\Sigma^\N$ is \emph{not} an object in $\wTop$, since $\N$ is not even compact, let alone spectral.

% This way, we have the following situation,
% \[
% \begin{tikzcd}
%   \DL\fp\op \ar[r, "\cong"] \ar[d, hook] & \Pos\fp \ar[d, hook] \\ 
%   \DL\cp\op \ar[r, "\cong"] & \wTop 
% \end{tikzcd}
% \]
% where the equivalence $\DL\cp\op \cong \wTop$ restricts to one between finite distributive lattices and finite posets, where we view each finite poset as a spectral space via its Scott topology. Hence, at the presheaf level, we have a local geometric morphism
% \[ \psh(\wTop) \surj \psh(\Pos\fp). \]

% Working with $\wTop$, the generic distributive lattice $\I$ as a presheaf on $\wTop$ is now the representable functor on the Sierpinski space $\Sigma$. Below are the corresponding admissible topologies on $\wTop$ which model the various locality principles we have considered in \Cref{sec:locality}. We encourage the readers to notice the connection between the topologies we discuss below, and the developments in \Cref{sec:local}.